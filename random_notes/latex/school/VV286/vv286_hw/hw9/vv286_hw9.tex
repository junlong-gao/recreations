\input mla
%\usepackage{nath}
\begin{document}
\titlehm{9}
\begin{vv286_ms}{1}
\item[(i)]
 See the sketched counter in Figure \ref{c:1}.
 
% \begin{figure}[htbp]
%\begin{center}
%\includegraphics[width=0.6\textwidth]{1.jpg}
%
%\caption{Counter for the $f(z)=\sq{z}/(z^2+a^2)$}
%\label{default}
%\end{center}
%\end{figure}

\begin{figure}[htbp]
\begin{center}
\begin{tikzpicture}
% Configurable parameters
\def\gap{0.2}
\def\bigradius{3}
\def\littleradius{0.5}

% Axes
\draw [help lines,->] (-1.25*\bigradius, 0) -- (1.25*\bigradius,0);
\draw [help lines,->] (0, -1.25*\bigradius) -- (0, 1.25*\bigradius);
% Red path
\draw[line width=1pt,   decoration={ markings,
  mark=at position 0.2455 with {\arrow[line width=1.2pt]{>}},
  mark=at position 0.765 with {\arrow[line width=1.2pt]{>}},
  mark=at position 0.87 with {\arrow[line width=1.2pt]{>}},
  mark=at position 0.97 with {\arrow[line width=1.2pt]{>}}},
  postaction={decorate}]
  let
     \n1 = {asin(\gap/2/\bigradius)},
     \n2 = {asin(\gap/2/\littleradius)}
  in (\n1:\bigradius) arc (\n1:360-\n1:\bigradius)
  -- (-\n2:\littleradius) arc (-\n2:-360+\n2:\littleradius)
  -- cycle;

% The labels
\node at (3.6,-0.2){Re};
\node at (-0.24,3.53) {Im};
\node at (-0.8,0.5) {$C_{\varepsilon}$};
\node at (-1.8,3) {$C_{R}$};
\node at (1.9,0.4) {$\Gamma_1$};
\node at (1.9,-0.4) {$\Gamma_2$};


\end{tikzpicture}
\caption{Counter for the $f(z)=\sq{z}/(z^2+a^2)$}
\label{c:1}
\end{center}
\end{figure}

  Consider the function $f(z)={\sq{z}}/{(z^2+a^2)}$ (by choosing the branch cut of $\{0\}\cup\R^+$) and the path of integration along the greater
  circle $C_R$, rotating through the negative axis, then making a small circle turn $C_{\varepsilon}$ around the origin, in the graph the angle $\p$ is understood to be sufficiently small such that as $\p\to0$ the two line will finally parallel with the real axis, then:
  \eq
  {
  \int_{\Gamma_1}+\int_{\Gamma_2}+\int_{C_R}+\int_{C_{\epsilon}}=2\pi i(\res{f}{ia}+\res{f}{-ia})
  }
   Then the residues at $z=\pm ia$ (careful at our branch choice):
  \eq
  {
  &\res{f}{ia}=\ff{\sq{a}e^{\pi/4i}}{2ai}\\
  &\res{f}{-ia}=-\ff{\sq{a}e^{3\pi/4i}}{2ai}\\
  &\implies \res{f}{ia}+\res{f}{-ia}=\ff{\sq{2}}{2i\sq{a}}
  }
  On the greater and smaller circle we have estimate:
  \eq
  {
  \left|\int_{C_R}\right|\le 2\pi R \ff{\sq{R}}{R^2}\to 0\\
  \left|\int_{C_{\e}}\right|\le 2\pi \e \ff{\sq{\e}}{a^2}\to 0
  }
  the two straight line we have parameterization:
  \eq
  {
  z&=te^{i\p},\quad t:0\to\infty\\
  z&=te^{2\pi i-i\p}\quad t:\infty\to0
  }
  due to our branch, then the integral becomes:
  \eq
  {
  \int_{\Gamma_1}+\int_{\Gamma_2}&=\int_{0}^{\infty}\ff{\sq{t}e^{i\phi/2}}{t^2e^{2i\phi}+1}e^{i\phi}\,dt+\int_{\infty}^{0}\ff{\sq{t}e^{\pi i-i\phi/2}}{t^2e^{4\pi i-2i\phi}+1}e^{2\pi i-i\phi}\, dt\\
  &\to \int_{0}^{\infty}\ff{\sq{t}}{t^2+1}\dt
  +\int_{0}^{\infty}\ff{\sq{t}}{t^2+1}\dt\quad\text{as $\phi\to0$}
  \\&=2\int_{0}^{\infty}\ff{\sq{t}}{t^2+1}\dt
  }
   the desired result is:
\begin{align*}
    \int_0^{\infty}\frac{\sq x}{x^2+a^2}\, dx=\frac{\pi}{\sq a}\frac{\sq 2}{2}
  \end{align*}
  
 \item[(ii)]
 We first consider $a>0$, then the sketched counter:
 \begin{figure}[htbp]
\begin{center}
\begin{tikzpicture}[scale=2]
% The axes
\draw[help lines,->] (-3,0) -- (3,0) coordinate (xaxis);
\draw[help lines,->] (0,-1) -- (0,3) coordinate (yaxis);

% The path
\path[draw,line width=0.8pt,postaction=decorate] (0.3,0) node[below] {$\varepsilon$} -- (2,0) node[below] {$r$} arc (0:180:2) -- (-0.3,0) arc (180:0:0.3);

% The labels
\node[below] at (xaxis) {Re};
\node[left] at (yaxis) {Im};
\node[below left] {$O$};
\node at (0.2,0.6) {$C_{\varepsilon}$};
\node at (1.5,1.8) {$C_{r}$};
\end{tikzpicture}
\caption{Counter for the $f(z)={\ln z}/(z^2+a^2)$}
\label{c:2}
\end{center}
\end{figure}

   Consider the function $f(z)=\ff{ \ln  z}{z^2+a^2}$ (by choosing the branch cut of $\{0\}\cup i\R^-$)
   and the path of integration along the punctuated semi-circle centered at the origin with the
   greater arc $C_R$ and the smaller arc at the origin $C_{\varepsilon}$.\\
   By our branch, the logarithm on the negative axis are properly defined by parameterization:
   \begin{align*}
     \ln (-x)=\ln  e^{i\pi}t=i\pi+\ln (t)
     \quad\text{for $t:0^+\to+\infty$}
   \end{align*}
   and the corresponding integral on the negative axis is:
  \begin{align*}
    I^-=\int_{\Gamma^-}\frac{\ln x}{x^2+a^2}\, dx=\int_{+\infty}^{0}\frac{\ln (te^{i\pi})}{x^2+a^2}e^{i\pi}\, dx=
    I^{+}+\int_{0}^{+\infty}\frac{\pi i}{x^2+a^2}\, dx=I^++\frac{\pi^2i}{2a}
  \end{align*}
  also it's easy to check that the integral along the greater and smaller semi-circle vanishes due to the decay of $1/R^2$ at infinity and $\e\ln \e\to 0$ as $\e\to0$. We
  therefore have:
  \begin{align*}
    I^++I^-=2I^++\frac{\pi^2i}{2a}=\res{f}{ia}=\frac{\pi \ln a}{a}+\frac{\pi^2i}{2a}
  \end{align*}
  comparing the real part, thus the desired result is:
  \[
  \int_{0}^{+\infty}\frac{\ln x}{x^2+a^2}\, dx=\frac{\pi \ln a}{2a}
  \]
  
  If $a<0$, the corresponding counter will be chosen at the negative half plane and the desired result is
  \[
  \int_{0}^{+\infty}\frac{\ln x}{x^2+a^2}\, dx=\frac{\pi \ln (-a)}{2(-a)}
  \] 
  
  If $a=0$ the improper integral does not converge near $0^+$ and the integral diverges.\\
  In conclusion we have:
  \eq
  {
  \int_{0}^{+\infty}\frac{\ln x}{x^2+a^2}\, dx=\frac{\pi \ln |a|}{2|a|} \quad\text{when $a\neq0$,} \quad\text{and it diverges when $a=0$} 
  }
\end{vv286_ms}

\begin{vv286_ms}{2}
\item[]
First we notice:
\eq
{
y''+y'=f(x)\implies y=\ff{1}{1-(-D)^2}f(x)
}
and
\eq
{
\ff{1}{1-(-D)^2}=\sum_{k=0}(-1)^kD^{2k}
}
symbolically. 
\item[(i)]
We calculate:
\eq
{
D^0(3x+5x^4)&=3x+5x^4\\
D^2(3x+5x^4)&=60x\\
D^4(3x+5x^4)&=120x\\
D^{2k}(3x+5x^4)&=0,\quad k\ge3
}
Therefore:
  \[
  y=3x-60x^2+5x^4+120
  \]
  Plugging back indeed this is a solution. The operator method worked.
 \item[(ii)]
 Note:
 \eq
 {
 D^{2k}e^{\mu x}=\mu^{2k}e^{\mu x}
 }
 Therefore:
 
   \[
   y=\sum_{k=0}(-1)^k\mu^{2k}e^{\mu x}=\e^{\mu x}\frac{1}{1+\mu^2}
   \]
   Plugging back indeed this is a solution(formally on some domain). The operator method worked.
\end{vv286_ms}

\begin{vv286_ms}{3}
\item[(i)]
Notice:
\eq
{
\sinh(bt)=\ff{e^{bt}-e^{-bt}}{2}=i\sin{(bt/i)}
}
Therefore:
  \[
  L[\sinh(bt)](p)=\frac{b}{p^2-b^2}\quad p>|b|
  \]
 \item[(ii)]
 This follows from the table:
   \[
   L[\cos(bt)](p)=\frac{p}{p^2+b^2}\quad p>|b|
   \]
  \item[(iii)]
   This follows from the table:
   \eq
   {
   L[tf(t)](p)=-\ff{d}{dp}L[f(t)](p)
   }
   therefore:
    \begin{align*}
      L[t\sin(at)](p)=\frac{2pa}{(p^2+a^2)^2}\quad p>0
    \end{align*}
  \item[(iv)]
     This follows from the table:
   \eq
   {
   L[tf(t)](p)=-\ff{d}{dp}L[f(t)](p)
   }
   therefore:
    \begin{align*}
      L[t^2\sinh(bt)](p)=2b\left( \frac{1}{(p^2-b^2)^2}-\frac{4p^2}{(p^2-b^2)^3} \right)
      \quad p>|b|
    \end{align*}
   \item[(v)]
   By definition:
   \eq
   {
   L[\sq{t}](p)=\int_{0}^{\infty}t^{3/2-1}e^{-pt}\dt=\ff{1}{p\sq{p}}\int_{0}^{\infty}(tp)^{3/2-1}e^{-pt}\,dpt
   }
     Therefore:
     \begin{align*}
       L[\sq{t}](p)=p^{-3/2}\Gamma(3/2)=p^{-3/2}\ff{1}{\Gamma(1/2)}=\ff{\sq{\pi}}{2}p^{-3/2}
       \quad p>0
     \end{align*}
    \item[(vi)]
     By definition:
   \eq
   {
   L[1/\sq{t}](p)=\int_{0}^{\infty}t^{1/2-1}e^{-pt}\dt=\ff{\sq{p}}{p}\int_{0}^{\infty}(tp)^{1/2-1}e^{-pt}\,dpt
   }
   Therefore:
      \begin{align*}
	L[1/\sq{t}](p)=\ff{\Gamma(1/2)}{\sq{p}}=\sq{\frac{\pi}{p}}
	\quad p>0
      \end{align*}
\end{vv286_ms}

\begin{vv286_mp}{4}
\item[(i)]
  Let $y$ to be any solution and construct $Y(x)=y(-x)$ for
  $Y(0)=y(0)=1$, $Y'(0)=y'(0)=0$, and we
  have $Y'(x)=-y'(-x)$, $Y''(x)=y''(-x)$, and we can show that:
  \begin{align*}
    xY''+Y'+xY=-\left((-x)y''(-x)+y'(-x)+(-x)y(-x)\right)=0
  \end{align*}
  thus by uniqueness we must have $y(x)=Y(x)=y(-x)$
 \item[(ii)]
   Indeed, this can be shown by setting $z=a+bi$, where $a, b\in\R$, and
   solve for
   \begin{align*}
     z^2+1=a^2-b^2+1+2abi\in\R
   \end{align*}
   and this gives $a=0$ and $b\in[-1, \, 1]$ or $b=0$ and $a\in\R$.
   Combining the two gives $\Gamma$ in question.
 \item[(iii)]
 Consider the sketched in Figure \ref{bone}:

\begin{figure}[htbp]
\begin{center}
\begin{tikzpicture}[scale=2]
% The axes
\draw[help lines,->] (-3,0) -- (3,0) coordinate (xaxis);
\draw[help lines,->] (0,-3) -- (0,3) coordinate (yaxis);

% The path
\path[draw,line width=0.8pt,postaction=decorate] 
(0.1,0.2)node[below] {$\varepsilon$}  -- (1,0.2)node[ right] {$1+i\varepsilon$}
(1,0.2)  -- (1,3)
arc (90:162:4) -- (-2.8,0.2)node[left] {$-R+i\varepsilon$}
(-2.8,0.2)  -- (-0.1,0.2)
(-0.1,0.2)  -- (-0.1,0.8)
(-0.1,0.8) arc (240:-60:0.2) -- (0.1,0.8)
(0.1,0.8)-- (0.1,0.2)
;

\path[draw,line width=0.8pt,postaction=decorate] 
(0.1,-0.2)node[above] {$\varepsilon$}  -- (1,-0.2)node[right] {$1-i\varepsilon$}
(1,-0.2)  -- (1,-3)
arc (270:200:4) -- (-2.8,-0.2)node[left] {$-R-i\varepsilon$}
(-2.8,-0.2)  -- (-0.1,-0.2)
(-0.1,-0.2)  -- (-0.1,-0.8)
(-0.1,-0.8) arc (-240:+60:0.2) -- (0.1,-0.8)
(0.1,-0.8)-- (0.1,-0.2)
;

% The labels
\node[below] at (xaxis) {Re};
\node[left] at (yaxis) {Im};
\node[below left] {$O$};
\node at (0.05,1) {$i$};
\node at (0.05,-1) {$-i$};
\end{tikzpicture}
\caption{Counter for the $f(z)$ contracting to dogbone as $\e\to0$}
\label{bone}
\end{center}
\end{figure}

   Since $i$ and $-i$ are the only two branch points, we choose the branch
   cut :$\{z\in\C:\, \Re z=0, \, -1\le\Im z\le1\}$, and let the path of
   integration follows from the vertical line $\Re z=1$ and rotating to the
   left of a great quarter, when touching the negative axis then going back
   to the imaginary axis, then follows along the axis to the point
   $i$, rotating a smaller circle back to the origin and joining to the
   point infinitesimally close to the point $1$, a symmetry path is chosen
   for then lower half plane.\\
   It's easy to see that the integral for the great quarter and the small
   circle around $i$ vanish, and the path following along the real line
   cancels each other for the upper half and the lower half, and what
   remains is the pair of the paths from the branch points $i$ to
   $-i$. 
   We identify the square root:
   \eq
   {
   \sq{z^2+1}=e^{\ff{1}{2}\ln(z^2+1)}
   }
   and the branch chosen as $(0,2\pi i]$.\\
   For the left half plane, we see that by $z=ite^{2\pi^-i}$:
   \begin{align*}
     \int_{-i}^{i}\frac{e^{xz}}{\sq{z^2+1}}\, dz=
     \int_{-1}^{1}\frac{e^{xti}}{e^{i\pi}\sq{1-t^2}}\, idt
     =-\int_{-1}^{1}\frac{e^{xti}}{\sq{1-t^2}}\, idt
   \end{align*}
   due to our branch.\\
   And the for the right half plane, we see by $z=ite^{0^+i}$:
   \begin{align*}
     \int_{i}^{-i}\frac{e^{xz}}{\sq{z^2+1}}\, dz
     =-\int_{-1}^{1}\frac{e^{xti}}{\sq{1-t^2}}\, idt
   \end{align*}
   Therefore we conclude that since the two whole closed integrals have no
   residues, the total sum must vanish:
   \begin{align*}
     \int_{1-i\infty}^{1+i\infty}\frac{e^{xz}}{\sq{z^2+1}}-2\int_{-1}^{1}\frac{e^{xti}}{\sq{1-t^2}}i\,
     dt=0\numberthis{1}
   \end{align*}
   and we set the substitution: $t=\sin \ta$, the second term yields:
   \begin{align*}
     2i\int_{-\pi/2}^{\pi/2}\cos (x\sin \ta)+i\sin (x\sin \ta)\,
   d\ta=2i\int_{0}^{\pi}\cos (x\sin \ta)\, d\ta
   \end{align*}
   where we used the symmetry of $\sin  x$ on $[0, \pi]$ and it's odd on
   $[-\pi/2, \pi/2]$. Subsititude it into \eqref{1} and divide it by
   $2\pi i$ gives us $J_0(x)$. This completes the proof.
 \end{vv286_mp}

 \begin{vv286_ms}{5}
 \item[(i)]
   \begin{align*}
     y=-\frac{1}{6}e^t+\frac{1}{2}e^{2t}-\frac{1}{2}e^{3t}+\frac{1}{6}e^{4t}
   \end{align*}
  \item[(ii)]
    \begin{align*}
      y=\frac{1}{4}t\sin t-\frac{1}{4}t^2\cos t+\cos t+2\sin t
    \end{align*}
     \item[(iii)]
    \begin{align*}
      y=1+e^{-t}+\ff{2}{\sq{3}}e^{-1/2t}\left(\ff{\sq{3}}{2}\cos\ff{\sq{3}}{2}t-\ff{7}{2}\sin\ff{\sq{3}}{2}t\right)
    \end{align*}
   \item[(iv)]
     \begin{align*}
       y=&\ff{2}{\sq3}e^{-1/2t}\sin(\ff{\sq3}{2}t+\ff{\pi}{3})+\\
       &H(t-\pi)\left(1+\ff{2}{\sq{3}}e^{-1/2(t-\pi)}\sin(\ff{\sq3}{2}(t-\pi)-\ff{2\pi}{3})\right)+\\
       &H(t-2\pi)\left(1+\ff{2}{\sq{3}}e^{-1/2(t-2\pi)}\sin(\ff{\sq3}{2}(t-2\pi)-\ff{2\pi}{3})\right)
     \end{align*}
   \item[(v)]
     \begin{align*}
       &\text{for $0\le t<\pi$:}\quad&y=\cos t+\ff{1}{2}\sin t-\ff{t}{2}\cos t\\
       &\text{for $\pi\le t$:}\quad&y=\left(\ff{t}{2}-\ff{\pi}{2}\right)\sin t-\left(\ff{\pi}{2}-1\right)\cos t
     \end{align*}
    \item[(vi)]
      \begin{align*}
	&\text{for $0\le t<\pi/2$:}\quad&y=3\cos t-\sin t+\ff{1}{2}t\sin t\\
       &\text{for $\pi/2\le t$:}\quad&y=(\ff{\pi}{4}-1)\sin t+\ff{5}{2}\cos t
         \end{align*}
 \end{vv286_ms}

\end{document}
