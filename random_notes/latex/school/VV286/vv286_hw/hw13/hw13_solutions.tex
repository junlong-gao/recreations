\input mla
\begin{document}
\begin{vv286}{13}
\begin{vv286_ms}{1}
  Done.
\end{vv286_ms}

\begin{vv286_ms}{2}
    Again we follow the slides and use the separation of variable $u(x,t)=X(x)T(t)$ to get:
    \eq
    {
      X\frac{d^2T}{d t^2}=
      -c^2\frac{d^4 X}{d x^4}T
      \implies
      \frac{-1}{c^2 T}\frac{d^2 T}{d t^2}
      =\frac{d^4 X}{d x^4}\frac{1}{x}=k
    }
    since the both side is independent and must be constant.
    \\
    Suppose $k>0$, then we have the equation:
    \eq
    {
      \frac{d^4X}{dx^4}=kX
    }
    since $k>0$, this gives the solution:
    \eq
    {
      X(x)=c_1e^px+c_2e^{-px}+c_3\cos px+c_4 \sin px
      \quad
      p=k^{1/4}
    }
    the initial condition requires:
    \eq
    {
    X(0)=0,\quad
    X(l)=0,\quad
    X''(0)=0,\quad
    X''(l)=0
    \quad
    \text{or}
    \quad
    T(t)\equiv0
    }
    the last one yields the trivial solution $u=0$, otherwise
    we obtain a system of linear equations, which solves:
    \eq
    {
    c_1=c_2=c_3=0, \quad
    c_4=0\quad
    \text{or}
    \quad
    p=\frac{n\pi}{l}\quad n=1,2,\ldots
    }
    If $c_4=0$ again we back to the trivial solution, otherwise we have:
    \eq
    {
      X(x)=c_n\sin\frac{n\pi x}{l}
    }
    this implies $k=p^4$ can only take on specific values, and we can solve for $T$:
    \eq
    {
    T''(t)=-c^2kT
    \implies
    T=A_n\cos \frac{cn^2\pi^2}{l^2}t+B_n\sin \frac{cn^2\pi^2}{l^2}t
    }
    thus the full separated solution is obtained:
    \eq
    {
      u_n(x,t)=\left(
      D_n\cos \frac{cn^2\pi^2}{l^2}t
      +E_n\sin \frac{cn^2\pi^2}{l^2}t
      \right)\sin\frac{n\pi x}{l}
    }
    where $D_n$ and $E_n$ are to be determined. Then the superposition yields:
    \eq
    {
      u(x,t)=\sum_{n=1}^{\infty}\left(
      D_n\cos \frac{cn^2\pi^2}{l^2}t
      +E_n\sin \frac{cn^2\pi^2}{l^2}t
      \right)\sin\frac{n\pi x}{l}
    }
    is a solution.
    \\
    Now we look at the boundary conditions:
    \eq
    {
    u(x,0)=x(l-x)\implies
    \sum_{n=1}^{\infty}
      D_n\sin\frac{n\pi x}{l}=x(l-x)
    }
    By the orthogonal Fourier series (we expand the initial $u(x,0)$ odd to origin) we know:
    \eq
    {
      D_n\frac{l}{2}=\int_0^{l}x(l-x)
      \sin\left( \frac{n\pi x}{l}\right)\,dx 
      \implies D_n=\frac{4l^2}{n^3\pi^3}(1-(-1)^n)
    }
    similarly we have another boundary condition yields:
    \eq
    {
    F_n=0
    }
    therefore the complete solution is:
    \eq
    {
      u(x,t)=\sum_{n=1}^{\infty}
      \frac{4l^2}{n^3\pi^3}(1-(-1)^n)
      \cos\left( \frac{cn^2\pi^2}{l^2}t \right)
      \sin\left( \frac{n\pi x}{l} \right)
    }
    When $k=0$ we easily find that the only possible situation is the trivial solution $u=0$.
    \\
    When $k=-\lam^2<0$, we note the $T(t)$ has the solution:
    \eq
    {
      T(t)=c_1e^{-\lam c t}+c_2e^{\lam c t}
    }
    Now the boundary condition $u(x,0)=0$ requires:
    \eq
    {
    X(x)(c_2+c_1)=x(l-x)
    }
  then we have $X=x(l-x)/C$, and this contradicts the boundary condition $u_{xx}(0,t)=u_{xx}(l,t)=0$.
  \\
  In conclusion, the solution is:
   \eq
    {
      u(x,t)=\sum_{n=1}^{\infty}
      \frac{4l^2}{n^3\pi^3}(1-(-1)^n)
      \cos\left( \frac{cn^2\pi^2}{l^2}t \right)
      \sin\left( \frac{n\pi x}{l} \right)
    }
  \end{vv286_ms}
  \begin{vv286_ms}{3}
    By the separation of variable we obtain, with boundary $u(0,t)=u(L,t)=0$:
    \eq
    {
      X(x)=A_n\sin\left( \frac{n\pi x}{L} \right)
    }
    and we obtain the equation for $T$:
    \eq
    {
    T(t)=e^{-\mu/2}\left( B_n\cos\w_n t+D_n\sin \w_n t \right)
    \quad
    \w_n=\frac{1}{2}\sq{\frac{4n^2\pi^2c^2}{L}-\mu^2}
    }
    before superposition, we notice $u_x(x,0)=\sin(\pi x/L)$ implies only $n=1$ exists,
    and the complete solution:
    \eq
    {
      u(x,t)=e^{-\mu t/2}\left( C_1\cos\w_1t+F_1\sin\w_1t \right)
      \sin\left( \frac{n\pi x}{L} \right)
    }
    the constants are given by:
    \eq
    {
      u(x,0)=\sin\left( \frac{\pi x}{L} \right)\implies C_1=1\\
      u_t(x,0)=0\implies F_1=\frac{\mu}{2\w_1}
    }
    The final solution is:
    \eq
    {
    u(x,t)=e^{-\mu/2t}\left( \cos\w_1t+\frac{\mu}{2\w_1}\sin\w_1t \right)
      \sin\left( \frac{n\pi x}{L} \right)
   \quad
      \w_1=\frac{1}{2}\sq{\frac{4\pi^2c^2}{L}-\mu^2} 
    }
  \end{vv286_ms}
  \begin{vv286_ms}{4}
    Again separation of variable $u=XY$ gives:
    \eq
    {
      \frac{1}{X}X''=\frac{1}{Y}\left( Y-Y'' \right)=k
    }
    and the only non-trivial solution is obtained through:
    \eq
    {
    X=B_n\sin nx\quad
    Y=C_n\left( e^{\sq{1+n^2}y}-e^{-\sq{1+n^2}y} \right)
    }
    and the superposition gives:
    \eq
    {
      u(x,y)=\sum_{n=1}^{\infty}D_n\sin nx\left( e^{\sq{1+n^2}y}-e^{-\sq{1+n^2}y} \right)
    }
    where $D_n$ are constants such that:
    \eq
    {
    \sum_{n=1}^{\infty}D_n\sin nx\left( e^{\sq{1+n^2}a}-e^{-\sq{1+n^2}a}\right)=1
    }
    by orthogonal Fourier integral (we expand the initial $u(x,0)$ odd to origin) we immediately have:
    \eq
    {
      D_n=\frac{2\int_0^{\pi}\sin nx\,dx
      /2\int_0^{\pi}\sin^2 n\pi\,dx}{ e^{\sq{1+n^2}a}-e^{-\sq{1+n^2}a}}=
      \frac{1-(-1)^n}{n\pi}\frac{1}{{ e^{\sq{1+n^2}a}-e^{-\sq{1+n^2}a}}}
    }
    plugging back to the solution:
    \eq
    {
      u(x,y)=\sum_{n=1}^{\infty}\frac{1-(-1)^n}{n\pi}\ff{ e^{\sq{1+n^2}y}-e^{-\sq{1+n^2}y}}
      { e^{\sq{1+n^2}a}-e^{-\sq{1+n^2}a}}\sin nx
    }
  \end{vv286_ms}
  \begin{vv286_ms}{5}
  \def\tt{(\w t+Bx)}
  By separation of variable $u=TX$, notice that the initial condition requires the $T=U_0\cos\w t$. Then, this essential requires:
  \eq
  {
  T''+(\a+\b)T'+\a\b T=\lam T\implies T=U_0\cos\w t
  }
  This essentially requires $\a\b<0$, but by condition $\a,\,\b>0$, this yields contradiction.
  
  
    To obtain the constants, simply plug in the solutions into the equation:
    \eq
    {
      &e^{-Ax}U_0\left( -\w^2\cos\tt-(\alpha+\beta)\sin\tt
      +\alpha\beta\cos\tt\right)=
      \\
      &\quad\quad e^{-Ax}U_0
      \left( 2AB\sin\tt+(A^2-B^2)\cos\tt \right)
    }
    this gives us the requirement:
    \eq
    {
    A^2-B^2=\alpha\beta-\w^2\\
    2AB=-(\alpha\beta)
    }
    which determines $A$ and $B$ once $\alpha$, $\beta$, $\w$ 
    is given.
  \end{vv286_ms}
  \begin{vv286_mp}{6}
These follow immediately from the last assignment:
\eq
    {
      2vJ_v(x)&=xJ_{v-1}(x)+xJ_{v+1}(x)\\
      2J_v'(x)&=J_{v-1}(x)-J_{v+1}(x)
    }
    solve for $J_{v-1}$ from the first one and plugging into the second equation:
    \eq
    {
      J'_v(x)=-J_{v+1}(x)+\frac{vJ_v(x)}{x}
    }
    solve for $J_{v+1}$ from the first one and plugging into the second equation:
    \eq
    {
      J'_v(x)=J_{v-1}(x)-\frac{vJ_v(x)}{x}    
    }
    After long research {\bf I essentially doubt the result we are asked to prove.} 
    \\
    In fact:
  \eq
  {
    &\frac{\partial}{\partial \beta}\left( 
    \beta J_v(\alpha)J_v(\alpha)
    -
    \alpha J_v'(\alpha)J_v(\beta)
    \right)
    \\
    =&\beta J_v(\alpha)J_v''(\beta)
    +J_v(\alpha)J_v'(\beta)
    -\alpha J_v'(\alpha)J_v'(\beta)
  }
  Now since the Bessel function must satisfy Bessel equation:
  \eq
  {
    J_v''(\alpha)+\ff{J'_v(\alpha)}{\alpha}
  =\left( \frac{v^2}{\alpha^2}-1 \right)J_v(\alpha)
  \numberthis{1}
  }
  let $\alpha\to\beta$ in the limit, apply L'Hospital's rule , we have:
  \eq
  {
    \lim_{\beta\to\alpha}
    \frac{\beta J_v(\alpha)J_v(\alpha)
    -
    \alpha J_v'(\alpha)J_v(\beta)
}{\alpha^2-\beta^2}
&=\frac{\alpha J_v(\alpha)J_v''(\alpha)
    +J_v(\alpha)J_v'(\alpha)
    -\alpha
  J_v'(\alpha)J_v'(\alpha)}{-2\alpha}
    \\
    &=\frac{1}{2}J_v'^2(\alpha)
    -\frac{1}{2}J_v(\alpha)
    \left( 
    J_v''(\alpha)+\frac{J'_v(\alpha)}{\alpha}
    \right)\\  
    & =\frac{1}{2}J_v'^2(\alpha)
      -\frac{1}{2}J_v^2(\alpha)
      \left( \frac{v^2}{\alpha^2}-1 \right)
      \quad\text{using \eqref{1}}
}
In fact, when $\alpha$ is a zero of $J_v$, the desired limit in the assignment holds.
  \end{vv286_mp}
 \begin{vv286_mp}{7}
 \item[i)]
   This follows immediately from the fact that:
   \eq
   {
     \left( i \right)^{2m+v}=
     \left( e^{\pi i/2} \right)^{2m+v}
     =e^{\pi mi+\pi i v/2}
     =(-1)^{m}e^{\pi iv/2}
   }
   and the series representation of Bessel function:
   \eq{
     J_v(ix)=\sum_{m=0}^{\infty}\frac{(-1)^m}{m!\Gamma(1+m+v)}
     \left( \frac{ix}{2} \right)^{2m+v}
     =\sum_{m=0}^{\infty}\frac{1}{m!\Gamma(1+m+v)}
     \left( \frac{x}{2} \right)^{2m+v}e^{\pi iv/2}
   }
   multiply by $e^{-v\pi i/2}$ and we obtain the series for $I_v$.
 \item[ii)]
   We recall from the last assignment:
   \eq
   {
     Y_{0}(ix)
     &=\frac{2}{\pi}J_0(ix)
     \left( \ln\left( \frac{ix}{2} \right)+\gamma \right)
     -\frac{2}{\pi}
     \sum_{n=1}^{\infty}\frac{(-1)^n}{(n!)^2}
     \left( \frac{ix}{2} \right)^{2n}
     H_n\\
     &=\frac{2}{\pi}J_0(ix)
     \left( \ln\left( \frac{x}{2} \right)+i\frac{\pi}{2}+\gamma \right)
     -\frac{2}{\pi}
     \sum_{n=1}^{\infty}\frac{(-1)^n}{(n!)^2}
     \left( \frac{ix}{2} \right)^{2n}H_n
     \\
     &=
     J_0(ix)
     +\frac{2}{\pi}J_0(ix)
     \left( \ln\left( \frac{x}{2} \right)+\gamma \right)
     -\frac{2}{\pi}
     \sum_{n=1}^{\infty}\frac{1}{(n!)^2}
     \left( \frac{x}{2} \right)^{2n}H_n
   }
 where a suitable branch for $\ln$ is chosen such that $\ln \i=\pi/2 i$.\\
   From the above derivation in $i)$:
   \eq{
   J_0(ix)=\sum_{m=0}^{\infty}\frac{1}{(m!)^2}
     \left( \frac{x}{2} \right)^{2m}
   }
   we have:
   \eq
   {
     K_0(x)&=
     \sum_{n=1}^{\infty}\frac{1}{(n!)^2}
     \left( \frac{x}{2} \right)^{2n}H_n
-J_0(ix)
     \left( \ln\left( \frac{x}{2} \right)+\gamma \right)\\
     &=\sum_{n=1}^{\infty}\frac{1}{(n!)^2}
     \left( \frac{x}{2} \right)^{2n}
     \left( 
     H_n-\ln\left( \frac{x}{2} \right)-\gamma 
     \right)
   }
   and indeed the series diverges at $x=0$ due to the $\ln\left( x/2 \right)$ term.
 \item[iii)]
   Note:
   \eq
   {
     (J_v(ix))'=iJ_v(ix),\quad(J_v(ix))''=-J_v(ix)
   }
   therefore:
   \eq
   {
   &x^2(J_v(ix))''+x(J_v(ix))'-(x^2-v^2)J_v(ix)\\
   =&(ix)^2J_v(ix)+ixJ_v(ix)+\left( (ix)^2-v^2 \right)J_v(ix)
   }
   But this is the Bessel equation in $ix$ of order $v$, and this is by definition 0. Thus $J_v(ix)$ solves:
   \eq
   {
   x^2y''+xy'-(x^2+v^2)y=0
   }
   and by linearity $I_v(x)=J_v(ix)e^{-v\pi i/2}$ solves the equation.
   As for $K_v(x)$, note
   \eq
   {
     iJ_v(ix)
   }
   solves the equation by our previous argument, and recall
   \eq
   {
     Y_v(ix)=\frac{J_v(ix)\cos v\pi-J_{-v}(ix)}{\sin(v\pi)}
   }
   note each linear part solves the equation by our previous argument, so is $Y_v(ix)$, and is 
   \eq
   {
     K_v(x)=\frac{\pi}{2}e^{v\pi i/2}\left( iJ_v(ix)-Y_v(ix) \right)
   }
   This completes the proof.
  \end{vv286_mp}

\end{vv286}
\end{document}
