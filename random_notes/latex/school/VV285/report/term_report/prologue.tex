\chapter{Prologue}
\setcounter{section}{0}
{\flushright
	{\sffamily\slshape
		"The time has come," the Walrus said,\\
		"To talk of many things"
		\vskip 0.1 in
		{from \it Through the Looking-Glass and What Alice Found There \rm (1872)}
	}	
\vskip 2in 
}
\section{Pendulum Revisited\hfill}
\lettrine[lines=2]{$\mathscr{T}$\,}\lowercase{i}me keeping has long been dominated by the means of a pendulum. A typical pendulum clock features a massive pendulum on one end of the rod, while the other end is fixed to some pivot. The beautiful mathematical property is that the periodical motion of the swing is almost isochrone, that is, as long as the amplitude is small, the period is invariant for different initial positions. It was through experimental fact that this property got first discovered by the young talented Italian Galileo Galilei at around 1602. We shall investigate this phenomenon through a more mathematical approach.

A {\it mathematical pendulum} is an ideal model of a physical pendulum 
where a particle of mass $m$ is connected by a massless thread or rod 
of length $l$ with one end fixed to a point so that it can rotate in a plane 
with uniform parallel gravitational force. \cite{awrejcewicz2012classical}

A {\it physical pendulum}, in contrast, takes into account of the moment of inertia of the thread or rod even if the resistance due to the pivot or air is
negligible. Also, note that the physical pendulum is affected by the Coriolis force
and cannot simply rotate in a fixed plane on earth.

Now, if we restrict ourselves to the ideal model of a mathematical pendulum,
the energy of the system is given by adding the kinetic energy and the 
gravitational potential energy:
\eq{
E=K+U=\ff{1}{2}mv^2+mgh
}
Denote the displacement of angle by $\ta$, we have relation
\eq
{
v=l\dot{\ta}, \qquad h=l-l\ct
}
we obtain
\etag{1}{
E(\ta,\dot{\ta})=\ff{1}{2}ml^2\dot{\ta}^2+mgl(1-\ct)
}
where we take the zero potential plane $h=0$ at the pivot and $g$ indicates the
constant gravity field.

The energy $E$ is invariant under time since the mechanical energy is
conserved in a constant field, we take derivative with respect to time in \eqref{1}:
\eq{
0=\dd{}{t}E=\ff{1}{2}ml^2\dd{}{t}\dot{\ta}^2+\dot{\ta}mgl\st
}
giving us the {\it pendulum equation}:\cite{awrejcewicz2012classical}
\etag{pendulum equation}{
\ff{d^2}{dt^2}\ta(t)+\ff{g}{l}\st(t)=0
}
The above differential equation determines the motion of a pendulum in terms of 
its angle $\ta$.  Unfortunately this is a non-linear equation and is unlikely to
 have an explicit solution. Yet this can be well approximated by a linear equation as long as the amplitude $\ta$ is small by using $\ta/\st\to 1$ as $\ta\to 0$:
 \ed{
 \ff{d^2\ta}{dt^2}=-\ff{g}{l}\ta
 }
 The general solution to this is 
 \eq{
 \ta(t)=A\cos(\w t+\p),\quad \w=\sq{\ff{g}{l}}\implies T=2\pi\sq{\ff{l}{g}}
 }
 which gives us a surprisingly simple description of the motion that the period is proportional to the square root of the pendulum arm.  However, for a precise 
 instrument like a clock,an approximation like this is overly optimistic. As one
 shall see in the next chapter, the real period is at least first order quadratic to the 
 initial angle $\ta_0$, and such a simple-looking timekeeping device suddenly presents all the wonders, from the elliptic integral to the charmingly mysterious cycloid curve, and this shall be our topic in the following chapters.
 
 


