\chapter{Epilogue}
\setcounter{section}{0}
\section{The Brachistochrone\hfill}
\lettrine[lines=2]{$\mathcal{L}$\,}\lowercase{e}t us conclude our study of the cycloid curve with another interesting story\cite{dunham1990journey}.
In 1696,  the Swiss mathematician 
Johann Bernoulli, who was once tutored of the immortal young Euler and made so many important contributions to the calculus and infinite series, published an
interesting problem in Leibniz's journal {\it Acta Eruditorum}: Two points {\it A} and
{\it B} at different heights are joint by a special curve in such a way that
the curve remains in a plane parallel to the gravity force that {\it minimizes} the time
for a particle to travel along the curve. He called the desired curve ``brachistochrone", which is borrowed from the Greek words ``shortest" and ``time".

Of course the first curve coming to one's mind is likely to be a straight line, yet
Johann warned that:
 \begin{quotation}
``...to forestall hasty judgment, although the straight line AB is indeed the shortest between the points A and B, it nevertheless is not the path traversed in the shortest time. However the curve AMB, whose name I shall give if no
one else has discovered it before the end of this year, is one well-known to geometers."\cite{dunham1990journey}
 \end{quotation}
 
 Indeed, the solution to the problem is something that has long been known to 
 mathematicians but it takes ingenuity and inspiration to resolve the 
 problem. It was an open challenge Johann put forward to the rest of the European mathematicians, as he put it:
  \begin{quotation}
 ``Let who can seize quickly the prize which we have promised to the solver. Admittedly this prize is neither of gold nor silver, for these appeal only to base and venal souls...Rather, since virtue itself is its own most desirable reward and fame is a powerful incentive, we offer the prize, fitting for the man of noble blood, compounded of honor, praise, and approbation..."\cite{dunham1990journey}
   \end{quotation}
   
   Now what followed was a charming story that most textbooks nowadays fail
   to introduce to the young students. On the extended due day, only five solutions were sent to Johann. In one of them which beard an English
   postmark, although anonymous, Johann found it totally correct.
   This, was from his match Sir. Isaac Newton. There is a legend that
   Johann at that instance, was half in shame and half in awe, intentionally
   commented that ``I recognize the lion by his paw."
  
  Now the correct solution to this problem, the curve that the whole Europe were seeking, is just an {\bf upside-down cycloid}. Indeed it was long known 
  to the mathematicians and is just studied in our report. Neither Huygens
  nor Pascal was able to realize that this is the quickest descent, who had made so many contributions to the studies of the curve.
  
  Even thought it takes ingenuity and considerable effort to use the principle
  of geometry to find so many properties of the curve, let us not forget that 
  it was the effort of these heroes that paved the way for the development of
  calculus and those problems seemed can only be tackled by the great
  minds, can now be solved with a handful of tricks that even an average high schooler can understand.
  \vfill
{\flushright
	{\sffamily\slshape
		  Before I end I must voice
	once more  the admiration I feel \\
		  for the unexpected identity of Huygens' tautochrone and my brachistochrone. \\
		  I consider it especially remarkable that\\
		   this coincidence can take place only under the hypothesis of Galileo,\\
		   so that we even obtain from this a proof of its correctness.\\ Nature always tends to act in the simplest way, \\
		   and so it here lets one curve serve two different functions,\\
		   while under any other hypothesis we should need two curves...
		 \vskip 0.1 in
		{\hfill\sffamily\upshape --- 
\textsc{Johann Bernoulli}}
	}	

}
