\chapter{Tautochrone Problem}
\setcounter{section}{0}
{\flushright
	{\sffamily\slshape
		We can face our problem. \\
		We can arrange such facts as we have \\
		with order and method.\\
		\vskip 0.1 in
		{\sffamily\upshape --- \textsc{Hercule Poirot}.  \it Murder on the Orient Express \rm (1934)}
	}	
\vskip 2in 
}


\section{The Relic from Harmonic Motion\hfill}
\lettrine[lines=2]{$\mathcal{I}$\,}\lowercase{f} we parameterize the curve through curve length, the tangential component of 
the acceleration uniquely determines the time it travels. Now we want to 
seek a differential relation such that the periodic solution is independent
of the initial displacement. 

However, such a differential relation has long been known to the students of physics, namely the harmonic oscillation:
\eq{
\ff{d^2x}{dt^2}=-kx\quad k>0\numberthis{periodical}
}
which yields the general solution:
\eq{
x=C\cos{(\sq k t+\p_0)}
}
where $\w=\sq k$ is the frequency of the motion:
\eq{
T=\ff{2\pi}{\w}
}
An interesting question immediately pops up: suppose we have a
periodical motion, does the acceleration long the trajectory have to
satisfy \eqref{periodical}? The answer is affirmative under reasonable assumption:
in one dimensional motion if the acceleration is only determined by particle's 
current position, there exists a unique potential function $U(x)$ such that
\eq
{
\ff{d}{dx}U(x)=-ma\quad, U(0)=0\numberthis{motioneq}
}
Then the periodic motion can be characterized by considering the motion near
$x=0$. As $x$ increases the $U$ increase monotonically until $U(x)$
reaches the initial mechanical energy, say, at $x=x_1$. Then the conservation of
mechanical energy shows that at this position the particle stops, turns around and
goes back to the origin. If the other side $x<0$ follows a similar configuration, this ``potential well''(since the particle can never go outside, invoking the term
From Vp160 Physics)
corresponds to a periodic motion.\\
The conservation law of mechanical energy shows that the time it
requires to travel from the origin to the turning point is:
\eq{
	&\int_{0}^{t_1} dt=\int_{t_1}^{t_2} \ff{dx}{v}\\
	&\quad=\int_{0}^{x_1} \sq{ \ff{m}{2E(x)}}\,dx\\
	&\quad=\int_{0}^{x_1}\sq{ \ff{m}{2(U(x_1)-U(x))}}\,dx
}
We are interested in the integral of the form
\eq
{
\int_0^{x_1}\sqrt{\ff{1}{U(x_1)-U(x)}}\,dx\numberthis{int}
}
such that the result doesn't depend on $x_1$.\\
Now we use the transformation:
\eq
{
\tau^2=\ff{U(x)}{U(x_1)},\quad \tau:0\to1\numberthis{invertible}
}
notice we use the fact that from $0$ to $x_1$ $U(x)$ is monodically increasing. \\
The integral \eqref{int} becomes:
\eq
{
\int_0^{1}2\sq{\ff{1}{1-\tau^2}}\ff{\sq{U(x(\tau))}}{U'(x(\tau))}\,d\tau\numberthis{principle}
}
since the invertibility in \eqref{invertible} implies the bijection $\tau\mapsto x$.\\
The only term that depends on $x_1$ in \eqref{principle} is
\eq
{
(\tau,x_1)\mapsto x\mapsto \ff{\sq{U(x)}}{U'(x)}\numberthis{map}
}
where the first map $\p$ is determined by \eqref{invertible}: 
\eq
{
x=U^{-1}(\tau^2U(x_1))\numberthis{xont}
}
In order to eliminate the dependence on $x_1$, we set
\eq
{
\pd{}{x_1}\ff{\sq{U(x)}}{U'(x)}=0
}
and by chain rule and inverse function theorem, we look at \eqref{map} and
\eqref{xont}:
\eq
{
0=\pd{}{x_1}\ff{\sq{U(x)}}{U'(x)}=\dd{}{x}\left(\ff{\sq{U(x)}}{U'(x)}\right)\cdot \pd{x}{x_1}=\dd{}{x}\left(\ff{\sq{U(x)}}{U'(x)}\right)
\ff{1}{\displaystyle\dd{U}{x}}\left(\tau^2\dd{U(x_1)}{x_1}\right)
}
this vanishes for all $x_1$ if and only if
\eq
{
\dd{}{x}\left(\ff{\sq{U(x)}}{U'(x)}\right)=0
}
or equivalently, $U(x)$ must satisfy the following differential equation:
\eq
{
\ff{\sq{U(x)}}{U'(x)}=\rm const
} 
The general solution for this is:
\eq
{
U(x)=cx^2
}
for some constant $c$ (recall $U(0)=0$ by the assumption).\\
Back in \eqref{motioneq} we conclude that
\eq
{
a=\ddot{x}=-\ff{1}{m}\ff{d}{dx}U(x)=-kx
}
This establishes the necessary and sufficient condition for the periodic motion in a potential field where the period doesn't depend on the initial position. 

Now we make an analogy that the tautochone should be a curve where \eqref{periodical} holds along the trajectory it travels:
\etag{5}
{
\ff{d^2s}{dt^2}=-ks\quad k>0
}
then in order to find the period of the motion, one can proceed as if the particle is moving along a straight line subjected to linear restoring force.

Of course, the real trajectory should be curved in such a way that the tangential
component of the gravity satisfies \eqref{5}, and that is what we are looking for in the next section.

\section{How Does the Cycloid Come into Play \hfill}
\lettrine[lines=2]{$\mathcal{W}$\,}\lowercase{e} want to formulate the trajectory so that \eqref{5} holds in terms of $x$ and $y$
axes in parameterized form. From Newtonian dynamics the acceleration in
the tangential component is due to projection of the gravitational force:
\eq
{
\ff{d^2s}{dt^2}=-g\st
} 
where we identify the acceleration is in the decreasing direction of the angle
$\ta$ from the vertical position. 
Combine it with \eqref{5} we have:
\eq{
s=\ff{g}{k}\st
}
we will be more interested in the time derivative to get the component along $x$ and $y$ axes:
\eq{
\ff{ds}{dt}=\ff{g}{k}\ct \,\dot{\ta}\numberthis{6}
}
If we set up our coordinate system in such a way that $y$ axis is along the 
equilibrium position pointing upwards then we have, by decomposing the
line velocity $ds/dt$ into two components\cite{laubenbacher2007mathematical}:
\eq{
\ff{dx}{dt}=\ff{ds}{dt}\ct\quad\ff{dy}{dt}=\ff{ds}{dt}\st\numberthis{7}
} 
with the initial condition 
\eq{
x\Big|_{\ta=0}=0,\quad y\Big|_{\ta=0}=-\ff{g}{2k}
}
Now we put the two equations from \eqref{7} into \eqref{6} to obtain:
\eq
{
\ff{dx}{dt}=\ff{g}{k}\cos^2\ta\ff{d\ta}{dt}\quad \ff{dy}{dt}=-\ff{g}{k}\st\ct\ff{d\ta}{dt}
}
then the integration over parametrical form gives us:
\eq
{
\ff{dx}{d\ta}=\ff{g}{k}\cos^2\ta,\quad\ff{dy}{d\ta}=\ff{g}{k}\st\ct
}
integrate with respect to $\ta$:
\eq
{
x(\ta)-x(0)=\ff{g}{4k}\left(2\ta+\sin2\ta\right),\quad
y(\ta)-y(0)=\ff{g}{4k}\left(1-\cos2\ta\right)
}
after a bit tackling from initial condition, this is a cycloid with parameter $\ta$:
\eq
{
x(\ta)=\ff{g}{4k}\left(2\ta+\sin2\ta\right),\quad
y(\ta)=-\ff{g}{4k}\left(1+\cos2\ta\right)
}

\section{Winding Cycloid \hfill}
\lettrine[lines=2]{$\mathcal{P}$\,}\lowercase{r}ovided that the isochrone curve is actually a cycloid curve, we need special
metal plates to twist the thread along the swing in such a way that the 
real trajectory formed by the massive pendulum is a cycloid, if we want
to use a fixed length thread. 

This new problem is solved through the surprising relation between the 
{\it involute}\cite{mccleary2012geometry} of the cycloid and the cycloid itself: if we fix a thread on one cusp of
a cycloid and start winding the thread to the curve in such a way that
at each spot the thread is tangential to the curve where it osculates with the curve, then
the end of the thread forms the involute of the curve. In the case
of a cycloid, the involute is again a cycloid. This amazing property allows us
to curve the metal plate into a cycloid shape. Once the thread starts
swinging and winding to it, the massive pendulum at the free end will produce again a
cycloid, as desired.

\begin{center}
\begin{tikzpicture}
    \begin{axis}[
	title={The Tautochone realized through metal plate},
	unit vector ratio*=1 1 1,
    width=11cm,
    legend entries={$r(\ta)$,$I(\ta)$},
        legend pos=north east,]
    
    \addplot[variable=\t, 
	domain=-0:pi,
	sample=100,
	smooth] 
	({t-sin(deg(t))}, {-1+cos(deg(t))});
    \addplot[variable=\t, 
	domain=-0:pi,
	sample=100,
	smooth,
	dashed] 
	({t+sin(deg(t))}, {-3-cos(deg(t))});
	    \end{axis}
\end{tikzpicture}
\end{center}

Now let's see mathematically how it comes to this.
Suppose we have a cycloid starting its parameterization from the origin
\eq
{
r(\ta)=k(\ta-\st,\,-1+\ct)\numberthis{plate}
}
and a thread with the length $4k$, we want to find the trajectory of the free end:
\eq
{
I(\ta)=r(\ta)+(4k-l(\ta))T_r(\ta)\numberthis{9}
}
where $T_r(\ta)$ is the unit tangent vector at the point $r(\ta)$. $l(\ta)$ is 
the length from starting point (origin) to point $r(\ta)$.

From definition in Vv285, the unit tangent vector is:
\eq{
T_r(\ta)=\ff{r'(\ta)}{\|r'(\ta)\|}=\ff{1}{\sq{2-2\ct}}(1-\ct,-\st)
}
and the length follows:
\eq
{
l(\ta)=\int_0^{\ta}\|r'\|=4k\left(1-\cos\ff{\ta}{2}\right)
}
back to \eqref{9} we have:
\eq
{
I(\ta)&=r(\ta)+4k\cos\ff{\ta}{2}\ff{1}{\sq{2-2\ct}}\left(1-\ct,\,-\st\right)\\
&\quad=r(\ta)+4k\left(\sq{\ff{1+\ct}{2}}\ff{1-\ct}{\sq{2-2\ct}},\,\ff{\cos\ff{\ta}{2}\st}{\sq{4\sin^2\ff{\ta}{2}}}\right)\\
&\quad=r(\ta)+2k\left(\st,\,-1-\ct\right)\\
&\quad=k(\ta+\st,-3-\ct)\\
&\quad=k(\p-\st-\pi,\,-(1-\ct)-2)\numberthis {10}
}
where in the last line we use the transformation $\p=\ta+\pi$. Now \eqref{10}
is exactly the curve of the plate in \eqref{plate}, yet shifted by $(-\pi k,-2k)$. In other words, given the length of $4k$ units and the metal plate parametrized with
coefficient $k$, the thread pendulum will swing along a cycloid with
the exact trajectory as a cycloid so that the period of the motion is the same no matter
what the initial position is.
\vskip 1in
{\flushright
	{\sffamily\slshape
	In a Cycloid with a vertical axis, \\
		and with the vertex seen to be the lowest point,\\
		 the times of descent for some body, \\
		 on leaving any point on the cycloid from rest until\\
		  it reaches the lowest point at the vertex,\\
		  are equal to each other;
		 \vskip 0.1 in
		{\hfill\sffamily\upshape --- 
\textsc{Christiaan Huygens}. p57, PROPOSITION XXV,  \it Horologium oscillatorium \rm (1673)}
	}	
}
\vskip 2in
\section{Remarks\hfill}
\lettrine[lines=2]{$\mathcal{T}$\,}\lowercase{h}e involute we obtained through the winding process can be generalized to a
much wider range of curves which have reasonably good behavior with curve
length parametrization\cite{mccleary2012geometry}:
\vskip 0.1 in
Let $\a$ be a smooth $C^2$-curve (where we adopt the notation in vv285)
, a curve $\b$ such that $\b$ lies along the tangent line to $\a$ and $\b'(s)\perp\a'(s)$ is called an {\it involute} of $\a$, and $\a$ is called an evolute of $\b$.
\vskip0.1 in
Evidently, the Huygens' idea of winding the cycloid is essentially finding the 
involute of a cycloid, and the amazing property we proved before can be
now rephrased as follows:
\begin{center}
``{\it The cycloid has a congruent cycloid as its evolute.}"\cite{mccleary2012geometry}
\end{center}
From the definition it seems that one can only find the involute of a given
curve, rather than the evolute of it. But there's much more to this\cite{mccleary2012geometry}:
\vskip 0.1 in
Let $\a$ to be a smooth $C^2$-curve and we denote its centers of curvature circles $C_\a(s)$ (which is again a curve, not necessarily parameterized by curve length). Then $C_\a(s)$ is an evolute of $\a$.
\vskip0.1 in
Indeed, from this point we conclude that to prove this statement it's sufficient
to show that the points of $C_\a(s)$ lies on the tangent line of $\a$ and 
the tangents are orthogonal. We first recall the properties of $C_{\a}(s)$:
\eq
{
&\kappa(s)=\ff{1}{\|\a(s)-C_\a(s)\|},\\
&(\a(s)-C_\a(s))\cdot\a''(s)=-1\quad \hbox{and}\quad (\a(s)-C_\a(s))\perp \a'(s)
}
The first one follows from the definition of curvature and the second uses
the fact that curve length parameterization  implies no tangential acceleration, thus $\a''$ must be pointing to the center of curvature with the proper magnitude and perpendicular to the tangential direction.

Now we begin by invoking the relation in assignment 7 of VV285:
\eq
{
\ff{dN}{ds}=-k(s)T(s)\numberthis{13}
}
On the other hand, we already know that $(\a(s)-C_\a(s))$ is in the normal
direction, we may write:
\eq
{
(\a(s)-C_\a(s))\cdot\a''(s)=m(s)N(s)\cdot\kappa(s)N(s)
}
the left hand side is simply $-1$ , and $N(s)\cdot N(s)=1$, we conclude that:
\eq
{
C_\a(s)=\a(s)+\ff{1}{\kappa(s)}N(s)\numberthis{12}
} 
differentiate it with respect to $s$:
\eq
{
C'_\a(s)=\a(s)-\ff{\kappa'}{(\kappa(s))^2}N(s)+\ff{1}{\kappa(s)}\ff{dN}{ds}=-\ff{\kappa'}{(\kappa(s))^2}N(s)\numberthis{relation}
}
by using \eqref{13} since curve length parameterization implies unit speed: $\a(s)=T(s)$.

From \eqref{relation} we see that $\a'(s)\perp C'_\a(s)$ and to show that
the points of $\a(s)$ is on the tangent line of $C_{\a}(s)$ we use plug \eqref{relation} into \eqref{12}:
\eq
{
\a(s)=C_{\a}(s)+\ff{\kappa(s)}{\kappa'(s)}C'_{\a}(s)
}
This completes the proof.

In order to construct his tautochrone, Huygens gave a practical definition of 
the involute in his book {\it Horologium oscillatorium} and provided various 
examples of finding involutes of different curves. Lacking differential tools, this genius used a complete 
classical Euclid approach to demonstrate his ideas. But what he did, had
made the very first step of what is now called differential geometry, long before
Newton's ideas on curves and derivatives.
\vfill
{\flushright
	{\sffamily\slshape
	If in one part of the cavity,\\
	 a thread or flexible line is understood to be wound around one of the curved lines;\\
	  and by keeping one end of that thread fixed to the curve,\\
	   with the other end drawn away thus so that the part which is free is always kept extended ;\\
	    then another curve is seen to be described by the free end of this thread. \\
	    Moreover this curve can be said to be described by evolution.
		 \vskip 0.1 in
		{\hfill\sffamily\upshape --- 
\textsc{Christiaan Huygens}. p61, DEFINITIONS III,  \it Horologium oscillatorium \rm (1673)}
	}	
}