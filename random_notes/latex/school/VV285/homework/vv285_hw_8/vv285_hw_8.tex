\documentclass{article}
\input mla
\begin{document}
\titlehm{8}
\ms{1}{
\item[(i)]
At $F_m=F_{\rm max}$ we reduce the first and the second into:
\eq{
&GJ\,H_0\sq{L^2-H_0^2}=F_mr_1r_0L^2\\
&\ff{L^2-H_0^2}{r_0}-\ff{L^2}{r_1}=0
}
where $r_1=r(F)$ and we many solve for $F_m$ and $r_1$:
\eq{
&F_m=\ff{L^2-H_0^2}{r^2_0L^4}GJ\,H_0\sq{L^2-H_0^2}\\
&r_1=\ff{L^2r_0}{L^2-H_0^2}
}
\item[(ii)]
We notice that by the given equations there defines an implicit function 
$g:\,F\mapsto(r,H,n)$,
and the differential is given by:
\eq
{
Dg\Big|_{F=0}=-\left(DG(0,\,\cdot\,)\Big|_{(H_0,n_0,r_0)}\right)^{-1}
DG(\,\cdot\,,H_0,n_0,r_0)\Big|_{F=0}
}
where $G(F,H,r,n)$ is defined as in question:
\eq{
G:\mm{F\\H\\r\\n}
\mapsto\mm{
&\displaystyleGJ\left(\ff{H_0}{Lr_0}\sq{L^2-H_0^2}-\ff{H}{Lr}\sq{L^2-H^2}\right)-Fr\sq{L^2-H^2}\\
&\displaystyleEJ\left(\ff{L^2-H_0^2}{r_0}-\ff{L^2-H^2}{r}\right)+FrLH\\
&\displaystyle(2n\pi r)^2+H^2-L^2
}
}
by setting $G(F,H,r,n)=(0,0,0)^T$ defines the $g$ around $(0,H_0,r_0,n_0)$
and by calculation we have
\eq{
&r'(0)=r_0^3\left(\ff{2}{GJ}\sa-\ff{1}{EJ}\sa\ff{\cos2\alpha}{\cos^2\alpha}\right)\\
&H'(0)=-Lr_0^2\left(\ff{\sin^2\alpha}{EJ}+\ff{\cos^2\alpha}{GJ}\right)\\
&n'(0)=\ff{r_0L}{2\pi}\sa\ca\left(\ff{1}{EJ}-\ff{1}{GJ}\right)
}
}
\mp{2}{
\item[i)]
We set: 
\eq{
F(e,\,x)=10\cdot\sum_{i=1}^{14}x^{3/2}\left(\cos\varphi_i\right)_+^{3/2}\cos\varphi_i=3000
}
and solve for 
\eq{
x=937.02\,\mu\m
}
and plugging it into $F_1$:
\eq{
F_1=937.02\N
}
and the loading ball are the half of the total: 7.
\item[ii)]
\def\cp#1{\cos\varphi_i}
We proceed as Lagrange Multiplier:
\eq
{
\ff{1}{C}D(F_1+\lam F)=0\implies&\ff{3}{2}(e+x)^{1/2}+
\lam\ff{3}{2}\sum_i\left(e+x\cp{i}\right)_+^{1/2}\cp{i}=0\\
&\ff{3}{2}(e+x)^{1/2}+\lam\ff{3}{2}\sum_i\left(e+x\cp{i}\right)_+^{1/2}\cos^2\varphi_i=0\\
&\sum_i\left(e+x\cp{i}\right)^{3/2}\cos\varphi_i-3000=0
}
The first two implies:
\eq{
\sum_i\left(\ff{e}{x}+\cp{i}\right)_+^{1/2}&\cp{i}=\sum_i\left(\ff{e}{x}+\cp{i}\right)_+^{1/2}\cos^2\varphi_i\\
&\implies \ff{e}{x}=0.75202
}
and substitute  it into the third equation:
\eq{
x^{3/2}\sum_i&\left(\ff{e}{x}+\cp{i}\right)_+^{3/2}\cos\varphi_i-3000=0\\
&\implies x=10.88\,\mu\m\qquad e=8.182\,\mu\m
}
to find the balls under load, we merely need to look into the quotient:
\eq{
\ff{e}{x}>-\cos\ff{\pi(i-1)}{7}
} 
and the number calculation shows:
\eq
{
-\cos\varphi_7=0.901>\ff{e}{x}
}
thus by symmetry ball 7,8,9 is unloaded and the other 11 balls are loaded.
}
\mp{3}{
\item[i)]
Notice that if $a>0$ and $\Delta=ac-b^2>0$
\ed{
Q_A(x,y)=\dotp{(x,y)}{A(x,y)}=ax^2+2bxy+cy^2=a\left(x+\ff{b}{a}y\right)^2+
\left(\ff{ac-b^2}{a}\right)y^2>0
}
for all $(x,y)\in\R^2$ since both of the term is positive. \\
On the other hand, if $a<=0$ or $\Delta=ac-b^2<=0$, the above expression will be negative 
eventually for sufficiently large $|x|,|y|$, and $Q_A$ is not positive definite. This establishes the 
if and only if condition.
\item[ii)]
Notice that $A$ is negative definite if and only if $-A$ is positive definite, and from (i) it happens
if and only if
\eq{
-a>0,\quad (-a)(-c)-(-b)^2>0
}
which exactly corresponds to 
\eq
{
a<0,\quad \Delta>0
}
\item[iii)]
By (1) if $ac-b^2<0$, then the quadric inequality 
in (1) has a pair of distinct solution $x_1/y_1,x_2/y_2$
such that the graph passes the axis $x/y$:
\eq{
y^2\left(a\left(\ff{x}{y}\right)^2+2b\ff{x}{y}+c\right)
}
thus there exists $x_0/y_0$ such that the above expression vanishes, and has
different signs near this point. 
}
\ms{4}{
\item[i)]
Notice that $f$ doesn't have any extrema in $\R^2/{(0,0)}$:
\eq{
Df=\mm{\ff{x}{\sq{x^2+y^2}}&\displaystyle\ff{y}{\sq{x^2+y^2}}}
}
and since it's not differentiable at $(0,0)$ we check by hand that it's a global minima.
In conclusion, we have
\begin{itemize}
\item[] Local minima:
\eq{
(0,0)
}
\item[]Local maxima:
None
\item[]Global minima:
\eq{
(0,0)}
\item[]Global maxima:
None
\end{itemize}
\item[ii)]
Setting the first differential to be $0$:
\eq{
Df=\mm{\sin y\sin(2x+y)&\sin x\sin(x+2y)}=\mm{0&0}
}
We get 
\eq{
(0,0),\,(0,\pi),\,(\pi,0),\,(\ff{\pi}{3},\ff{\pi}{3})
}
and 
we immediately conclude that on the boundary the points are the local and global minima's since $f>0$ inside the region and
vanishes on the boundary, and the point $({\pi}/{3},{\pi}/{3})$ is a global maxima since 
\eq{
{\rm Hess}f\Big|_{(\pi/3,\pi/3)}=\mm{\displaystyle-\sq3&\displaystyle-\ff{\sq3}{2}\\\displaystyle-\ff{\sq3}{2}&
\displaystyle-\sq3}
}
which is negative definite, making it the only local maxima, and larger then the boundary making
it the global maxima.\\
In conclusion, we have
\begin{itemize}
\item[] Local minima:
The boundary.
\item[]Local maxima:
\eq{
({\pi}/{3},{\pi}/{3})
}
\item[]Global minima: 
The boundary.
\item[]Global maxima:
\eq{
({\pi}/{3},{\pi}/{3})
}
\end{itemize}
\item[iii)]
Again we look for points such that 
\eq{
Df=\mm{\cos x+\cos(x+y)&\displaystyle\cos y+\cos(x+y)}=\mm{0&0}
}
which solves for:
\eq{
\left(\ff{\pi}{3},\ff{\pi}{3}\right)
}
and the Hess$f$ is 
\eq{
{\rm Hess}f\Big|_{(\pi/3,\pi/3)}=\mm{\displaystyle-\sq3&\displaystyle-\ff{\sq3}{2}\\\displaystyle-\ff{\sq3}{2}&
\displaystyle-\sq3}
}
which is negative definite.
therefore it's a local maxima $f=3\sq3/2$\\
On the boundary, we can check by hand that the local
minima:
\eq{
(\ff{\pi}{2},\,\ff{\pi}{2})
}
and the local minima:
\eq{
(0,0)
}
which is also a global minima.\\
In conclusion, we have
\begin{itemize}
\item[] Local minima:
\eq{
(0,0)\quad(\ff{\pi}{2},\,\ff{\pi}{2})
}
\item[]Local maxima:
\eq{
\left(\ff{\pi}{3},\ff{\pi}{3}\right)
}
\item[]Global minima:
\eq{
(0,0)
}
\item[]Global maxima:
\eq{
\left(\ff{\pi}{3},\ff{\pi}{3}\right)
}
\end{itemize}
}
\ms{5}{
We set $DE=0$ which gives us an extrema:
\eq{
(\ff{2a}{3},2)
}
and the calculated Hess$E$:
\eq{
\mm{-8ae^{-2}&\displaystyle0\\
		\displaystyle0&\displaystyle-\ff{8a^3}{27}e^{-2}}
}
which is negative definite, making it a local maxima.\\
Finally check the boundaries, notice $E$ vanishes when $x=0$ or $x=a$, 
and this shows $({2a}/{3},3)$ is the global maxima, as desired.
}
\ms{6}{
By Lagrange Multipliers:
\eq{
&0+\lam_1-\lam_2\ff{z^2}{x^2y^3}=0\\
&0+\lam_1-3\lam_2\ff{z^2}{y^4x}=0\\
&1+\lam_1+\lam_2\ff{2z}{xy^3}=0\\
&x+y+z-1=0\\
&\ff{z^2}{xy^3}-3=0
} 
which solves for
\eq{
x=\ff{-2\pm+\sq{13}}{9},\quad y=3x,\quad z=\ff{17\pm4\sq{13}}{9}
}
and the desired maximum is (since $f$ must attains its maximum on the set):
\eq{
f_m=z_m=\ff{17+4\sq{13}}{9}
}
}
\ms{7}{
This can be tackled by:
\eq{
V=h\pi r^2\implies h=\ff{V}{\pi r^2}
}
substituting it into the surface area:
\eq
{
V=h(2\pi r)+2\pi r^2=\ff{2V}{r}+2\pi r^2\ge3(V^22\pi)^{1/3}
}
and the equality holds only if $h=(V/\pi)^{1/3}2^{2/3}$
}
\ms{8}{
\item[i)] Lagrange Multipliers:
\eq{
&8x-5y-8+\lam=0\\
&6y-5x-\lam=0\\
&x+y-4=0
}
gives (the only) extrema
\eq{
x=\ff{13}{6},\quad y=\ff{11}{6}
}
\item[ii)] Lagrange Multipliers:
\eq
{
&8x-4+\lam=0\\
&18y+6-3\lam=0\\
&x-3y+3=0\\
\implies
&x=-\ff{2}{5},\quad y=\ff{13}{15}
}
\item[iii)] Lagrange Multipliers:
\eq
{
&2(x-1)+2\lam_1+\lam_2=0\\
&2(y+2)+3\lam_1+\lam_2=0\\
&2(z-2)+2\lam_2=0\\
&2x+3y-1=0\\
&x+y+2z-4=0
}
and solves for:
\eq{
x=\ff{91}{53},\quad y=-\ff{43}{53},\quad z=\ff{82}{53}
}
}
\end{document}