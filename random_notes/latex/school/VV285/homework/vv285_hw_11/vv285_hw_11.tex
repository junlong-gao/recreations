\documentclass{article}
\input mla
\begin{document}
\titlehm{11}
\ms{1}
{
\item[i)]
\eq{
\int_E\,dV=\int_{-c}^c\,dz\int_{\ff{x^2}{a^2(1-z^2/c^2)}+\ff{y^2}{b^2(1-z^2/c^2)}=1}\,dxdy=\int_{-c}^cab\left(1-\ff{z^2}{c^2}\right)\pi\,dz=\ff{4}{3}abc\pi
}
\item[ii)]
\eq{
\int_{B^4}\,dV =\int_0^1\,dr\int_0^{\pi}\,d\ta_1\int_0^{\pi}\,d\ta_2
\int_0^{2\pi}r^3\sin^2\ta_1\st_2\,d\ta_3=\pi^2/2
}
}
\ms{2}
{
\item[i)]
\eq{
U(p)=U(0,0,r)=-\ff{M}{r}G
}
\item[ii)]
\eq{
U(p)=U(0,0,r)=-2\pi G\rho(R^2_b-R^2_a)=-\ff{3}{2}GM\left(\ff{R^2_b-R^2_a}{R_b^3-R_a^3}\right)
}
\item[iii)]
\eq{
U(p)=U(0,0,r)=-\ff{GM}{2}\left(\ff{3}{R}-\ff{r^2}{R^3}\right)
}
}
\ms{3}
{
\item[i)]
\eq{
V(p)=V(0,0,d)=
\begin{cases}
\ff{\rho}{2\epsilon_0}\left(R^2-\ff{1}{3}r^2\right),\quad{0<r<R}\\
\ff{\rho R^3}{3\epsilon_0 r},\quad{R\leq r}
\end{cases}
}
\item[ii)]
\eq{
W_e=\ff{3}{20}\ff{Q^2}{\pi\epsilon_0 R}
}
\item[iii)]
\eq{
\ff{3}{20}\ff{Q^2}{\pi\epsilon_0 R}=m_ec^2\Longrightarrow R=\ff{1}{m_ec^2}
\ff{3}{5}\ff{Q^2}{4\pi\epsilon_0}=1.68\times 10^{-15}\m
}
}
\ms{4}
{
We setup the integral:
\item[]
\eq{
V(0,0,r)=\ff{1}{4\pi\e_0}\int_0^{\pi}\,d\ta\int_0^{R}\rho_c\ff{\rho\,d\rho}{\sq{\rho^2+r^2}}-\ff{1}{4\pi\e_0}\int_{\pi}^{2\pi}\,d\ta\int_0^{R}\rho_c\ff{\rho\,d\rho}{\sq{\rho^2+r^2}}=0
}
This is obviously true: if we move a charged particle along the $z$ axis the force is alway perpendicular to the motion and no potential change here.
}
\mp{5}
{
First on $x$:
\eq
{
\int_0^1\int_0^1\,f(x,y)\,dxdy=\int_0^1
\left(\int_0^yy^{-2}\,dx+\int_y^1-x^{-2}\,dx\right)\,dy=1
}
and on $y$:
\eq
{
\int_0^1\int_0^1\,f(x,y)\,dydx=\int_0^1
\left(\int_0^x-x^{-2}\,dy+\int_x^1y^{-2}\,dx\right)\,dx=-1
}
We observe that the function is unbounded in the every neighborhood of $(0,0)$, thus it does not contradict Fubini's  theorem because it is not
integrable in the region. (Of course the unbounded behavior does not imply it is not integrable, but since Fubini's law doesn't apply, this must be the case).
}
\mp{6}
{
\item[]
We start by considering the implicit mapping $(x,y)\mapsto(p,q)$ defined by
\eq
{
&x^2+y^2-2px-2\sq{r^2-p^2}y=0\\
&x^2+y^2-2qy-2\sq{r^2-q^2}x=0
}
and from above we have:
\eq
{
\pd{p}{x}=\ff{x-p}{x-py/q},\quad\pd{q}{y}=\ff{y-q}{y-qx/p}
}
therefore:
\eq
{
\pd{p}{x}+\pd{q}{y}=\ff{qx-pq}{qx-py}+\ff{py-pq}{py-px}=1
}
}
\ms{7}
{
\item[i)]
Consider the parameterization:
\eq
{
\Gamma_1:x=t,\,y=t^2,\, t:0\to 1\quad\Gamma_2:x=t,\,y=t,\,t:1\to0
} 
Then
\eq
{
\int_{\partial R}=\int_{\Gamma_1}+\int_{\Gamma_2}=\ff{3}{4}+\ff{1}{5}-1=-\ff{1}{20}
}
\item[ii)]
By Green's theorem:
\eq
{
\int_{\partial R}(xy+y^2)\,dx+x^2\,dy=\int_{R}2x-x-2y\,dxdy=\int_0^1\,dx\int_{x}^{x^2}x-2y\,dy=-\ff{1}{20}
}
}
\ms{8}
{
\item[i)]
\eq
{
\ff{1}{4}A=\int_0^a(a^{2/3}-x^{2/3})^{3/2}\,dx=\ff{1}{32}a^2\pi
}
thus the area is 
\eq
{
A=\ff{3}{8}a^2\pi
}
\item[ii)]
By Green's theorem we have:
\eq
{
2\times\ff{1}{4}A=\int_{{\rm astroid}, x>0,\,y>0}x\,dy-y\,dx+\int_a^00\,dy
+\int_{0}^a-0\,dx
}
consider parameterization:
\eq
{
x=a\cos^3\ta,\,y=a\sin^3\ta,\quad \ta:0\to\pi/2
}
notice the path along the $x$ and $y$ axis has zero contribution to the line integral, we have:
\eq{
\ff{1}{2}A=a^2\int_0^{\pi/2}3\cos^4\ta\sin^2\ta+3\sin^4\ta\cos^2\ta\,d\ta=\ff{3}{16}a^2\pi
}
which reproduces 
\eq
{
A=\ff{3}{8}\pi a^2
}
}
}
\end{document}