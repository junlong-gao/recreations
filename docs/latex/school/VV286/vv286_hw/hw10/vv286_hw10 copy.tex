


\input mla.tex

\begin{document}
\begin{vv286}{9}
  \begin{vv286_mp}{1}
  \item[(i)]
    By definition:
    \begin{align*}
      {\L}[f](p)&=\int_{0}^{+\infty}f(t)e^{-tp}\, dt\\
      &=\int_0^{T}f(t)e^{-tp}\, dt+\int_{T}^{+\infty}f(t)e^{-tp}\, dt\\
      &=\int_0^{T}f(t)e^{-tp}+e^{-tp}\L[f](p)
    \end{align*}
    and we solve for $\L[f](p)$ to obtain
    \begin{equation*}
      \L[f](p)=\frac{1}{1-e^{-pT}}\int_0^{T}f(t)e^{-tp}
    \end{equation*}
   \item[(ii)]
     \begin{align*}
       \L[f](p)=\ff{1}{1-e^{-pT}}\int_0^{T}ate^{-tp}=a\left(\ff{1}{p^2}-\ff{1}{p}\frac{e^{-p}}{1-e^{-p}}\right)
     \end{align*}
  \end{vv286_mp}
  \begin{vv286_mp}{2}
  \item[(i)]h
    We first notice that the residues are at:
    \begin{align*}
      p_{-1}=\frac{1}{2}&
      p_{k}=(2k+1)i\pi\quad k\in\Z
    \end{align*}
    thus we let $\b>0$, and notice that the integral exists along it, we
    calculate the residue:
   \begin{align*}
     &\res{f}{0}=\frac{1}{2}\\
     &\res{f}{p_k}=-\frac{e^{p_kt}}{p_k}
   \end{align*}
   and we pair $k$ with $-k-1$ to get:
   \begin{align*}
     \frac{e^{p_kt}}{p_k}+\frac{e^{p_{-k-1}t}}{p_{-k-1}}=
     \frac{1}{(2k+1)i\pi}e^{(2k+1)i\pi t}-\frac{1}{(2k+1)i\pi}e^{(-2k-1)i\pi
     t}
     =\frac{2}{\pi}\frac{\sin (2k+1)\pi t}{2k+1}
   \end{align*}
   for $k\ge0$.\\
   Close the integral, and notice by Jordan's Lemma the integral on the
   greater circle vanishes. We therefore obtain:
   \begin{align*}
     \frac{1}{2\pi i}\int_{\b-i\infty}^{\b+i\infty}e^{pt}F(p)\, dp=
     \frac{1}{2}-\frac{2}{\pi}\sum_{k=0}^{\infty}\frac{\sin (2k+1)\pi t}{2k+1}
   \end{align*}
 \item[(ii)]
   Again we calculate the residue by pairing $k$ with $-k-1$:
   \begin{align*}
     &\res{f}{0}=1\\
     &\res{f}{p_k}+\res{f}{p_{-k-1}}=\frac{4}{(-1)^{k+1}\pi}\frac{\cos p_kt}{2k+1}\quad
     p_k=\frac{2k+1}{2}i\pi, \, k\in Z
   \end{align*}
    Close the integral, and notice by Jordan's Lemma the integral on the
   greater circle vanishes. We therefore obtain:
 \begin{align*}
     \frac{1}{2\pi i}\int_{\b-i\infty}^{\b+i\infty}e^{pt}F(p)\, dp=
     1-\frac{4}{\pi}\sum_{k=0}^{\infty}\frac{(-1)^k}{2k+1}\cos \frac{(2k+1)\pi t}{2}
   \end{align*}
  \end{vv286_mp}
  \begin{vv286_ms}{3}
  \item[]
    From exercise 1 we have:
    \begin{align*}
      \L[f](p)=\frac{2}{p}\frac{1}{1+e^{-p}}=\frac{1}{p}(1+\frac{e^{p/2}-e^{-p/2}}{e^{p/2}+e^{-p/2}})=\frac{1+\tanh(p/2)}{p}
    \end{align*}
    And the series expansion:
   \begin{align*}
     f(t)=\L^{-1}L[f](t)=y=1+\frac{4}{\pi }\sum_{k=1}^{30}{\frac{\sin \left( \left( 2k-1 \right)\pi x \right)}{2k-1}}
   \end{align*}
   See the plots in the attachment.
   (Note: careful in pairing since we need to make sure $k$ and
   $-k+1$ enumerates all possible integers, with NO repetition.)\\
   Observation:
Notice the functions converges to the square-wave function in the {\bf open} set $(0,2)$, and the it's safe to notice that for all sufficiently small $\epsilon>0$, on $(0+\epsilon,  1-\epsilon)$ and $(1+\epsilon,  2-\epsilon)$ the convergence is uniform, while near boundary the convergence is no longer uniform. For example, when $x=0$ the approximation always gives the wrong result $y=1$ while the number of approximation increases the uniform convergence (hence point wise convergence)region increases (approaching to $(0,1)$ and $(1,2)$)

  \end{vv286_ms}
  \begin{vv286_ms}{4}
  \item[(i)]
    The solution is
    \begin{align*}
      y=\frac{\sin t}{2}-\frac{t\cos t}{2}+H(t-\pi)\sin (t-\pi)
    \end{align*}
   \item[(ii)]
     The solution is
     \begin{align*}
       y=&\frac{2}{\sq{3}}e^{-1/2t}\sin (\frac{\sq3}{2}t+\ff{\pi}{3})\\
       &\quad+\frac{4}{\sq3}H(t-1)e^{-1/2(t-1)}\sin \frac{\sq3}{2}(t-1)+
       \frac{2}{\sq3}H(t-2)e^{-1/2(t-2)}\sin \frac{\sq3}{2}(t-2)
     \end{align*}
  \end{vv286_ms}

  \begin{vv286_ms}{5}
  \item[(i)]
    By Laplace transformation we have the solution
    \begin{align*}
      y=\sum_{j=0}^{\infty}H(t-j\pi)\sin (t-j\pi)
    \end{align*}
    and we notice that when $2k\pi<t<(2k+1)\pi$, the last term
    of none zero is:
    \begin{align*}
      H(t-2k\pi)\sin (t-2k\pi)=\sin  t
    \end{align*}
    and this cancels out the previous term:
    \begin{align*}
      H(t-(2k-1)\pi)\sin (t-(2k-1)\pi)=-\sin t
    \end{align*}
    This cancellation goes all the way to the very first term
    that survives:
    \begin{align*}
      H(t)\sin t=\sin t
    \end{align*}
    and for $(2k+1)\pi<t<(2k+2)\pi$, we adopt the previous
    result:
    \begin{align*}
      y=\sum=\sin t+H(t-(2k+1)\pi)\sin (t-(2k+1)\pi)=\sin t-\sin t=0
    \end{align*}
    as desired.
    In conclusion:
\begin{equation*}
 y(t)= \left\{\begin{aligned}
    &\sin t &\text{n is even}\\
    &0 &\text{n is odd}
       \end{aligned}
 \right.
 \qquad
 \end{equation*}
\item[(ii)]
  The solution is
  \begin{align*}
    y(t)=\sum_{j=0}^{\infty}H(t-2j\pi)\sin (t-2j\pi)
  \end{align*}
  and, we see that when $2n\pi<t<(2n+2)\pi$, the last none zero
  term is $H(t-2n\pi)\sin (t-2n\pi)=\sin t$, and there are
  $0,1, \ldots n$ contributions, namely $n+1$, all takes on the
  effective value $\sin t$, $t\in[0, 2\pi]$, resulting
  \begin{align*}
    y=(n+1)\sin t \quad 2n\pi<t<2(n+1)\pi
  \end{align*}
  \end{vv286_ms}
  \begin{vv286_mp}{6}
  \item[(i)]
    By definition:
    \begin{align*}
      \widehat{\Pi}(\xi)&=\frac{1}{\sqrt{2\pi}}\int_{-\infty}^{+\infty}\Pi(x)e^{-i\xi
      x}\, dx\\
      &=\frac{1}{\sqrt{2\pi}}\int_{-1}^{1}e^{-i\xi x}\, dx\\
      &=\frac{1}{\sqrt{2\pi}}\frac{1}{i\xi}(e^{i\xi}-e^{-i\xi})\\
      &=\sqrt{\frac{2}{\pi}}\frac{\sin \xi}{\xi}
    \end{align*}
   \item[(ii)]
     We are to evaluate:
     \begin{align*}
       \int_{\R}e^{ix\xi}\frac{\sin \xi}{\xi}\, d\xi
       \numberthis{eq:1}
     \end{align*}
     We first notice that:
     \eq
     {
     \ff{\sin z}{z}=\ff{1}{2zi}(e^{iz}-e^{-iz})
     }
     does not have a decay at infinity (consider $z=ia$, $a\in \mathbb{R}$), {\bf thus Jordan lemma doesn't work for $\sin z/z$}.\\
     By separating the integrand:
     \begin{align*}
       \int\frac{\sin z}{z}e^{izx}=\int\frac{1}{2iz}
       e^{iz(x+1)}-\int\frac{1}{2iz}e^{iz(x-1)}
       \numberthis{eq:3}
     \end{align*}
    If $x>1$ we choose the path rotating in the upper
     half plane (for $x>1$) around the greater circle
     $C_R^+$ and near the origin rotating clockwise
     $C_{\varepsilon}^-$ to close the path {\bf in the upper half
     plane} (thus no residue inside the path) and apply Jordan's lemma to $e^{iz(x+1)}$ and $e^{iz(x-1)}$ separately on the greater circle. (Note now we can use Jordan's lemma), then the integral on the great circle vanishes. 
     Then, we have 
     \begin{align*}
       0=\oint\frac{e^{iz(x+1)}}{2iz}\,dz=\int_{-R}^{+R}+\int_{C_{\varepsilon}^-}
       \numberthis{3}
     \end{align*}
Note the path on $C_{\varepsilon}^-$ can be tackled by noticing that 
\eq
{
\frac{e^{iz(x+1)}}{2iz}=\ff{1}{2iz}+h(z)
}
where $h(z)$ is holomorphic including $z=0$ by using Laurent expansion around 0. Thus the integral:
\eq{
 \left|\int_{C_{\varepsilon}^-}\frac{e^{iz(x+1)}}{2iz}\,dz\right|=\left|\int_{C_{\varepsilon}^-}\ff{1}{2iz}+h(z)\,
       dz\right|=\left|\int_{C_{\varepsilon}^-}\ff{1}{2iz}\right|+\e |h|\to\left|\int_{C_{\varepsilon}^-}\ff{1}{2iz}\right|
       }
       therefore 
       \eq
       {
       \int_{C_{\varepsilon}^-}\frac{e^{iz(x+1)}}{2iz}\,dz=\int_{C_{\varepsilon}^-}\ff{1}{2iz}=-\ff{\pi}{2}
       }
       Since the path for two integrand is {\bf the same}, we have:
       \eq
       {
       \int_{C_{\varepsilon}^-}\frac{e^{iz(x-1)}}{2iz}\,dz=\int_{C_{\varepsilon}^-}\ff{1}{2iz}=-\ff{\pi}{2}
       }
       Back to \eqref{3} and plugging in \eqref{eq:3} we have:
       \eq
       {
       \int\frac{\sin z}{z}e^{izx}=0\quad\text{$|x|>1$}
       }
     On the other hand, if $-1<x<1$, we again separate the 
     integrand:
     \begin{align*}
       \frac{\sin z}{z}e^{izx}=\frac{1}{2iz}\left(
       e^{iz(x+1)}-e^{iz(x-1)}
       \right)
       \numberthis{eq:3}
     \end{align*}
     and the first exponent has positive
     coefficient $x+1$ whereas the second exponent has the {\bf negative}
     coefficient $x-1$. Now, 
     \begin{align*}
       \int_{-\infty}^{+\infty}\frac{\sin z}{z}e^{izx}=\int_{-\infty}^{+\infty}\frac{e^{iz(x+1)}}{2iz}\,dz-
       \int_{-\infty}^{+\infty}\frac{e^{iz(x-1)}}{2iz}
       \numberthis{3}
     \end{align*}
     and let the first path to be rotating around the greater circle
     $C_R^+$ and near the origin rotating clockwise
     $C_{\varepsilon}^-$ to close the path {\bf in the upper half
     plane} (thus no residue inside the path). The second path is the symmetric one {\bf in the lower
     half plane.}(note the path for two integrand is different, unlike when $|x|>1$, to make Jordan lemma work, and this is the sole reason why the integration is different.)\\
     Then, we have 
     \begin{align*}
       \oint\frac{e^{iz(x+1)}}{2iz}\,dz=\int_{-R}^{+R}+\int_{C_R^+}+\int_{C_{\varepsilon}^-}=0
     \end{align*}
     Again the integral on the greater circle vanishes as
     $R\to+\infty$, the
     function is homomorphic in the enclosed path, and:
     \begin{align*}
       \int_{C_{\varepsilon}^-}\frac{1}{2iz}
       \,dz=-\frac{\pi}{2}
     \end{align*}
  as argued previously.\\
       Thus we have
       \begin{align*}
	 \int_{-\infty}^{+\infty}\frac{e^{iz(x+1)}}{2iz}\,dz=\frac{\pi}{2}
       \end{align*}
       A similar procedure, yet performed on the lower half
       plane gives:
 \begin{align*}
	 \int_{-\infty}^{+\infty}\frac{e^{iz(x-1)}}{2iz}\,dz=-\frac{\pi}{2}
       \end{align*}
       Plugging the inverse transform: 
       \begin{align*}
	 \frac{1}{\sqrt{2\pi}}\int_{\R}e^{ix\xi}\widehat{\Pi}({\xi})\,
	 d\xi=\frac{1}{\pi}\int_{\R}e^{ix\xi}\frac{\sin \xi}{\xi}\, d\xi
	 =\frac{1}{2}+\frac{1}{2}=1
       \end{align*}
       Finally, if $x=1$, we tentatively notice that in the
       previous procedure:
       \begin{align*}
	 \int_{-\infty}^{+\infty}\frac{e^{iz(x-1)}}{2iz}\,dz=0
       \end{align*}
       since it is odd on the path of integration,
       and the total integral will leave us only $ \int_{-\infty}^{+\infty}\frac{e^{iz(x+1)}}{2iz}\,dz=\frac{\pi}{2}$, thus
 \begin{align*}
	 \frac{1}{\sqrt{2\pi}}\int_{\R}e^{ix\xi}\widehat{\Pi}({\xi})\,
	 d\xi=\frac{1}{\pi}\int_{\R}e^{ix\xi}\frac{\sin \xi}{\xi}\, d\xi
	 =\frac{1}{\pi}
	 \int_{-\infty}^{+\infty}\frac{e^{iz(x+1)}}{2iz}\,dz=\frac{1}{2}
       \end{align*}
and for similar reason we have when $x=-1$:
 \begin{align*}
	 \frac{1}{\sqrt{2\pi}}\int_{\R}e^{ix\xi}\widehat{\Pi}({\xi})\,
	 d\xi=\frac{1}{\pi}\int_{\R}e^{ix\xi}\frac{\sin \xi}{\xi}\, d\xi
	 =-\frac{1}{\pi}
	 \int_{-\infty}^{+\infty}\frac{e^{iz(x-1)}}{2iz}\,dz=\frac{1}{2}
       \end{align*}
       and this completes the proof.
  \end{vv286_mp}
   \begin{vv286_ms}{7}
       \item[]
	 We integrate by parts:
	 \eq
	 {
	 \widehat{f}(\xi+i\eta)&=\ff{1}{\sq{2\pi}}\int_0^{\infty}x^2e^{(\eta-i\xi)}\,dx\\
	 &=\ff{e^{cx}}{\sq{2\pi}}\left(\ff{x^2}{c}-\ff{2x}{c^2}+\ff{2}{c^3}\right)\Big|_{0}^{\infty}\quad\text{$c=\eta-i\xi$}
	 }
	 and notice that the integral
	 converges only if $\eta<0$
	 \begin{align*}
	   \widehat{f}(\xi+i\eta)=-\frac{2}{\sq{2\pi}(\eta-i\xi)^3}
	 \end{align*}
       \end{vv286_ms}
       \begin{vv286_ms}{8}
    \item[(i)]
      We assume the solution has the series expansion:
      \begin{align*}
	x(t)=\sum_{k=0}^{\infty}a_kt^k
      \end{align*}
      Then:
      \begin{align*}
	x''&=\sum_{k=0}^{\infty}a_{k+2}(k+2)(k+1)t^k\\
	-2 x'&=\sum_{k=0}^{\infty}(-2)a_kkt^k\\
	\lam x&=\sum_{k=0}^{\infty}\lam a_k t^k
      \end{align*}
      plugging into the differential equation we have:
      \begin{align*}
	&2a_2+\lam a_0=0\\
	&a_{k+2}(k+2)(k+1)-2a_kk+\lam a_k=0\quad
	k\ge2\numberthis{1}
      \end{align*}
      Thus $a_0$ and $a_1$ yields two independent solutions:
      \begin{align*}
	&x_1(t)=a_1\sum_{k=0}^{\infty}\frac{(2-\lam)(6-\lam)\cdots
	(4k+2-\lam)}{(2k+1)!}t^{2k+1}\\
	&x_2(t)=a_0+a_0\sum_{k=1}^{\infty}\frac{(-\lam)(4-\lam)\cdots(4k-\lam)}{(2k)!}t^{2k}
      \end{align*}
    \item[(ii)]
      This follows from the solution that if $\lam=2n$, the
      first
      series vanishes at $k=(2n-2)/4=(n-1)/2$, the second
      vanishes at $k=2n/4=n/2$. In both cases we have the
      exponent cut at $n$, leaving us a polynomial of degree
      $n$.
    \end{vv286_ms}
    \begin{vv286_ms}{9}
    \item[(i)]
    First we inspect a solution:
      \begin{align*}
	&y_1=e^t\\
      \end{align*}
      And consider the reduction of order:
      \begin{align*}
	y=y_1v
      \end{align*}
      Plugging in the equation gives us a second independent solution:
      \eq
      {
      	&y_2=e^t\int \frac{e^{-t}}{\sqrt{t}}
	}

    \item[(ii)]
    We assume the solution has the series expansion:
      \begin{align*}
	y(t)=\sum_{k=0}^{\infty}a_kt^k
      \end{align*}
      Then:
      \begin{align*}
	t^2y''&=\sum_{k=0}^{\infty}a_{k+2}(k+2)(k+1)t^{k+2}\\
	(t-t^2) y'&=\sum_{k=0}^{\infty}a_kk(t^{k+1}-t^{k+2})\\
	-y&=\sum_{k=0}^{\infty}- a_k t^k
      \end{align*}
      plugging into the differential equation we have:

      \begin{align*}
	&y_1=\frac{e^{t}}{t}\\
      \end{align*}
       And consider the reduction of order:
      \begin{align*}
	y=y_1v
      \end{align*}
      Plugging in the equation gives us a second independent solution:
      \eq{
      	&y_2=1+\frac{1}{t}
	}

   \item[(iii)]
   We assume the solution has the series expansion:
      \begin{align*}
	y(t)=\sum_{k=0}^{\infty}a_kt^k
      \end{align*}
      Then:
      \begin{align*}
	t^2y''&=\sum_{k=0}^{\infty}a_{k+2}(k+2)(k+1)t^{k+2}\\
	-(t+t^2) y'&=\sum_{k=0}^{\infty}-a_kk(t^{k+1}+t^{k+2})\\
	y&=\sum_{k=0}^{\infty} a_k t^k
      \end{align*}
      plugging into the differential equation we have:

      \begin{align*}
       &y_1=te^t\\
      \end{align*}
     And consider the reduction of order:
      \begin{align*}
	y=y_1v
      \end{align*}
      Plugging in the equation gives us a second independent solution:
      \eq{
y_2=te^t\int \frac{e^{-t}}{t}
	}
    \end{vv286_ms}
\end{vv286}
\end{document}

