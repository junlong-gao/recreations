\input mla
\begin{document}
\titlehm{2}
\begin{vv286_ms}{1}
\item[(i)]:
Let $X(t)$ denote the chemical $C$ formed after time t,we plug in the numbers:
\eq
{
\ff{dX}{dt}=k(60-X)(140-X)
}
Given IVP: $X(0)=0$ we obtain the unique solution:
\eq
{
\sin\ff{150-X}{60-X}=\ln\ff{5}{2}+90kt\numberthis{1}
}
comparing the condition $X(5)=10$ we get:
\[
\ln\ff{60-X(20)}{150-X(20)}=4\ln\ff{14}{5}-3\ln\ff{5}{2}\implies X(20)=29.32\,{\rm g}
\]
\item[(ii)]:
From \eqref{1} we immediately get $X\to60$ as $t\to\infty$.
\item[(iii)]:
A remains 0 gram, B remains 30 grams.
\end{vv286_ms}


\begin{vv286_ms}{2}
\item[]
The solution is obtained by consider the substitution: 
\eq
{
u=-\ff{1}{x}
}
and we are left with the ODE of the form $u'=f(u/x)$:
\eq
{
\ff{du}{dt}=2\ff{u^2}{t^2}-1
}
and using the method of substituting $y=u/t$ gives us the original solution  satisfying IVP:
\eq
{
x=\ff{2}{t}
}
\end{vv286_ms}


\begin{vv286_ms}{3}
\item[]
This is a separable ODE if we set
\eq{
u(x)=3x+2y+2
}
and plugging in the initial equation we have:
\eq
{
u'=5-\ff{4}{u}
}
where we shift the IVP:
\eq
{
u(-1)=-3
}
and the unique solution is:
\eq
{
&u+\ff{4}{5}\left( \ln{|5u-4|}-\ln19 \right)=5x+2,
\quad
u=3x+2y+2\\
\implies\quad &15x+10y+6=-19e^{\ff{5}{2}(x-y)}
}
is the desired implicit curve.
\end{vv286_ms}


\begin{vv286_ms}{4}
\item[]
Let $u(t)$ denote the multiple of the population with
respect to the initial population: $y(t)=y_0u(t)$
We also let $t$ in the units of $40$ minutes thus $t=3$
implies after 2 hours\\
Then we have:
\item[(a)]
	\eq
	{
	u_{\gamma}(t)=5\ff{1}{1+4e^{-(\ln 2)t}}\implies
	y(3)=y_0u(3)=3.33\times 10^8
	}
\item[(b)]
	\eq
	{
	u_{\gamma}(t)=\ff{1}{2}\ff{1}{1-1/2e^{-(\ln 2)t}}\implies
	y(3)=y_0u(3)=5.33\times 10^8
	}
\end{vv286_ms}


\begin{vv286_ms}{5}
\item[(i)]
From the separation of variables:
\eq
{
\ln\left|\ff{y-1}{y-1/4}\right|=-3+C
}
Incorporating the absolute value into the constant $C$ and rewrite the constant into $C_0$
	The general solution is
	\eq
	{
	y=\ff{1+C_0e^{3t}}{4C_0e^{3t}+1},\,C_0\in\R
	\quad
	\hbox{or}
	\quad
	y=\ff{1}{4}
	\quad
	\hbox{or}
	\quad
	y=1
	}
\item[(ii)]
	We write
	\eq
	{
	y'=-\ff{1}{4}(y-1)(\ff{1}{4}-y)
	}
	thus the carrying capacity is $1/4$
\item[(iii)]
	The threshold is 
	$1$
\item[(iv)]
	See attachment.
\end{vv286_ms}


\begin{vv286_ms}{6}
\item[(a)]
	The general solution for the model:
	\eq
	{
	p_1=\ff{a}{b}\ff{1}{1+\gamma^{-at}}\\
	p_2=\ff{A}{B}\ff{1}{1+\gamma^{-At}}\\
	}
	We set $p_1(t^*)=q$ and $p_1(t^{**})=Q$, looking
	for the difference $T_1=t^{**}-t^*$ by considering:
	\eq
	{
	p_1(t^*)/p_1(t^{**})=q/Q
	}
	This will yield:
	\eq
	{
	T_1=t^{**}-t^*=\ff{1}{a}\ln\ff{Q(a-bq)}{q(a-bQ)}
	}
\item[(b)]
	The similar procedure follows when (a) is applied
	to $p_2$
	\eq
	{
	p_2(t^*)/p_2(t^{**})=Q/q
	}
	yields:
	\eq
	{
	T_2=t^{**}-t^*=\ff{1}{A}\ln\ff{q(QB-A)}{Q(qB-A)}
	}
\item[(c)]
	For $T_1$ it's trivial by setting $b=0$ in (a),
	for $T_2$, we consider first order linear
	approximation:
	\eq
	{
	T_2&=\ff{1}{A}\ln\ff{Qqb-QA+QA-qA}{QqB-QA}=
	\ff{1}{A}\ln\left( 1+\ff{QA-qA}{Q(qB-A)} \right)\\
	&\quad=
	\ff{1}{A}\left( \ff{QA-qA}{Q(qB-A)}+o(A^2) \right)\\
	&\quad=
	\ff{Q-q}{qQB}+o(A)
	}
	as desired. \\
	By numerical calculations:
	\eq
	{
	a=\ff{\ln\left(\ff{Q}{q}\right)}{T_1}\approx0.978=1
	}
\end{vv286_ms}


\end{document}
