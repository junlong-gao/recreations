\input mla
\begin{document}
\titlehm{5}


\begin{vv286_ms}{1}
\item[]
	We first solve the homogenous equation:
	\eq
	{
	\dot{x}=\mm{2&0&1\\
				0&2&0\\
				0&0&3}
				x
	}
	to obtain a fundamental system:
	\eq
	{
	x=e^{At}\quad A=\mm{2&0&1\\
				0&2&0\\
				0&0&3}
	}
	by finding the Jordan normal form of $A$.
	Then using the method of variation of parameters:
	\eq
	{
	x(t)=e^{A(t-t_0)}x_0+\int_{t_0}^{t}e^{(t-s)A}b(s)\,ds
	}
	The solution:
	\eq
	{
	x(t)=\mm{e^{2(t-t_0)}&0&e^{3(t-t_0)}-e^{2(t-t_0)}\\
	0&e^{2(t-t_0)}&0\\
	0&0&e^{3(t-t_0)}}
	\mm{x_1(t_0)\\x_2(t_0)\\x_3(t_0)}+
	\mm{e^{3t-t_0}\\0\\e^{3t-t_0}-e^{2t}}
	}
\end{vv286_ms}
\begin{vv286_ms}{2}
\item[]
	Consider the system:
	\eq
	{
	{\bf y}(t)=\mm{y(t)\\y'(t)\\y''(t)}
	}
	hence the equation is equivalent to:
	\eq
	{
	{\bf y'}(t)=\mm{y'(t)\\y''(t)\\y'''(t)}=\mm{0&1&0\\
												0&0&1\\
												0&-1&0}\mm{y(t)\\y'(t)\\y''(t)}
												+\mm{0\\0\\ \sec t\tan t}
	}
	with the initial value:
	\eq
	{
	{\bf y}(0)=\mm{0
	\\0\\0}
	}
	We first solve the homogenous equation:
	\eq
	{
	\dot{y}=\mm{0&1&0\\
				0&0&1\\
				0&-1&0}
				y
	}
	to obtain a fundamental system:
	\eq
	{
	x=e^{At}\quad A=\mm{0&1&0\\
				0&0&1\\
				0&-1&0}
	}
	by finding the Jordan normal form of $A$.
	Then using the method of variation of parameters:
	\eq
	{
	y(t)=e^{A(t-t_0)}y_0+\int_{t_0}^{t}e^{(t-s)A}b(s)\,ds
	}
	The solution:
	\eq
	{
	y(t)=\ff{1}{\cos  t}+\cos  t\ln\cos t+\sin t(t-\tan t)-1
	}
\end{vv286_ms}
\begin{vv286_mp}{3}
\item[]
	We set $\det\left( A-\lam I \right)=0$ to get $\lam=-b/2a$ with algebraical multiplicity
	$2$. Now if $b=0$ it's trivial that 
	\eq
	{
	\mm{0&1\\0&0}
	}
	is not diagonalisable since $\lam=0$ there is no non-trivial solution.\\
	if $b\neq 0$, then we have 
	\eq
	{
	V_{\lam}={\rm span}\mm{-\ff{2a}{b}\\1}
	}
	with geometric multiplicity $1<2$. Thus either case it's not  diagonalizable. 
\end{vv286_mp}
\begin{vv286_ms}{4}
\item[(i)]
	We consider the infinitesimal increment:
	\eq
	{
	\Delta Q_1=\ff{3}{2}\Delta t-3\Delta t\ff{Q_1}{V_1}+1.5\ff{Q_2}{V_2}\\
	\Delta Q_2=3\Delta t+3\Delta t\ff{Q_1}{V_1}-4\Delta t\ff{Q_2}{V_2}
	}
	Notice the volume stays the same $V_1=30$, $V_2=20$.\\
	Dividing both side by $\Delta t$ and let the increment of time approaching $0$ we get the
	desired differential equations:
	\eq
	{
	Q_1'(t)&=\ff{3}{2}-\ff{1}{10}Q_1+\ff{3}{40}Q_2\\
	Q_2'(t)&=3+\ff{1}{10}Q_1-\ff{1}{5}Q_2
	}
\item[(ii)]
	We have the linear system:
	\eq
	{
	\mm{Q'_1(t)\\Q'_2(t)}
	=
	\mm{-1/10&3/40\\
		1/10&-1/5}
		\mm{Q_1(t)\\Q_2(t)}
		+
		\mm{3/2\\3}
	}
	We first solve the homogenous equation:
	\eq
	{
	\dot{Q}(t)=\mm{-1/10&3/40\\
				1/10&-1/5}
				Q(t)
	}
	to obtain a fundamental system:
	\eq
	{
	Q(t)=e^{At}\quad A=\mm{-1/10&3/40\\
					1/10&-1/5}
	}
	by finding the Jordan normal form of $A$.
	Then using the method of variation of parameters:
	\eq
	{
	Q(t)=e^{A(t-t_0)}Q(0)+\int_{t_0}^{t}e^{(t-s)A}b(s)\,ds
	}
	and the initial:
	\eq
	{
	Q(0)=\mm{25\\15}
	}
	Solutions:
	\eq
	{
	\mm{Q_1(t)\\Q_2(t)}=\mm{\ff{29}{8}e^{-t/4}-\ff{165}{8}e^{-t/20}+42\\
	-\ff{55}{4}e^{-t/20}-\ff{29}{4}e^{-t/4}+36}
	}
\end{vv286_ms}
\begin{vv286_ms}{5}
\item[(i)]
	The general solution:
	\eq
	{
	y=c_1\left( t+\ff{2}{t+1}+c_2 \right)(t+1)\quad c_1,\,c_2\in\R
	}
\item[(ii)]
	The general solution:
	\eq
	{
	y=\ff{1}{\sq{t}}(c_1\sin t+c_2\cos t)\quad c_1,\,c_2\in\R
	}
\end{vv286_ms}
\begin{vv286_mp}{6}
\item[]
	Plugging the numbers and ignore the buoyant force (positive direction along gravity)
	\eq
	{
	\ddot{u}+2\dot{u}+u=0
	}
	where we set $u=x+\ff{mg}{k}$ and $x$ is measured from the displacement of the spring.
	Then the solution is:
	\eq
	{
	u=\left( -\ff{1}{4}+\ff{5}{4}t \right)e^{-t}
	}
	notice that $u(1/5)=0$ (equilibrium), $u\to0$ as $t\to\infty$ and $\dot{u}<0$ for sufficiently large $t$.
\end{vv286_mp}
\begin{vv286_ms}{7}
\item[]
	The driven force can be decomposed into 
	\eq
	{
		F(t)=\cos^3\w t=\ff{1}{4}\cos3\w s+\ff{3}{4}\cos\w s
	}
	and
	\eq
	{
		\w_0=\sq{k/m}=4
	}
	by superposition principle, we see that resonance occur at
	\eq
	{
		\w=\w_0,\quad 3\w=\w_0
	}  
	Thus the frequencies are $4$ and $4/3$.
\end{vv286_ms}
\begin{vv286_ms}{8}
\item[]
	We solve the particular solution:
	\eq
	{
	y&=\ff{150\sq{2}}{\a}e^{-2/3t}\sin \ff{\sq2}{3}\a t\\
	y'&=\ff{150\sq{2}}{\a}e^{-2/3t}\left( -\ff{2}{3}\sin \ff{\sq2}{3}\a t+\ff{\sq2}{3}\cos \ff{\sq2}{3}\a t\right)
	}
	and set the quantity $y^2(1)+y'^2(1)<0.01$ yields $\a>6$ (round off to integers).
\end{vv286_ms}
\begin{vv286_ms}{9}
\item[]
	We have the second order ODE:
	\eq
	{
	m\ddot{x}+c\dot{x}+x=3\cos t
	}
	and solve the system, plugging the numbers:$m=4$, $k=1$,$\gamma=c$, $\w=1/2$ and $F_0=3$, we
	have $R=3/\sq{9+c^2}\leq0.5$, $\iff c_{\rm mim}=3\sq3$.
\end{vv286_ms}










\end{document}
