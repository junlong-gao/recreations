%lead
\input my_macros
\mydoc
\baselineskip 16pt plus 2pt
\vv8 November {18}
%

%problem 1
\snum1
\noindent1) $$\sum_{n=1}^{\infty}{2^{2n}3^{3n}\over5^{3n}}=\sum_{n=1}^{\infty}\Big({108\over125}\Big)^n$$
Now, since it's geometric series with $|q|<1$, it {\bf converges}.
\medskip
\noindent2) $$ {n+4\over n^2-3n+1}\cdot n={n^2+4n\over n^2-3n+1}\to 1$$
So by comparison test $\sum{n+4\over n^2-3n+1}$ and  $\sum{1\over n}$ shares same series behavior. Since  $\sum{1\over n}$ diverges, $\sum{n+4\over n^2-3n+1}$ {\bf diverges}.
\medskip
\noindent3) $$ {(n+1)^4\over 3^{(n+1)}}\cdot{3^n\over n^4}={(1+{1\over n})^4\over 3}\to{1\over3}<1$$
So by ratio test the series {\bf converges}.
\medskip
\noindent4) $${2^{n+1}\over (n+1)!}\cdot{n!\over2^n}={2\over n+1}\to 0<1$$
So by ratio test the series {\bf converges}.
\medskip
\noindent5) $${ \root n \of {\left(2\over n\right)^n}}={2\over n}\to 0<1$$
So by root test the series {\bf converges}.
\medskip
\noindent6) $$ {n\over 10n^3-100}\cdot n^2={n^3\over10n^3-100}\to{1\over10}$$
So by comparison test $\sum{n\over 10n^3-100}$ and  $\sum{1\over n^2}$ shares same series behavior. Since  $\sum{1\over n^2}$ converges, $\sum{n\over 10n^3-100}$ {\bf converges}.
\bigskip
%problem 2
\pnum2
\noindent i) Given $x-p>0$, we can find $N>0$ such that for all $n>N$, we have
$$
\left|{a_{n+1}\over a_n}-p\right|<x-p
$$
Then for any $n>N$, we conclude:
$$
|a_n|\left|{a_{n+1}\over a_n}-p\right|<|a_n||x-p|\Longrightarrow \left|a_{n+1}-pa_n\right|<|x-p||a_n|
$$
By triangle inequality we get
$$|a_{n+1}|\le\left|a_{n+1}-pa_n\right|+|pa_n|<|xa_n|<x^{n-N+1}|a_N|
$$
Take nth root of both side, noticing all terms are positive:
$$
{\root n \of{ a_{n}}}<x^{n-N\over n}{\root n \of {a_N}}\le\max(x^{n-N\over n},x)\cdot \max(1,{\root n \of {a_N}})
$$
So for sufficiently large $n>0$, we have
$$
\sup_{m>n}\root m \of {a_m}\le\max(\lim_{m\to \infty}x^{m-n\over m},x)\cdot \max(1,\lim_{m\to \infty}{\root m \of {a_m}})=x
$$
So we conclude 
$$
\varlimsup_{n\to\infty}{a_{n+1}\over a_n}<x
$$
for any $x>p>0$ (so the equality sign never holds), so $p$ is the least upper bound of $\displaystyle\varlimsup_{n\to\infty}{a_{n+1}\over a_n}$. So $\displaystyle\varlimsup_{n\to\infty}{a_{n+1}\over a_n}\le p$.
\medskip
\noindent ii)Given $p-x>0$, we can find $N>0$ such that for all $n>N$, we have
$$
\left|{a_{n+1}\over a_n}-p\right|<p-x
$$
Then for any $n>N$, we conclude:
$$
|a_n|\left|{a_{n+1}\over a_n}-p\right|<|a_n||p-x|\Longrightarrow \left|a_{n+1}-pa_n\right|<|p-x||a_n|
$$
Taking off the absolute value we have
$$a_{n+1}>-|p-x||a_n|+|pa_n|=|xa_n|>x^{n-N+1}|a_N|
$$
Take nth root of both side, noticing all terms are positive:
$$
{\root n \of{ a_{n}}}>x^{n-N\over n}{\root n \of {a_N}}\le\min(x^{n-N\over n},x)\cdot \min(1,{\root n \of {a_N}})
$$
So for sufficiently large $n>0$, we have
$$
\inf_{m>n}\root m \of {a_m}\ge\min(\lim_{m\to \infty}x^{m-n\over m},x)\cdot \min(1,\lim_{m\to \infty}{\root m \of {a_m}})=x
$$
So we conclude 
$$
\varliminf_{n\to\infty}{a_{n+1}\over a_n}>x
$$
for any $0<x<p$ (so the equality sign never holds), so $p$ is the greatest lower bound of $\displaystyle\varliminf_{n\to\infty}{a_{n+1}\over a_n}$. So $\displaystyle\varliminf_{n\to\infty}{a_{n+1}\over a_n}\ge p$.
\smallskip
By assignment 3 exercise 1,  $\displaystyle\varliminf_{n\to\infty}{a_{n+1}\over a_n}\le\lim_{n\to\infty}{a_{n+1}\over a_n}\le\varlimsup_{n\to\infty}{a_{n+1}\over a_n}$, so the limit exists and equals to  $p$.
\medskip
\noindent iii) When $p=0$, we have, for $\forall \varepsilon>0$, $\exists N>0$ such that $\forall n>N$:
$$
\left|{a_{n+1}\over a_n}-0\right|<\varepsilon
$$
Then by triangle inequality we have 
$$
a_{n+1}<\varepsilon a_n<{\varepsilon}^{n-N+1} a_N
$$
Take n th root we have
$$
{\root n\of {a_n}}<{\varepsilon}^{n-N+1\over n} {\root n\of {a_N}}
$$
Let $n\to\infty$ on both sides we have
$$
\lim_{n\to\infty}{\root n\of {a_n}}<\varepsilon
$$
for arbitrary $\varepsilon>0$. Also, we have $\displaystyle\lim_{n\to\infty}{\root n\of {a_n}}\ge0$, we conclude $\displaystyle\lim_{n\to\infty}{\root n\of {a_n}}=0$
\smallskip
When $p=+\infty$, $\forall M>0$, $\forall N>0$, $\exists n>N$ such that:
$${a_{n+1}\over a_n}>M
$$
Then we conclude that
$$
a_n\ge M^{n-N}a_N
$$
Take nth root we have
$$
{\root n\of {a_n}}\ge M^{n-N\over n}{\root n\of {a_N}}
$$
Since $M^{n-N\over n}\to M$, ${\root n\of {a_N}}\to 1$ as $n\to \infty$, we can find sufficiently large $n$ such that $M^{n-N\over n}>M/2$, ${\root n\of {a_N}}> 1/2$. This shows that given arbitrary $M'>0$, we just pick a large enough $M>4M'$, then
$$
{\root n\of {a_n}}\ge M^{n-N\over n}{\root n\of {a_N}}>M/4>M'
$$
This shows that  ${\root n\of {a_n}}$ diverges to infinity. 
\bigskip
%problem 3
\pnum3
First for ratio test:
$$
{2+(-1)^{n+1}\over 2^n}\cdot{2^{n-1}\over 2+(-1)^n}={1\over2}{2+(-1)^{n+1}\over2+(-1)^n}
$$
then the $\varlimsup=3/2>1$, $\varliminf=1/6<1$. The test tells us nothing.

Now for root test, we have:
$$
{\root n\of {2+(-1)^n\over 2^{n-1}}}={{\root n\of{ 4+2(-1)^n} }\over2}\to {1\over2} <1
$$
as $n\to\infty$, so the series converges.
\bigskip
%problem 4
\pnum4
\noindent i) First partial sum 
$$
\sum_{k=0}^n(-1)^m\le1
$$
is bounded. Then sequence ${1\over\sqrt{n+1}}$ decrease to 0, so by Leibniz Theorem the series converges.

Also 
$$
\sum_{k=0}^{\infty}\left|{(-1)^k\over\sqrt{k+1}}\right|=
\sum_{k=0}^{\infty}{1\over\sqrt{k+1}}>\sum_{k=0}^{\infty}{1\over k+1}\to\infty
$$
So it doesn't absolute converge, thus it is conditionally converge.
\smallskip
\noindent ii) Now the Cauchy product:
$$
A^2=\sum_{k=0}^{\infty}\sum_{j=0}^k{(-1)^j\over\sqrt{j+1}}{(-1)^{k-j}\over\sqrt{k-j+1}}
$$
But $$
\eqalign{
|{(-1)^j\over\sqrt{j+1}}{(-1)^{k-j}\over\sqrt{k-j+1}}|
&=\sum_{j=0}^k{1\over\sqrt{j+1}\sqrt{k-j+1}}\cr
&\quad\ge
\sum_{j=0}^k{2\over j+1 + k-j+1}
={2(k+1)\over k+2}\to 2\not=0
}$$ so it doesn't converge.
\bigskip
%problem 5
\pnum5
\noindent i) Since ${1\over \sqrt n}>{1\over n}$ we have
$$
{1\over\sqrt n}+{(-1)^n\over n}\ge{1\over\sqrt n}-{1\over n}>0
$$
So the summand is alternating its sign.
\noindent ii) Now, if the series converges, it implies
$$
\sum_{n=2}^{\infty}(-1)^n\left({1\over\sqrt n}+{(-1)^n\over n}\right)=\sum_{n=2}^{\infty}{(-1)^n\over\sqrt n}+\sum_{n=2}^{\infty}{1\over n}
$$
converges. But by Leibniz Theorem $ \sum_{n=2}^{\infty}{(-1)^n\over\sqrt n}$ converges, this implies $\sum_{n=2}^{\infty}{1\over n}$ converges, which is impossible, so the original series diverges.

This does not contradict Leibniz criterion since the sequence ${1\over\sqrt n}+{(-1)^n\over n}$ is not strictly decreasing. Take, for example, odd number $n$, we have:
$$\eqalign{
&{1\over\sqrt{n+1}}+{1\over n+1}\ge{1\over\sqrt n}-{1\over n}\cr
&\Longleftarrow{n+1+\sqrt{n+1}\over(n+1)\sqrt{n+1}}\ge{n+\sqrt{n}\over(n)\sqrt{n}}\cr
&\Longleftarrow{1+\sqrt{n+1}\over n+1}\ge{\sqrt{n}-1\over n}\cr
&\Longleftarrow2n+1\ge{\sqrt n\sqrt{n+1}\over \sqrt{n+1}-\sqrt{n}}\cr
}
$$
And we always have
$$
{\sqrt n\sqrt{n+1}\over \sqrt{n+1}-\sqrt{n}}\le{{n+n+1\over 2}\over \sqrt{n+1}-\sqrt{n}}<2n+1
$$
Thus for odd $n$, we have
$$
{1\over\sqrt{n+1}}+{1\over n+1}\ge{1\over\sqrt n}-{1\over n}
$$
which is not decreasing, and Leibniz criterion can't be applied.
\bigskip
\pnum6
\noindent i)
$$
{1\over\rho}=\root n \of {n^3\over n^2-5}\to1\Longrightarrow\rho=1
$$
and when $\rho=1$ or $\rho=-1$ the terms ${n^3\over n^2-5}\to\infty$ so they both {\bf diverge}.
\medskip
\noindent ii)
$$
{1\over\rho}={(n+1)^2\over 4^{(n+1)}+3(n+1)}\cdot{4^{n}+3n\over n^2}\to{1\over4}\Longrightarrow \rho=4
$$
and when $\rho=4$ or $\rho=-4$ the terms' absolute value $\displaystyle{n^2\over1+{3n\over4^n}}$ doesn't converge to zero, so they both {\bf diverge}.\medskip
\noindent iii)
$$
{1\over\rho}=\root n\of {(\ln n)^2}=\exp\{{{2\over n}\ln\ln n}\}\to1\Longrightarrow \rho=1
$$
and when $\rho=1$ or $\rho=-1$   terms' absolute value $(\ln n)^2$ doesn't converge to zero so the series {\bf diverges}.
\medskip
\noindent iv) Regard $z^{2n}$ as $(z^2)^n$
$$
{1\over\rho'}=\root n\of {n\over 2^{n+1}}={1\over2}\Longrightarrow \rho'=2
$$
And since $\rho=\sqrt{\rho'}$, we have $\rho=\sqrt 2$. When $\rho=\sqrt 2$ or $\rho=-\sqrt 2$, the series becomes $\sum (-1)^n {n\over 2}$ and {\bf diverges}.
\bigskip
\pnum7
\noindent i) For sufficiently large n, we have:
$$
{1\over \rho}=\left|{\alpha\choose n+1}\Big/{\alpha\choose n}\right|={n-\alpha\over n+1}\to 1\Longrightarrow \rho=1
$$
Now, consider 
$$\eqalign{
{d\over dx}[(1+x)^{-\alpha}&B_{\alpha}(x)]
=\sum_{n\ge0}{\alpha \choose n}\left(nx^{n-1}(1+x)^{-\alpha}+(-\alpha)(1+x)^{-\alpha-1}x^n\right)\cr
&=\sum_{n\ge0}{\alpha \choose n}\left(nx^{n-1}(1+x)^{-\alpha-1}(1+x)+(-\alpha)(1+x)^{-\alpha-1}x^n\right)\cr
&=(1+x)^{-\alpha-1}\left(\sum_{n\ge0}{\alpha \choose n}\left(nx^{n-1}+nx^{n}+(-\alpha)x^n\right)\right)\cr
&=(1+x)^{-\alpha-1}\left(\sum_{n\ge0}{\alpha \choose n}nx^{n-1}+
\sum_{n\ge0}{\alpha \choose n}nx^{n}+
\sum_{n\ge0}{\alpha \choose n}(-\alpha)x^n\right)\cr
&=(1+x)^{-\alpha-1}\alpha\left(\sum_{n\ge0}{\alpha-1 \choose n-1}x^{n-1}+
\sum_{n\ge0}{\alpha-1 \choose n-1}x^{n}+
\sum_{n\ge0}{\alpha \choose n}(-1)x^n\right)\cr
&=(1+x)^{-\alpha-1}\alpha\left(\sum_{n\ge0}{\alpha-1 \choose n}x^{n}+
\sum_{n\ge0}{\alpha-1 \choose n-1}x^{n}-
\sum_{n\ge0}{\alpha \choose n}x^n\right)\cr
&=(1+x)^{-\alpha-1}\alpha\left(\sum_{n\ge0}{\alpha-1 \choose n}x^{n}+
{\alpha-1 \choose n-1}x^{n}-
\sum_{n\ge0}{\alpha \choose n}x^n\right)\cr
&=(1+x)^{-\alpha-1}\alpha\left(\sum_{n\ge0}{\alpha \choose n}x^{n}-
\sum_{n\ge0}{\alpha \choose n}x^n\right)\cr
&=0
}
$$
So the product is constant, which means $(1+x)^{-\alpha}B_{\alpha}(x)=(1+0)^{-\alpha}B_{\alpha}(0)=1$ for all $|x|<1$, namely
$$
(1+x)^{\alpha}=\sum_{n=0}^{\infty}{\alpha \choose n}x^n
$$
\medskip
\noindent ii) Now  for $\alpha\in {\rm I\!N}$, for all integer $j>\alpha$, the product 
$$
\alpha(\alpha-1)\cdots(\alpha-(j-1))
$$
will turn to zero because for some $k$ in the product such that $k=\alpha\le j-1$, $\alpha-k =0$. Then the sum is finite since ${\alpha \choose j}=0$ when $j>\alpha$
\medskip
\noindent iii) Let $\alpha=1/2$ in $B_{\alpha}(x)$ we have
$$\eqalign{
\sqrt{x+1}&=B_{1/2}(x)=\sum_{n=0}^{\infty}{1/2 \choose n}x^n\cr
&=\sum_{n=0}^{\infty}{{1\over 2}({1\over 2}-1)\cdots({1\over 2}-(n-1))\over 2n!}x^n\cr
&=\sum_{n=0}^{\infty}\left(-{1\over 2}\right)^{n-1}
{(1\cdot 3\cdot 5\cdots (2n-3))\over 2n!}x^n\cr
&=\sum_{n=1}^{\infty}\left(-{1\over 2}\right)^{n-1}
{(1\cdot 2\cdot 3\cdot 4 \cdots (2n-3)(2n-2))\over 2n!2^{n-1}(n-1)!}x^n+1\cr
&=\sum_{n=1}^{\infty}\left(-{1\over 4}\right)^{n}(-2){(1\cdot 2\cdot 3\cdot 4 \cdots (2n-3)(2n-2))\over (n-1)!(n-1)!}{x^n\over n}+1\cr
&=1-2\sum_{n=1}^{\infty}{(-1)^n\over 4^n}{2n-2\choose n-1}{x^n\over n}
}
$$
\medskip
\noindent iv) When $x=-1$,  note that
$$
{\alpha \choose n}(-1)^n={(n-1-\alpha)(n-2-\alpha)\cdots (2-\alpha)\cdot (1-\alpha)\cdot(-\alpha)\over n!}>{1\over n}
$$
for $\alpha<0$,
So the sum diverges by comparison test.

As for $\alpha=0$, all summand (except the first one) becomes 0, clearly it converges to 1;

Now, for $k\ge\alpha>k-1\ge0$
$$\eqalign{
{\alpha\choose n}=&{\alpha(\alpha-1)\cdots(\alpha-(k-1))(\alpha-k)\cdots(\alpha-(n-1))\over n!}\cr
&\quad=(-1)^{n-k}{\alpha(\alpha-1)\cdots(\alpha-(k-1))(k-\alpha)\cdots((n-1)-\alpha)\over n!}=(-1)^{n-k}b_n
}
$$
So $b_n>0$, also, the sum deduced to 
$$
\sum_{n=0}^{\infty}{\alpha \choose n}(-1)^n=\sum_{n=0}^{\infty}=b_n(-1)^{2n-k}=(-1)^{-k}\sum_{n=0}^{\infty}b_n
$$
Now by ratio test:
$$
b_{n+1}/b_n={n-\alpha\over n+1}\le1
$$ for sufficiently large $n$, so the series converges.
\medskip
\noindent v) Note that 
$$
{\alpha \choose n}={n-\alpha-1\choose n}(-1)^n
$$
Since $n-\alpha-1> n$ when $\alpha\le -1$ implies ${n-\alpha-1\choose n}>0$, so we have an alternating sum and by Leibiniz test we find that
$$
{n+1-\alpha-1\choose n+1}\Big/{n-\alpha-1\choose n}=1-{\alpha+1\over n+1}\ge 1\Longrightarrow b_{n+1}\ge  b_n
$$
So by comparison test the series diverges.
\medskip
\noindent vi) Let $b_n:={\alpha \choose n}$. First we notice that
$$
b_{n+1}/b_n={\alpha-n\over n+1}<0
$$
for $n>N_0>\alpha$, so the series is alternating. Also noticing that:
$$
|b_{n+1}|/|b_n|={n-\alpha\over n+1}<1
$$
So $b_n$ is decreasing for $n>N_0$.

Now consider $f(x)=1-x-e^{-x}$, $f'(x)=-1+e^{-x}<0$ for $x>0$, so we have $f(x)<f(0)=0$, that is,  $1-x<e^{-x}$ for $x>0$. In previous proof we already have $|b_{n+1}|/|b_n|=(\alpha-n)/ (n+1)=1-{(\alpha+1)/(n+1)}$, so 
$$
0\le\left|{b_n\over b_{N_0}}\right|=\left|{b_n\over b_{n-1}}\right|\left|{b_{n-1}\over b_{n-2}}\right|\cdots\left|{b_{N_0+1}\over b_{N_0}}\right|\le
\exp\{-{\alpha+1\over n}-{\alpha+1\over n-1}+\cdots-{\alpha+1\over N_0+1}\}\to 0
$$
Since the right hand side is negative harmonic series except finite many terms and it must diverges to $-\infty$.
So by Leibiniz test the series converges.
%end
\end