\documentclass[12pt]{article}
\usepackage{geometry}
\usepackage{amsmath} 
\usepackage{amsthm}
\usepackage{amsfonts}
\usepackage{cases}
\usepackage{graphicx}
\usepackage{caption}
\usepackage{subcaption}


\linespread{1.4}
\geometry{a4paper,centering,scale=0.8}
\rmfamily 
\normalsize
\setlength{\parindent}{0em}

\begin{document} 
\begin{flushleft}
  Junlong Gao 5133709126\\ 
  Prof.  Hohberger\\ 
  Vv186 HW 5\\
  \today 
\end{flushleft}
{\bf Exercise 1}\\
{\it Solution:}
\begin{align*}
&(1)\qquad f'(x)=\frac{1}{2\sqrt x} -\frac{3}{2}\sqrt x,\\ 
&(2)\qquad f'(x)=1+\frac{1}{\sqrt{x^3}},\\
&(3)\qquad f'(x)=\sqrt{3}-\frac{\sqrt{2}}{2\sqrt{x}},\\
&(4)\qquad f'(x)=\frac{2}{3{\root 3 \of x}}-\frac{3}{2}\sqrt{x},\\
&(5)\qquad f'(x)=2x{\root 3\of {x^2+2}}+\frac{2x(x^2+1)}{3{\root 3\of {(x^2+2)^2}}},\\
&(6)\qquad f'(x)=\frac{1}{2\sqrt{x+\sqrt{x+\sqrt{x}}}}\left(1+\frac{1}{2\sqrt{x+\sqrt{x}}}(1+\frac{1}{2\sqrt{x}})\right),\\
&(7)\qquad f'(x)=\frac{2{(x-1)}^3(-3x^2+4x+5)}{{(x^2+2x)}^{6}},\\
&(8)\qquad f'(x)=-\frac{x}{(1+x^2)\sqrt{1+x^2}}\quad.
\end{align*}\\

{\bf Exercise 2}\\
{\it Proof:}\\
i) Given $f(x)$ is even, then
\begin{align*}
\frac{f(x+\Delta x)-f(x)}{\Delta x}&=-\frac{f(-x+(-\Delta x))-f(-x)}{-\Delta x}
\intertext{Let $\Delta x\to 0$ on both sides gives us:}
f'(x)&=-f'(-x)
\end{align*}
That is, $f'(x)$ is odd function.
ii) Given $f(x)$ is odd, then
\begin{align*}
\frac{f(x+\Delta x)-f(x)}{\Delta x}&=\frac{f(-x+(-\Delta x))-f(-x)}{-\Delta x}
\intertext{Let $\Delta x\to 0$ on both sides gives us:}
f'(x)&=f'(-x)
\end{align*}
That is, $f'(x)$ is even function.\qed\\

{\bf Exercise 3}\\
{\it Proof:}
Let's start by $n=1$:, then we have
\[
(fg)^{(1)}=(fg)'=f'g+fg'=\sum_{k=0}^{1}{1 \choose k}f^{(k)}(x)g^{(n-k)}(x)
\]
true. \\
Now suppose the {\it Leibniz rule} holds for $n$, then for $n+1$, we have\\
\begin{align*}
(fg)^{(n+1)}&=\frac{d}{dx}\sum_{k=0}^{n}{n \choose k}f^{(k)}(x)g^{(n-k)}(x)\\
&=\sum_{k=0}^{n}{n \choose k}\frac{d}{dx}\left(f^{(k)}(x)g^{(n-k)}\right)\\
&=\sum_{k=0}^{n}{n \choose k}\left(f^{(k+1)}(x)g^{(n-k)}+f^{(k)}(x)g^{(n-k+1)}\right)\\
&=\sum_{k=1}^{n+1}{n \choose k-1}\left(f^{(k+1)}(x)g^{(n+1-k)}\right)+
\sum_{k=0}^{n}{n \choose k}\left(f^{(k)}(x)g^{(n+1-k)}\right)\\
&=\sum_{k=0}^{n+1}\left({n \choose k-1}+{n \choose k}\right)\left(f^{(k)}(x)g^{(n+1-k)}\right)\\
&=\sum_{k=0}^{n+1}{n+1 \choose k}f^{(k)}(x)g^{(n+1-k)}
\end{align*}\qed\\
{\bf Exercise 4}\\
{\it Solution:}
\begin{align*}
\frac{d^{100}}{dx^{100}}{(x^2+3x+2)}^{-1}&=\frac{d^{100}}{dx^{100}}\left(\frac{1}{x+1}-\frac{1}{x+2}\right)\\
&=\frac{d^{100}}{dx^{100}}\left(\frac{1}{x+1}\right)-\frac{d^{100}}{dx^{100}}\left(\frac{1}{x+2}\right)\\
&=100!\left(\frac{1}{{(x+1)}^{101}}-\frac{1}{{(x+2)}^{101}}\right)
\end{align*}
For the second, we note that
\[
\frac{x^2+1}{x^3-x}=\frac{1}{x-1}+\frac{1}{x+1}-\frac{1}{x}
\]

So have
\begin{align*}
{\left(\frac{x^2+1}{x^3-x}\right)}^{(100)}&={\left(\frac{1}{x-1}+\frac{1}{x+1}-\frac{1}{x}\right)}^{(100)}\\
&=100!\left(\frac{1}{(x-1)^{101}}+\frac{1}{(x+1)^{101}}-\frac{1}{x^{101}}\right)
\end{align*}\\

{\bf Exercise 5}\\
{\it Proof:}\\
i) Given $f(x)=O(x^{\alpha})$ with $\alpha>1$, by definition first there exists $C,\delta_{\epsilon}>0$ such that
\[
|f(x)|<C|x^{\alpha}|
\]
For all $0\leq x<\delta_{\epsilon}$. Now if given $\epsilon>0$, just pick $\delta<\min(\root \alpha \of {\epsilon/C},\delta_{\epsilon})$, since $\alpha>0$. Now we have $|f(x)|<\epsilon$, this implies
\[
\lim_{x\to0}f(x)=0
\]
Then the continuity gives us $f(0)=0$.\\
Finally, back to $f(x)=O(x^{\alpha})$,  by definition there exists $C,\delta_{\epsilon}>0$ such that
\[
\left|\frac{f(0+\Delta x)-f(0)}{\Delta x}\right|=\left|\frac{f(\Delta x)}{\Delta x}\right|<C\left|\frac{{(\Delta x)}^{\alpha}}{\Delta x}\right|=C|{(\Delta x)}^{\alpha-1}|
\]
For all $0\leq x<\delta_{\epsilon}$. Now if given $\epsilon>0$, we can choose $\delta<\min(\root \alpha-1 \of {\epsilon/C},\delta_{\epsilon})$, since $\alpha>0$. Now we have $|f(x)|<\epsilon$, this implies:
\[
f'(0)=0
\]
That is, $f(x)$ is differentiable at $x=0$.\\
ii) To show it's not differentiable, we consider the rate of change near $x=0$:
\[
\left|\frac{f(0+\Delta x)-f(0)}{\Delta x}\right|=\left|\frac{f(\Delta x)}{\Delta x}\right|\ge{|\Delta x|}^{\beta-1}=\frac{1}{|\Delta x|^{1-\beta}}
\]
Since $\beta\le1$, if given $\epsilon=1$, for all $\delta>0$, we can find $x<\min(1,\delta)$, such that
\[
\frac{1}{|\Delta x|^{1-\beta}}\ge1=\epsilon
\]
This shows that $\displaystyle\lim_{x\to0}\frac{f(0+\Delta x)-f(0)}{\Delta x}$ does not exists, namely not differentiable at $x=0$.

\end{document}