%lead
\input my_macros
\mydoc
\baselineskip 16pt plus 2pt
\vv9 November {23}
\def\l{\lambda_i}
%
%problem 1
\pnum1
\noindent i) Since 
$$
\sum_{i=1}^{n}\l x_i\le\sum_{i=1}^{n}\l \max_{1\le i\le n}x_i=\max_{1\le i\le n}x_i\sum_{i=1}^{n}\l=\max_{1\le i\le n}x_i
$$
and
$$
\sum_{i=1}^{n}\l x_i\ge\sum_{i=1}^{n}\l \min_{1\le i\le n}x_i=\min_{1\le i\le n}x_i\sum_{i=1}^{n}\l=\min_{1\le i\le n}x_i
$$
we obtain the desired inequality.\medskip
\noindent ii) Since
$$
{1\over t}\sum_{i=1}^{n-1}\l x_i\le{1\over t}\sum_{i=1}^{n-1}\l \max_{1\le i\le n}x_i=\max_{1\le i\le n}x_i{1\over t}\sum_{i=1}^{n-1}\l=\max_{1\le i\le n}x_i
$$
and
$$
{1\over t}\sum_{i=1}^{n-1}\l x_i\ge{1\over t}\sum_{i=1}^{n-1}\l \min_{1\le i\le n}x_i=\min_{1\le i\le n}x_i{1\over t}\sum_{i=1}^{n-1}\l=\max_{1\le i\le n}x_i
$$
we obtain the desired inequality.\medskip
\noindent iii) For $n=1$, obviously the inequality holds. Now, suppose it holds for $n-1$, then set
$$
t=\sum_{i=1}^{n-1}\l x_i\Longrightarrow \qquad t+\lambda_n=1,\quad\sum_{i=1}^{n-1}{\l\over t} x_i=1
$$
By ii) we know that $t\sum_{i=1}^{n-1}{\l\over t} x_i\le\max_{1\le i\le n}x_i\in I$.
thus we have
$$
\eqalign{
f\left(\sum_{i=1}^{n}\l x_i\right)&=
f\left(t\sum_{i=1}^{n-1}{\l\over t} x_i+\lambda_nx_n\right)\cr
&\le tf\left(\sum_{i=1}^{n-1}{\l\over t} x_i\right)+\lambda_nf(x_n)\qquad{\rm (Convex)}\cr
&\le t\sum_{i=1}^{n-1}{\l\over t}f(x_i)+\lambda_nf(x_n)\qquad {\rm (Induction)}\cr
&=\sum_{i=1}^{n}\l f(x_i)
}
$$
For concave, we just need to reverse the inequality:
$$
f\left(\sum_{i=1}^{n}\l x_i\right)\ge\sum_{i=1}^{n}\l f(x_i)
$$
\noindent iv) $(\ln x)''=-1/x^2<0$ so logarithm is a concave function, by Jensen's inequality we have 
$$
\ln\left(\prod_{i=1}^{n}{x_i}^{\l}\right)=\sum_{i=1}^{n}\l \ln(x_i)\le \ln\left(\sum_{i=1}^{n}\l x_i\right)
$$
Taking the left-hand side and right-hand side we have:
$$
\prod_{i=1}^{n}{x_i}^{\l}\le\sum_{i=1}^{n}\l x_i
$$
Since logarithm is a monotonic increasing function.
\bigskip
%problem 2
\snum2
\noindent i) Consider function
$$
f(x)=\cos(x^3)
$$
Then $\sup |f(x)|=0$, $\sup|f(x)|=\sup\,|3x^2\sin(x^3)|=+\infty$
\medskip
\noindent ii) Consider function
$$
f(x)=\cases{\ln x, &if $x\ge1$;\cr
		{1\over2}x^2-{1\over2},  &if $|x|<1$.\cr
		\ln (-x) &if $x\le-1$}
		$$
Since it's piecewise differentiable, we have
$$
f'(x)=\cases{{1\over x}, &if $x\ge1$;\cr
		x ,  &if $|x|<1$ ;\cr
		{1\over x},  &if $x\le-1$.}
$$
And we can see that at $x=1$, $x=-1$ the left and right derivative are equal. Also, we see that $\lim_{x\to\infty}|f'|=0$ while $\lim_{x\to\infty}|f|=\infty$.
\bigskip
%problem 3
\pnum3
\noindent i)
By l'Hopital's rule we have:
$$
\lim_{x\to\infty}{e^x-e^{-x}\over e^x+e^{-x}}=\lim_{x\to\infty}{e^x+e^{-x}\over e^x-e^{-x}}
$$
And it's not very effective. We can simply divide the whole thing by $e^x$:
$$
\lim_{x\to\infty}{e^x-e^{-x}\over e^x+e^{-x}}=\lim_{x\to\infty}{1-e^{-2x}\over 1+e^{-2x}}=1
$$
\noindent ii)
By l'Hopital's rule we have:
$$
\lim_{x\to 0}{x^2\cos{1\over x}\over \sin x}=\lim_{x\to 0}{2x\cos{1\over x}+\sin {1\over x}\over \cos x}={\lim_{x\to 0}2x\cos{1\over x}+\sin {1\over x}\over \lim_{x\to 0}\cos x}
$$
And limit can't be evaluated.
Another approach:
$$
\lim_{x\to 0}{x^2\cos{1\over x}\over \sin x}=\lim_{x\to 0}x\cos{1\over x}{x\over \sin x}=\lim_{x\to 0}x\cos{1\over x}\lim_{x\to 0}{x\over \sin x}=0\cdot 1=0
$$
\noindent iii)
$$
{f'(x)\over g'(x)}={2\cos^2 x\over e^x(2\cos^2 x+x+\sin x\cos x)}={1\over e^x(1+x/2\!\cos^2x+\sin x/2\!\cos x)}\to 0
$$
Since we have $e^x\cdot x/2\!\cos^2x\to \infty$.
But clearly 
$$
\lim_{x\to\infty}{f(x)\over g(x)}=\lim_{x\to\infty}e^{\sin x}
$$
doesn't exist.
And it {\bf doesn't} contradict l'Hopital's rule since we first have to convert it into $0\over0$ type:
$$
{\displaystyle{1\over g(x)}}\over {\displaystyle{1\over f(x)}}
$$
and the corresponding limit is
$$
{{\displaystyle\left({1\over g(x)}\right)'}\over {\displaystyle\left({1\over f(x)}\right)'}}
={g'(x)\over f'(x)}{f^2(x)\over g^2(x)}
$$
and it doesn't satisfy l'Hopital's rule's condition because $f'(x)=2\cos^2x$ have infinitely many zero throughout ${\rm I\!R}^+$ and it's impossible to find a $C$ such that for all $x>C$, $f'(x)\not=0$. Thus we can't use the rule for the original limit.
\bigskip
\snum4
\noindent i) All seven roots:
$$
z_k={\root 7\of5}e^{i(\theta+{2k\pi\over 7})},\quad k=0,1,2\cdots,6\qquad {\rm where}\, \sin\theta={3\over5}, \,\cos\theta={4\over5}
$$
\noindent ii) Let $z=a+bi$ where $a,b\in {\rm I\!R}$, plug in the equation:
$$
a^2-b^2+2abi-ai+b+1=0
$$
This implies
$$a^2-b^2+b+1=0;\qquad 2ab-a=0.
$$
Solve for:
$$
z_1={1+\sqrt5\over2}i,\qquad z_2={1-\sqrt5\over2}i.
$$
\noindent iii) Solve for $z^2$ we have:
$$
z^2={-1+\sqrt3i\over2}=e^{i{2\over3}\pi}\qquad or\qquad z^2={-1-\sqrt3i\over2}=e^{i{4\over3}\pi}
$$
Now we solve for z on the unit circle:
$$\eqalign{
x_1={1\over2}+{\sqrt3\over2}i;\quad x_2=-{1\over2}-{\sqrt3\over2}i;\cr
x_3=-{1\over2}+{\sqrt3\over2}i;\quad x_4=+{1\over2}-{\sqrt3\over2}i.\cr
}
$$
\noindent iv) By general method for solving linear equation we have:
$$
z= {7\over 3}-{4i\over3},\qquad w={1\over3}+2i
$$
\bigskip
%problem 5
\snum5
\noindent i) $$
s(\varphi)=l+r-\sqrt{l^2-r^2\sin^2\varphi}-r\cos\varphi
$$
\medskip
\noindent ii)
Solve for:
$$
s=l\Longrightarrow\quad {\cos\varphi=r\over2l}
$$
so 
$$
\varphi_c=\arccos{r\over2l}
$$
\medskip
\noindent iii) A bit of calculation tells us that:
$$
s(0)=0;\quad s(\pi)=2r;\quad s'(0)=0;\quad s''(0)=r+{r^2\over l}.
$$
Thus by comparing the coefficients we have:
$$
S(\varphi)=r+{r^2\over4l}-r\cos\varphi-{r^2\over4l}\cos2\varphi
$$
\medskip
\noindent iii)
We have:
$$
v(\varphi)=\omega_0S'(\varphi)=\omega_0(r\sin\varphi+{r^2\over2l}\sin2\varphi),\quad
a(\varphi)= \omega_0^2 S''(\varphi)=\omega_0^2(r\cos\varphi+{r^2\over l}\cos2\varphi)
$$
Now for $S''(x)$ it can be deduced to:
$$
g(t)=rt+{r^2\over l}(2t^2-1)={2r^2\over l}t^2+rt-{r^2\over l}
$$
should attain it maximal at $t=-l/4r<-1$ or $t=1$ or $t=-1$:
$$
|a_{\max}|=\omega_0^2\max( r+{r^2\over l}\,,\, {r^2\over l}+{l\over8})=\omega_0^2 (r+{r^2\over l})
$$
or the maximal acceleration is at $t=-1$ or $t=1$ if $l\ge4r$, we have:
$$
|a_{\max}|=\omega_0^2 (r+{r^2\over l})
$$
In conclusion, the maximal acceleration is:$$
|a_{\max}|=\omega_0^2 (r+{r^2\over l})
$$
As for the maximal speed, we solve for the root of $g(t)=0$:
$$
t_1={-l+\sqrt{l^2+8r^2}\over4r}\in(0,1),\qquad t_2={-l-\sqrt{l^2+8r^2}\over4r}<-1
$$
So the maximal speed attends at $t=t_1$:
$$
|v_{\max}|=\omega_0r\sqrt{1-t_1^2}(1+{r\over l}t_1)
$$
\noindent iv) Plug in the number we have:
$$
s(x)=6-\sqrt{25-\sin^2\varphi}-\cos\varphi=6-\sqrt{{49\over2}-{\cos2\varphi\over2}}+\cos\varphi,\quad S(x)={21\over20}-\cos\varphi-{1\over20}\cos2\varphi
$$
And the difference:
$$
|s(\varphi)-S(\varphi)|=\left|{99\over20}+{1\over20}\cos2\varphi-\sqrt{{49\over2}+{\cos2\varphi\over2}}\,\right|
$$
Investigate $e(x)={1\over20}x-\sqrt{{49\over2}+{x\over 2}}<0$ where $-1\le x\le1$, we have:
$$
e'(x)={1\over20}-{1\over4}\left({49\over2}+{x\over2}\right)^{-1/2}\le0
$$
So the maximum error occurs at $\cos2\varphi=-1$ thus $\max|s(\varphi)-S(\varphi)|=4.9-\sqrt{24}=0.00102$.
\end