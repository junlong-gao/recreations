%lead
\input my_macros
\mydoc
\baselineskip 16pt plus 2pt
\vv11 December {6}


%
%problem 1
\medskip
\pnum1
\noindent Consider function
$$
F(x)=\int_a^xf(t)\,dt,\qquad x\in [a,b]
$$
Then we have:
$$
F'(x)=f(x),\qquad x\in [a,b]
$$
And by Mean Value Theorem(of differential calculus) we have:
$$
{F(b)-F(a)\over b-a}=F'(\xi)=f(\xi),\qquad\exists\,\xi\in [a,b] 
$$
Since
$$
F(b)-F(a)=\int_a^bf(t)-\int_a^af(t)=\int_a^bf(t)-0=\int_a^bf(t)
$$
That is:
$$
\int_a^bf(t)\,dt=(b-a)f(\xi),\qquad\exists\,\xi\in [a,b] 
$$
\bigskip
%problem 2
\snum2
i)
$$
\int e^x\sin x\,dx={1\over 2}e^x(\sin x-\cos x)
$$
\medskip
ii)
$$
\int \tan x\,dx=-\ln|\cos x|
$$
\medskip
iii)
$$
\int {1\over x^2(1+x)^2}\,dx=-{1\over x}-{1\over1+x}-2\ln\left|{x\over1+x}\right|
$$
\medskip
iv)
$$
\int e^{x^2}x(1+x^2)\,dx={x^2\over2}e^{x^2}
$$
\medskip
v)
$$
\int \sqrt{1-x\over1+x}\,dx=\arcsin x+\sqrt{1-x^2}
$$
\bigskip
%problem 3
\snum3
\noindent i)
$$
\int_0^1\ln x\,dx=x\ln x-x\Big|_0^1=-1
$$
\medskip
\noindent ii)
$$
\int {\ln\ln x\over x}\,dx=\ln x\ln\ln x-\ln x
$$
\medskip
\noindent iii)
$$
\int \tan^2 x\,dx=\tan x-x
$$
\medskip
\noindent iv)
$$
\int_1^{e^{\pi/2}}\sin\ln x\,dx={x\over 2}(\sin\ln x-\cos\ln x)\Big|_1^{e^{\pi/2}}
={1\over2}(e^{\pi/2}+1)
$$
\medskip
\noindent v) First we show the convergence:
$$
\lim_{x\to0}{\ln\sin x\over{1\over \sqrt x}}=\lim_{x\to0}{{\cos x\over\sin x}\over-{1\over2}x^{-3/2}}=\lim_{x\to0}(-2)\sqrt x{x\over\sin x}\cos x=0
$$
And the improper integral
$$\int_0^{\pi/2}{1\over\sqrt x}\,dx=2\sqrt{\pi\over 2}
$$ 
Converges, so by comparison test the improper integral $\int_0^1\ln\sin x\,dx$ converges.
Furthermore, we notice that
$$\eqalign{
\int_0^{\pi/2}\ln\sin x\,dx&=\int_0^{\pi/2}\ln2\sin {x\over2}\cos{x\over2}\,dx\cr
&={\pi\over2}\ln2+2\int_0^{\pi/4}\sin x\,dx-2\int_{\pi/2}^{\pi/4}\cos x\,dx\cr
&={\pi\over2}\ln2+2\int_0^{\pi/2}\sin x\,dx
}
$$
Altogether we have:
$$
\int_0^{\pi/2}\ln\sin x\,dx=-{\pi\over2}\ln2
$$
\bigskip
%problem 4
\snum4
\noindent a) If $a=0$ and $b=0$, we have:
$$
\int {dx\over ax^2+bx+c}={x\over c}
$$
\noindent b) If $a=0$ and $b\not=0$, we have:
$$
\int {dx\over ax^2+bx+c}=\int {dx\over bx+c}={1\over b}\ln|bx+c|
$$
c) If $a\not=0$ and $\Delta>0$; we have:
$$
\int {dx\over ax^2+bx+c}={1\over a}\int {d(x+{b\over 2a})\over (x+{b\over 2a})^2+{\Delta\over 4a^2}}
={2\over \sqrt{\Delta}}\arctan\left({2|a|\over \sqrt{\Delta}}(x+{b\over 2a})\right)
$$
d) If $a\not=0$ and $\Delta=0$; we have:
$$
\int {dx\over ax^2+bx+c}={1\over a}\int {d(x+{b\over 2a})\over (x+{b\over 2a})^2}=
-{2\over 2ax+b}
$$
f) If $a\not=0$ and $\Delta<0$; we have:
$$
\int {dx\over ax^2+bx+c}={1\over a}\int {dx\over (x-x_1)(x-x_2)}={1\over \sqrt{-\Delta}}\ln{\left|x-x_1\over x-x_2\right|}
$$
where
$$
x_1={-b+\sqrt{-\Delta}\over2a},\qquad x_2={-b-\sqrt{-\Delta}\over2a}
$$
\bigskip
%problem 5
\snum5
\noindent i)
$$
\sin x={2\sin{x\over 2}\cos{x\over2}\over\sin^2{x\over2}+\cos^2{x\over2}}={2t\over1+t^2}
$$
$$
\cos x={\sin^2{x\over 2}-\cos^2{x\over 2}\over\sin^2{x\over2}+\cos^2{x\over2}}={1-t^2\over1+t^2}
$$
Finally
$$
x=2\arctan t\Longrightarrow dx={2\over1+t^2}dt
$$
\medskip
\noindent ii)
a)
$$
\int {1\over\sin x}\,dx=\int {1+t^2\over 2t}{2\over 1+t^2}\,dt=\ln t=\ln \left|\tan{x\over2}\right|
$$
b)
$$
\int {1\over\cos x}\,dx=\int {1+t^2\over1-t^2}{2\over 1+t^2}\,dt=
\ln\left |{1+t\over1-t}\right|=
\ln\left|{1+ \tan{x\over2}\over 1- \tan{x\over2}}\right|
=\ln\left|{1\over\cos x}+\tan x\right|
$$
c)
$$
\int {dx\over\sin x+\cos x}={1\over\sqrt2}\int {d(x+{\pi\over4})\over \sin(x+{\pi\over4})}
={1\over\sqrt2}\ln \left|\tan({x\over2}+{\pi\over8})\right|
$$


\bigskip
%problem 6
\snum6
\noindent i) Take $x={\tan t\over a}$, we have:
$$\eqalign{
J(a)&=\int_0^{\arctan a}{1\over \cos^3 t}dt
=\int_0^{\arctan a}{1\over (1-\sin^2 t)^2}\,d(\sin t)\cr
&={1\over 2}{\sin t\over \cos^2t}+{1\over 2}\ln\left|{1+\sin t\over \cos t}\right|\Big|^{\arctan a}_0\cr
&={a\sqrt{1+a^2}\over 2}+{1\over 2}\ln{|a+\sqrt{1+a^2}|}\cr
}
$$
\medskip
\noindent ii) Take $x=a\sin t$, we have:
$$
I(a)=\int {1\over \cos t}\cos t\,dt=t=\arcsin {x\over a}
$$

\bigskip
%problem 7
\snum7
\noindent i) Since 
$$
\lim_{x\to \infty}{x^{3/2}\over\sqrt{1+x^3}}={1\over\sqrt{1+x^{-3}}}=1
$$
And $\int_1^{\infty}x^{-3/2}\,dx$ converges\eject
\noindent Thus the improper integral:
$$
\int_0^{\infty}{dx\over\sqrt{1+x^3}}=\int_0^1{dx\over\sqrt{1+x^3}}+\int_1^{\infty}{dx\over\sqrt{1+x^3}}
$${\bf coverges}.
\medskip
\noindent ii) Since
$$
\lim_{x\to \infty}x^{1/2}{x\over\sqrt{1+x^{3/2}}}={1\over1+x^{-3/2}}=1
$$
And $\int_1^{\infty}x^{-1/2}\,dx$ diverges, the improper integral:
$$
\int_0^{\infty}{x\over\sqrt{1+x^{3/2}}}\,dx=\int_0^1{x\over\sqrt{1+x^{3/2}}}\,dx
+\int_1^{\infty}{x\over\sqrt{1+x^{3/2}}}\,dx
$${\bf diverges}.
\medskip
\noindent iii) Since
$$
\lim_{x\to \infty} x^{3/2}{1\over\ x \sqrt{1+x }}={1\over\sqrt {1+x^{-1}}}=1
$$
And $\int_1^{\infty}x^{-3/2}\,dx$ converges, $\int_0^{1}x^{-3/2}\,dx$ diverges, so the improper integral:
$$
\int_0^{\infty}{1\over\ x \sqrt{1+x }}\,dx=\int_0^1{1\over\ x \sqrt{1+x }}\,dx
+\int_1^{\infty}{1\over\ x \sqrt{1+x }}\,dx
$$
{\bf diverges}.
\medskip
\noindent ii) Consider function
$$
f(x)=x^2\cos x^3-{1\over x^2}\sin x^3,\qquad x>0
$$
for $x=0$ we define $f(x)=\lim_{x\to 0^+}f(x)=0$.\smallskip
Then $\sup f(x)$ doesn't exists but
$$
\int_0^{\infty} f(x)\,dx={\sin x^3\over x}\Big|_0^\infty=0-0=0.
$$
so $\int_0^{\infty}f(x)$ converges.


\bigskip
%problem 8
\pnum8
\noindent i) We see that $a_n>0$ since $\sin x>0$ for $x\in [0,\pi/2]$ 
\eject
\noindent Also:
$$
0<a_n=\int_0^{\pi/2}\sin^nx<\int_0^{\pi/2}1\cdot\sin^{n-1}x=a_{n-1}
$$
So the limit of $a_n$ exists.\medskip
\noindent ii) Integrating by parts yields:
$$\eqalign{
\int_0^{\pi/2}\sin^nx\,dx&=-\int_0^{\pi/2}\sin^{n-1}x\,d(\cos x)\cr
&=-\sin^{n-1}x\cos x\Big|^{\pi/2}_0+(n-1)\int_0^{\pi/2}\sin^{n-2}\cos^2 x\,dx\cr
&=0+(n-1)\int_0^{\pi/2}\sin^{n-2}\,dx-(n-1)\int_0^{\pi/2}\sin^{n}\,dx
}
$$
Thus we have:
$$
\int_0^{\pi/2}\sin^n x\,dx={n-1\over n}\int_0^{\pi/2}\sin^{n-2}\,dx;
$$
Also we have:
$$
\int_0^{\pi/2}\sin^2 x\,dx={\pi\over 4},\qquad \int_0^{\pi/2}\sin x\,dx=1
$$
\medskip
\noindent iii) Now, by ii) we have $a_2={\pi\over 4}$, $a_1=1$, for larger $n$ we observe that:
$$
a_{2n}={(2n)!\over(2^n n!)^2}{\pi\over2};\qquad a_{2n+1}={(2^n n!)^2\over (2n+1)!}
$$
Since they turn out nicely for $n=0$ and $n=1$, now for induction step:
$$
a_{2n+2}={2n+1\over 2n+2}2{(2n)!\over(2^n n!)^2}{\pi\over2}=2{(2n+2)(2n+1)\over (2n+2)^2}{(2n)!\over(2^n n!)^2}{\pi\over2}={(2n+2)!\over(2^{n+1}(n+1)!)^2}{\pi\over2}
$$
$$
a_{2n+3}={2n+2\over 2n+3}{(2^n n!)^2\over (2n+1)^2}
={(2n+2)^2\over(2n+2)( 2n+3)}{(2^n n!)^2\over (2n+1)!}
={(2^{n+1} (n+1)!)^2\over (2n+3)!}
$$
Finally, since
$$
\lim_{n\to\infty}a_{2n}=\lim_{n\to\infty}a_{2n+1}
$$
 We conclude that:
 $$
 {\pi\over 2}{1\cdot3\cdot5\cdot7\over2\cdot4\cdot6\cdot8}\cdots=
 {2\cdot4\cdot6\cdot8\over3\cdot5\cdot7\cdot9}\cdots
 $$
 Namely
  $$
{\pi\over 2}={2\cdot2\cdot4\cdot4\cdot6\cdot6\cdot8\cdot8
\over1\cdot3\cdot3\cdot5\cdot5\cdot7\cdot7\cdot9}\cdots$$
\bigskip
\bigskip
{\obeylines\smallskip\sl
\hfill What we have to learn to do we learn by doing
\smallskip\hfill\rm --- {\tt ARISTOTLE},{\sl~Ethica Nicomachea II~\/}(c. 325~B.C.)
}
\end