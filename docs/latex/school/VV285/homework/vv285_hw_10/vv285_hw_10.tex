\documentclass{article}
\input mla
\begin{document}
\titlehm{10}
\ms{1}{
\item[i)]
See the attachment
\item[ii)]
See the attachment, streamlines are hand drawn.
\item[iii)]
In the following order:
\begin{center}
\begin{tabular}{|c|c|c|}
\hline
Field&Grad&Rot\\
\hline
i)&$0$&$-c$\\
ii)&$0$&$2cy$\\
iii)&$0$&$2c$\\
iv)&$0$&$0$\\
v)&$0$&$0$\\
vi)&$0$&$0$\\
\hline
\end{tabular}
\end{center}
\item[iv)]
In the following order (counter-clockwise):
\begin{center}
\begin{tabular}{|c|c|c|}
\hline
Field&Flux&Circulation\\
\hline
i)&$0$&$-c\pi$\\
ii)&$0$&$0$\\
iii)&$0$&$2c\pi$\\
iv)&$0$&$2c\pi$\\
v)&$2\pi c$&$0$\\
vi)&$0$&$0$\\
\hline
\end{tabular}
\end{center}
}
\ms{2}{
\item[i)]
We check the first component of rot$(\phi F)$:
\eq{
(\curl{(\phi F)})_1=\pd{}{y}(\phi F_3)-\pd{}{z}(\phi F_2)&=\pd{\phi}{y}F_3+\phi\pd{F_3}{y}
-\pd{\phi}{z}F_2-\phi\pd{F_2}{z}\\
&=\phi\left(\pd{F_3}{y}-\pd{F_2}{z}\right)+\left(F_3\pd{\phi}{y}-F_2\pd{\phi}{z}\right)\\
&=(\phi\curl{F})_1+((\grad{\phi})\times F)_1
}
the other two components follow a similar proof.
\item[ii)]
We check the first term of div$(\phi F)$:
\eq{
(\hbox{the first term of }\diver{(\phi F)})&=F_1\pd{\phi}{x}+\phi\pd{F_1}{x}\\
&=(\hbox{the first term of }\dotp{F}{\grad{\phi}})+(\hbox{the first term of }\phi\diver{F})
}
the other two terms follow a similar proof.
\item[iii)]
For div$(F\times G)$:
\eq
{
(\diver{(F\times G)})=\,&\pd{}{x}(F_2G_3-F_3G_2)+\pd{}{y}(F_3G_1-F_1G_3)+
\pd{}{z}(F_1G_2-F_2G_1)\\
=\,&G_1\left(\pd{F_3}{y}-\pd{F_2}{z}\right)-F_1\left(\pd{G_2}{z}-\pd{G_3}{y}\right)\\
& G_1\left(\pd{F_3}{y}-\pd{F_2}{z}\right)-F_1\left(\pd{G_2}{z}-\pd{G_3}{y}\right)\\
&G_1\left(\pd{F_3}{y}-\pd{F_2}{z}\right)-F_1\left(\pd{G_2}{z}-\pd{G_3}{y}\right)\\
&=\dotp{G}{{\curl{F}}}-\dotp{F}{\curl{G}}
}
\item[iv)]
Again we check the first component of rot$($rot$F)$:
\eq
{
(\curl{(\curl{F})})_1&=\curl\left({\pd{F_3}{y}-\pd{F_2}{z}}\,
,{\pd{F_1}{z}-\pd{F_3}{x}},\,
{\pd{F_2}{x}-\pd{F_1}{y}}\right)_1\\
&=\dpd{F_1}{x}+\spd{F_2}{y}{x}+\spd{F_3}{z}{x}-\left(\dpd{}{x}+\dpd{}{y}+\dpd{}{z}\right)F\\
&=(\grad{(\diver{F})})_1-(\Delta{F})_1
}
other two components follows a similar proof.
\item[v)]
Again we check the first component of rot$($grad$\phi)$
\eq
{
(\curl{(\grad{\phi})})_1=\pd{}{y}\pd{\phi}{z}-\pd{}{z}\pd{\phi}{y}=0
}
other two components follows a similar proof to be $0$.
}
\ms{4}
{
\item[]
First we notice that the differential operators communicate with each
other:
\eq{
\pd{}{t}\nabla=\nabla\pd{}{t}
}
since for arbitrary coordinates $\pd{}{x}\pd{}{t}=\pd{}{t}\pd{}{x}$.
Then we may proceed by 
\def\u{\mu_0}
\def\e{\epsilon_0}
\def\c{c_0}
\eq
{
-\curl{(\curl{E})}&=\u\curl{\pd{H}{t}}=\u\pd{}{t}\curl{H}=\u\e\dpd{E}{t}
}
and the left hand side is 
\eq
{
\curl{(\curl{E})}=\Delta E-\grad({\diver{E}})=\Delta E
}
the above two give us the relation:
\eq
{
\ff{1}{\u\e}\Delta E=\c^2\Delta E =\dpd{E}{t}
}
The conjugate counterpart for $H$ is completely analogy:
\eq
{
-\curl{(\curl{H})}&=\e\curl{\pd{E}{t}}=\e\pd{}{t}\curl{E}=\e\u\dpd{H}{t}
}
and the left hand side is 
\eq
{
\curl{(\curl{H})}=\Delta H-\grad({\diver{H}})=\Delta H
}
the above two give us the relation:
\eq
{
\ff{1}{\u\e}\Delta H=\c^2\Delta H =\dpd{H}{t}
}
}
\ms{5}
{
We calculate the rot$F$:
\eq
{
\curl{F}=(b-a,\,0,\,0)=(0,\,0,\,0)
}
gives us 
\eq
{
b=a
}
and we proceed setting $u=\grad{F}$ and replacing $b$ by $a$:
\eq
{
\pd{u}{x}&=x^2+xy\\
\pd{u}{y}&=\ff{x^2}{2}+y+az\\
\pd{u}{z}&=ay
}
giving us:
\eq
{
u=\ff{1}{3}x^3+\ff{1}{2}x^2y+\ff{1}{2}y^2+ayz+\rm const
}
}
\end{document}