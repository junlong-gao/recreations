\input vv285mla
\begin{document}
\titlehm{6}
\ep{1}{
\item[(i)] Let $\|\cdot\|$ denote the infinity norm $\|\cdot\|_{\infty}$:
\eq{
\|g(x_1,x_{2})-g(0,0)\|=
\|(x_{1}^{2}+x_{2}^{2})\sin{\ff{1}{\sq{x_{1}^{2}+x_{2}^{2}}}}\|
\leq \|x\|\cdot 1
=\|x\|,\qquad \quad 0<\|x\|<1
}
Using the infinity norm we see that given $\epsilon>0$ we let $\delta=\epsilon/2$
and $\|x\|<\delta$ implies $\|g(x_1,x_{2})-g(0,0)\|<\delta<\epsilon$
\item[(ii)]
\def\uu{{\ff{1}{\sq{x_{1}^{2}+x_{2}^{2}}}}}
Now we use convention: $\partial_{x_{1}}g=g_{1}$
For the points not equal to $(0,0)$:
\ed{
&g_{1}(x_{1},x_{2})=x_{1}(2\sin\uu-\uu\cos\uu)\label{eq:1}\\
&g_{2}(x_{1},x_{2})=x_{2}(2\sin\uu-\uu\cos\uu)\label{eq:2}
}
and for partials at $(0,0)$:
\eq{
&g_{1}(x_{1},0)=\lim_{x_{1}\to0}\ff{x_{1}^{2}\sin\ff{1}{|x_{1}|}-0}{x_{1}-0}=0\\
&g_{2}(0,x_{2})=\lim_{x_{2}\to0}\ff{x_{2}^{2}\sin\ff{1}{|x_{2}|}-0}{x_{2}-0}=0
}
Now we notice that since
\eq{
\ff{x_{1}}{\sq{x_{1}^{2}+x_{2}^{2}}}=\ff{1}{\sq{1+x_{2}^{2}/x_{1}^{2}}}
}
remains bounded when $(x_{1},x_{2})\to(0,0)$, yet 
\eq{
\sin\uu,\qquad\cos\uu
}
fails to exist when $(x_{1},x_{2})\to(0,0)$, we conclude that (\ref{eq:1}) and 
(\ref{eq:2}) all fail to exist at $(0,0)$, therefore the partial derivatives are not continous at $(0,0)$.
\item[(iii)] It is differentiable at the origin since 
\eq{
g(h_1,h_2)-g(0,0)&=(h_{1}^{2}+h_{2}^{2})\sin{\ff{1}{\sq{h_{1}^{2}+h_{2}^{2}}}}\\
&\quad=0+o(\|h\|)
}
since $(h_{1}^{2}+h_{2}^{2})\sin{\ff{1}{\sq{h_{1}^{2}+h_{2}^{2}}}}\leq\|h\|^{2}\cdot1<\|h\|$ for sufficiently small $\|h\|$, therefore it's differentiable at origin, its differential being zero mapping.
}
\es{2}{
\item[(i)]
We set $G:t\mapsto(f(t),g(t))$, and $H:(x,y)\mapsto F(x,y)$, then $F=H\circ G$
in question, therefore
\eq{
\ff{d}{dt}F(t)=DF\at{t}=DH\at{G(t)}\circ DG\at{t}
}
where
\eq{
DH\at{G(t)}=\mm{\pd{F}{x}(x(t),y(t))&\pd{F}{y}(x(t),y(t))}\qquad
DG\at{t}=\mm{f'(t)\\g'(t)}
}
plugging in we obtain:
\eq{
\ff{d}{dt}F(x(t),y(t))&=DF\at{t}=\mm{\pd{F}{x}(x(t),y(t))&\pd{F}{y}(x(t),y(t))}
\mm{f'(t)\\g'(t)}\\
&=\quad f'(t)\partial_{1}F(x(t),y(t))+g'(t)\partial_{2}F(x(t),y(t))
}
\item[(ii)]
We set $G:y\mapsto(y,\a(y),\b(y))$, $H:(y,\a,\b)\mapsto\int_{\a}^{\b}f(x,y)\,dx$ 
then the mapping $I=H\circ G$
we apply chain rule:
\eq{
I'=DI=DH\circ DG
}
and 
\eq{
DG=\mm{1\\\a'(y)\\\b'(y)}
}
Now, we proceed by theorem:
\eq{
\pd{H}{y}=\int_{\a}^{\b}\pd{f}{y}(x,y)\,dx
}
and observe that
\eq{
&\pd{H}{\a}=\ff{d}{d\a}\int_{\a}^{\b}f(x,y)\,dx=-\ff{d}{d\a}\int_{\b}^{\a}f(x,y)\,dx=-f(\a,y)\\
&\pd{H}{\b}=\ff{d}{d\b}\int_{\a}^{\b}f(x,y)\,dx=f(\b,y)
}
by the fundamental theorem of calculus.\\
Now we expand out the differential of $H$:
\eq{
DH\at{G(y)}=\mm{\int_{\a(y)}^{\b(y)}\pd{f}{y}(x,y)\,dx&-f(\a(y),y)&f(\b(y),y)}
}
And plugging in:
\eq{
I'(y)=DI\at{y}=DH\at{G(y)}\circ DG\at{y}=\int_{\a(y)}^{\b(y)}\pd{f}{y}(x,y)\,dx-f(\a(y),y)\a'(y)+f(\b(y),y)\b'(y)
} 
}
\es{3}{
\item{$f(x,y)$:}
\eq{
&\dpd{f}{x}=2e^{xy}+4xye^{xy}+y^{2}(x^{2}+y^{2})e^{xy}\\
&\ff{\partial^{2}f}{\partial x\partial y}=\ff{\partial^{2}f}{\partial y\partial x}=
2x^{2}e^{xy}+xye^{xy}(x^{2}+y^{2})+2y^{2}e^{xy}\\
&\dpd{f}{y}=2e^{xy}+2xye^{xy}+x^{2}(x^{2}+y^{2})e^{xy}+2xye^{xy}
}
\item{$g(x,y,z)$:}
\def\sss{\sin(x+y+z)}
\def\css{\cos(x+y+z)}
\def\damm#1#2#3{
\eq{
&\dpd{g}{#1}=2#2#3\cos(#1+#2+#3)-#1#2#3\sin(#1+#2+#3)\\
&\spd{g}{#2}{#1}=#3\sss+#2#3\css+#1#3\css-#1#2#3\sss\\
&\spd{g}{#3}{#1}=#2\sss+#2#3\css+#1#2\css-#1#2#3\sss
}
}
\damm{x}{y}{z}
\damm{y}{x}{z}
\damm{z}{y}{x}
}
\es{4}{
\item[(i)]
First
\eq{
f\times g=\mm{s^{3}t^{2}-4st^{2}\\
			2s^{2}t-2st^{2}-t^{2}s^{2}-t^{3}s\\
			2st+2t^{2}-s^{3}+s^{2}t}
}
then differentiate it:
\eq{
D(f\times g)=\mm{3s^{2}t^{2}-4t^{2}&2s^{3}t-8st\\
				4st-2t^{2}-2st^{2}-t^{3}&2s^{2}-4st-2ts^{2}-3t^{2}s\\
				2t-3s^{2}+2ts&2s+4t+s^{2}	
			}
}
\item[(ii)]
Now we calculate the differential seperately:
\eq{
Df=\mm{1&1\\
		2s&0\\
		2t&2s}
\qquad
Dg=\mm{1&-1\\
		0&2\\
		t^{2}&2ts}
}
And use the product formula:
\eq{
&Df\at{(s,t)}(h)\times g+f\times Dg\at{(s,t)}(h)\\
&=\mm{h_{1}(2s^{2}t^{2}-4t^{2})-4sth_{2}\\
		h_{1}(2st-2t^{2}-t^{2}s)+h_{2}(2s^{2}-2st-t^{2}s)\\
		h_{1}(2t-2s^{2}+2ts)+2th_{2}}
\\&\quad+\mm{h_{1}s^{2}t^{2}+h_{2}(2ts^{3}-4st)\\
		h_{1}(2st-st^{2}-t^{3})+h_{2}(-2st-2ts^{2}-2^{2}s)\\
		h_{1}(-s^{2})+h_{2}(2s+2t+s^{2})}\\
		&\quad=D(f\times g)\at{(s,t)}(h)\qquad\hbox{from (i)}
}
as desired.
}
\es{5}{
\def\p{\varphi}
\def\P{\Phi}
\def\cp{\cos{\varphi}}
\def\ccp{\cos^{2}{\varphi}}
\def\sp{\sin{\varphi}}
\def\ssp{\sin^{2}{\varphi}}
\item[]
Now we consider mapping: $f:(x_{1},x_{2})\mapsto f(x_{1},x_{2})$, and $\Phi:
(r,\varphi)\mapsto (x_{1},x_{2})$ and now we need to find the second differential of the composite mapping $g:=f\circ\Phi$\\
First, by chain rule:
\eq{
Dg&=Df\at{\Phi(r,\p)}\circ D\P\at{(r,\p)}=\mm{\pd{f}{x_{1}}&\pd{f}{x_{2}}}
\mm{\cp&-r\sp\\
	\sp&r\cp}\\
	&\quad=\mm{f_{1}\cp+f_{2}\sp&-f_{1}r\sp+f_{2}r\cp}
}
On the other hand, we know from the definition of Jacobian:
\eq{
Dg=\mm{\pd{g}{r}&\pd{g}{\phi}}=\mm{\pd{f}{r}(r\cp,r\sp)&\pd{f}{\p}(r\cp,r\sp)}
}
Again we have use the convention $f_{1}=\pd{f}{x_{1}}$.\\
Now in order to get the second differential we need to differentiate $f_{1}\circ\P$ and $f_{2}\circ\P$, by treating them as before by simply replacing $f$ in $f\circ\P$  by $f_{1}$ and $f_{2}$, then we get that
\eq{
&\mm{\pd{f_{1}}{r}(r\cp,r\sp)&\pd{f_{1}}{\p}(r\cp,r\sp)}
=\mm{f_{11}\cp+f_{21}\sp&-f_{11}r\sp+f_{21}r\cp}\\
&\mm{\pd{f_{2}}{r}(r\cp,r\sp)&\pd{f_{2}}{\p}(r\cp,r\sp)}=\mm{f_{12}\cp+f_{22}\sp&-f_{12}r\sp+f_{22}r\cp}
} 
Now we plug the second differential:
\eq{
&\mm{\dpd{f}{r}(r\cp,r\sp)&\spd{f}{r}{\p}(r\cp,r\sp)\\
	\spd{f}{\p}{r}(r\cp,r\sp)&\dpd{f}{\p}(r\cp,r\sp)}\\
	&\quad=D^{2}g=\mm{\pd{}{r}(f_{1}\cp+f_{2}\sp)&(*)\\
						(*)&\pd{}{\p}(-f_{1}r\sp+f_{2}r\cp)}
}
where we merely need the first and the last entry:
\eq{
\dpd{f}{r}(r\cp,r\sp)&=\pd{}{r}(f_{1}\cp+f_{2}\sp)&\\
&=f_{11}\ccp+f_{12}\sp\cp+f_{11}\sp\cp+f_{12}\ssp&\\
\ff{1}{r^{2}}\dpd{f}{\p}(r\cp,r\sp)&
=\ff{1}{r^2}\pd{}{\p}(-f_{1}r\sp+f_{2}r\cp)&\\
&=(f_{11}\sp-f_{12}\cp\sp)\sp-\ff{1}{r}\cp f_{1}&\\
&+(-f_{12}\sp+f_{22}\cp)r\cp+(-\ff{1}{r}f_{2}\sp)&\\
\ff{1}{r}\pd{f}{r}(r\cp,r\sp)&=\ff{1}{r}f_{1}\cp+\ff{1}{r}f_{2}\sp&
}
Finally, we add them together to have:
\eq{
(\dpd{}{r}+\ff{1}{r^{2}}\dpd{}{\p}+\ff{1}{r}\pd{}{r})f(r\cp,r\sp)=f_{11}+f_{22}=(\dpd{}{x_{1}}+\dpd{}{x_{2}})f(x_{1},x_{2})
}
which completes the proof.
}
\es{6}{
\item[]
First we calculate:
\eq{
DF=\mm{-1&-1\\2x_{1}&2x_{2}}\qquad DFy=\mm{-3\\2x_{1}+4x_{2}}\qquad x+ty=\mm{t\\2t}
}
Then
\eq{
&F(x+y)-F(x)=\mm{-2\\5}-\mm{1\\0}=\mm{-3\\5}\\
&\int_{0}^{1}DF\at{x+ty}y\,dt=\int_{0}^{1}\mm{-3\\2t+8t}\,dt=\mm{-3\\5}\\
&\left(\int_{0}^{1}DF\at{x+ty}\,dt\right)y=\mm{-1&-1\\1&2}\mm{1\\2}=\mm{-3\\5}
}
Therefore
\eq{
F(x+y)-F(x)=\int_{0}^{1}DF\at{x+ty}y\,dt=\left(\int_{0}^{1}DF\at{x+ty}\,dt\right)y
}
}
\es{7}{
\def\ii{\int_{0}^{\infty}}
\item[(1)]
\eq{
g'(x)=\ff{d}{dt}\int_{0}^{\infty}\ff{\sin t}{t}e^{-xt}\,dt=\int_{0}^{\infty}\ff{\sin t}{t}
(-t)e^{-xt}\,dt=(-1)\int_{0}^{\infty}\sin te^{-xt}\,dt
}
and by integration by parts we obtain:
\eq
{
g'(x)=-\ff{1}{1+x^{2}}
}
\item[(2)]
Consider function $f(t)=\sin t-t$ which is an odd function:
\eq{
f'(t)=\cos t-1\leq0\Longrightarrow f(t)\leq f(0)=0
}
This shows that $\sin t\leq t$ when $0<t<1<\pi$ and therefore $|f(t)|/t\leq 1$ for $0<t<1$, and for $t>1$ it's trivial that $|\sin t|<1$\\
Using this inequality we can estimate that
\eq{
|g(x)|\leq\ii|\ff{\sin t}{t}e^{-xt}|\leq\ii e^{-xt}\,dt=\ff{1}{x}
}
therefore $g(+\infty):=\lim_{x\to\infty}g(x)=0$
\item[(iii)]
Now we apply Newton's formula:
\eq{
g(+\infty)-g(0)=\ii g'(x)\,dx=-\ii\ff{1}{1+x^{2}}\,dx=-\arctan x\Big|_{0}^{+\infty}=-\ff{\pi}{2}
}
and by previous results:
\eq{
\ii\ff{\sin t}{t}\,dt=g(0)=\ff{\pi}{2}
}
}
\es{8}{
\def\io{\int_{0}^{1}}
\item[]
Due to the decay of $e^{-\ff{1}{y}}$ when $y$ is near $0$, the integrals here are all proper.\\
When $x\neq 0$:
\eq{
\io\ff{x^{3}}{y^{2}}e^{-\ff{x^{2}}{y}}\,dy=xe^{-\ff{x^{2}}{y}}\Big|_{0}^{1}=xe^{-x^{2}}
}
and when $x=0$
\eq{
\io 0\,dy=0}
For another integral, when $x\neq 0$:
\eq{
\pd{}{x}f(x,y)=e^{-\ff{x^{2}}{y}}\left(\ff{3x^{2}}{y^{2}}-\ff{2x^{4}}{y^{3}}\right)
}
then
\eq{
\io \pd{}{x}f(x,y)\,dy=(1-2x^{2})e^{-x^{2}}
}
again for $x=0$:
\eq{
\pd{}{x}f(x,y)=0\Longrightarrow\io \pd{}{x}f(x,y)\,dy=0
}
then finally 
\eq{
&\ff{d}{dx}\io f(x,y)\,dy\at{x=0}=\lim_{x\to0}\ff{xe^{-x^{2}}-0}{x-0}=1\\
&\io \pd{}{x}f(x,y)\,dy\at{x=0}=0\\
&\qquad\Longrightarrow
\ff{d}{dx}\io f(x,y)\,dy\at{x=0}\neq\io \pd{}{x}f(x,y)\,dy\at{x=0}
}
\it Note: this is due to the fact that $f(x,y)$ is not continous at $(0,0)$ since 
if we consider the path $y=x^{2}$
}
\end{document}

