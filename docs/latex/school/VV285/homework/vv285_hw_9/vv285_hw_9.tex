\documentclass{article}
\input mla
\begin{document}
\titlehm{9}
\ms{1}{
\item[i)]
\eq{
E(p)=-\grad V(p)=\ff{Q}{4\pi\epsilon_0}\ff{x}{\|x\|^3}
}
\item[ii)]
\eq{
E(p)=-\grad V(p)=\ff{1}{4\pi\epsilon_0}\ff{3\dotp{d}{x-y}(x-y)-\|x-y\|^2d}{\|x-y\|^5}
}
}
\ms{2}{
\item[i)]
\eq{
&V(0,0,z_0)=\ff{1}{4\pi\epsilon_0}\int_{-L/2}^{L/2}\ff{\rho_l}{|z_0-t|}\,dt
= \begin{cases}
 -\ff{\rho_l}{4\pi\epsilon_0}\ln\ff{|2z_0-L|}{|2z_0+L|}
 &\mbox{if } z_0> L/2 \\ 
\ff{\rho_l}{4\pi\epsilon_0}\ln\ff{|2z_0-L|}{|2z_0+L|} & \mbox{if } z_0<-L/2
\end{cases}
\\
&E(0,0,z_0)=-\grad V(0,0,z_0)=
\begin{cases}
\ff{\rho_l}{4\pi\epsilon_0}\left(\ff{1}{z_0-L/2}-\ff{1}{z_0+L/2}\right)&\mbox{if } z_0> L/2\\
-\ff{\rho_l}{4\pi\epsilon_0}\left(\ff{1}{z_0-L/2}-\ff{1}{z_0+L/2}\right) & \mbox{if } z_0<-L/2\\
\end{cases}
}
\item[ii)]
\eq{
&V(x_0,0,0)=\ff{\rho_l}{2\pi\epsilon_0}\ln\ff{\sq{(L/2)^2+x_0^2}+L/2}{|x_0|}\\
&E(x_0,0,0)=\ff{\rho_l}{4\pi\epsilon_0}\left(\ff{L}{x_0\sq{(L/2)^2+x_0^2)}},0,0\right)
}
\item[iii)]
\eq{
&V(0,0,z_0)=\ff{\rho}{2\epsilon_0z_0}\ff{R}{\sq{1+R^2/z_0^2}}\\
&E(0,0,z_0)=-\grad V(0,0,z_0)=\ff{\rho}{2\epsilon_0}\left(0,0,\ff{R}{z_0^2}\right)
\left(1+R^2/z_0^2\right)^{-3/2}
}
Notice when $R/|z_0|\to0$, then term $\left(1+R^2/z_0^2\right)^{-3/2}\to 1$, thus up to a constant, the field intensity behaves as if the circle is replaced by a 
point charge.
\item[iv)]
\eq{
V(0,0,0)=\ff{\rho}{2\epsilon_0 }
}
\item[v)]
\eq{
|E(0,0)|=\ff{\rho}{2\pi\epsilon_0 b}
}
pointing outward from the semicircle.
}
\ms{3}
{
\item[]
Consider parameterization:
\eq
{
y=\st\quad x=\ct
}
\eq{
\oint_{C^+}y^2\,dx+3xy\,dy=\ff{2}{3}
}
}
\ms{4}
{
\item[i)]
\eq{
\pd{F_1}{y}=\ff{y^2-x^2}{(x^2+y^2)^2}=\pd{F_2}{x}\quad
\rm on
\quad
\R^2\,\backslash\, \{0\}
}
\item[ii)]
\eq{
\oint_{S^1}F\ds=\oint_{S^1}\ff{-y}{x^2+y^2}\,dx+\ff{x}{x^2+y^2}\,dy
=2\pi
}
with the parametrization:
\eq{
y=\st\quad x=\ct
}
and this concludes that the field is not conservative, thereby it's impossible to have a potential.
\item[iii)]
One can verify 
\eq{
f(x,y)=-\arctan\ff{x}{y},\quad \Omega\mapsto \R^2
}
satisfy the relation
\eq{
F(x,y)=\grad f(x,y)\quad {\rm on }\quad\Omega
}
}
\ms{5}
{
\item[i)]
We use the parameterization:
\eq{
x=t\quad y=2t\quad z=4t \quad t:\,0\to 1
}
Then
\eq{
\int_{S}L=\int_0^1 16t^2-3t\,dt=\ff{23}{6}
}
\item[ii)]
We use the parameterization:
\eq{
&\Gamma_1:\,x=t^2\quad y=t \quad t:\,-1\to 1\\
&\Gamma_2: x=1\quad y:-1\to 1
}
Then
\eq{
\int_{S}L=\int_{-1}^1 2t^6+t^3\,dt+\int_{-1}^{1}y^2\,dy=\ff{4}{15}
}
}
\ms{6}
{
\item[i)]
\eq{
\pd{F_1}{y}=-2x\sin y=\pd{F_2}{x}\quad{\rm on}\quad\R^2
}
This it's satisfy the condition over a simply connected region, and it's
a gradient field, i.e., there exists a potential $u$ such that $F=\grad u$.
\item[ii)]
\eq{
u(x,y)=x^2\cos y
}
\item[iii)]
\eq{
\int_{\Gamma}F=\int_0^12e^{2t}\cos t-e^{2t}\sin t dt=e^{2t}\cos t\Big|_0^1=e^2\cos 1-1
}
\item[iv)]
\eq{
\int_{\Gamma}F=u(e,1)-u(0,0)=e^2\cos 1-1
}
which reproduces the result, as desired.
}
\end{document}