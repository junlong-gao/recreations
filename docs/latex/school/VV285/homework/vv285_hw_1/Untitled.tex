\documentclass[12pt]{article}
\usepackage{geometry}
\usepackage{amsmath} 
\usepackage{amsthm}
\usepackage{amsfonts}
\usepackage{cases}
\usepackage{graphicx}
\usepackage{caption}
\usepackage{subcaption}
\usepackage{gauss}

\def\dotp#1#2{\langle#1,#2\rangle}
\def\ss#1#2{\sum_{#1=1}^{#2}}
\def\lam{\lambda}
\def\vec#1{\{#1_1,#1_2\ldots#1_n\}}
\def\es#1#2{{\bf Exercise #1}\\~{\it Solution:}\\~#2\\[1em]}
\def\ep#1#2{{\bf Exercise #1}\\~{\it Solution:}\\~#2\\[1em]}
\def\inn#1#2{(#1): #2\\[0.5em]}
\def\ff#1#2{\frac{#1}{#2}}


\def\sedb{$2^{\rm nd}$ }

\linespread{1.4}
\geometry{a4paper,centering,scale=0.8}
\rmfamily 
\normalsize
\setlength{\parindent}{0em}

\begin{document} 
\begin{flushleft}
  Junlong Gao 5133709126\\ 
  Prof.  Hohberger\\ 
  VV285 Assignment 1\\
  \today 
\end{flushleft}

\es{1}{
\inn{a}{
	\begin{align*}
	&\qquad\quad \left.\begin{gmatrix}
			   4 & 2 & -2 \\
			   -3 &1 & 0 \\
			   1 & 4 & 2
	\end{gmatrix}\right|\
	\begin{gmatrix}
	 -2 \\
	 6 \\
	 -9
	\rowops
	\swap 02
	\end{gmatrix}
	 &&\sim\quad
	\left.\begin{gmatrix}
			   1 & 4 & 2 \\
			   -3 &1 & 0 \\
			   4 & 2 & -2
	\end{gmatrix}\right|\
	\begin{gmatrix}
	 -9 \\
	 6 \\
	 -2
	\rowops
	\add[\cdot 3]01
	\add[\cdot (-4)]02
	\end{gmatrix} \\[1Em]
	&\quad\sim\quad
	\left.\begin{gmatrix}
			   1 & 4 & 2 \\
			   0 &13 & 6 \\
			   0 & -14 & -10
	\end{gmatrix}\right|\
	\begin{gmatrix}
	 9 \\
	 -21 \\
	 -38
	\rowops
	\add[\cdot (\ff{14}{13})]12
	\mult1{:13}
	\end{gmatrix}
	 && \sim\quad
	\left.\begin{gmatrix}
			   1 & 4 & 2 \\
			   0 &1 & \frac{6}{13} \\
			   0 & 0 & -\frac{48}{13}
	\end{gmatrix}\right|\
	\begin{gmatrix}
	 -9 \\
	 -\frac{21}{13} \\
	 \frac{148}{13}
	\rowops
	\mult2{:(\frac{48}{13})}
	\end{gmatrix} \\[1Em]
	&\quad\sim\quad
	\left.\begin{gmatrix}
			   1 & 4 & 2 \\
			   0 &1 & \frac{6}{13} \\
			   0 & 0 & 1
	\end{gmatrix}\right|\
	\begin{gmatrix}
	 -9 \\
	 -\frac{21}{13} \\
	 -\frac{74}{23}
	\rowops
	\add[\cdot (-\frac{6}{13})]21
	\add[\cdot (-2)]20
	\end{gmatrix}
	&&\quad\sim\quad
	\left.\begin{gmatrix}
			   1 & 4 & 0 \\
			   0 &1 & 0 \\
			   0 & 0 & 1
	\end{gmatrix}\right|\
	\begin{gmatrix}
	 -\frac{59}{23} \\
	 -\frac{3}{23} \\
	 -\frac{74}{23}
	\rowops
	\add[\cdot (-4)]10
	\end{gmatrix}\\[1Em]
	&\quad\sim\quad
	\left.\begin{gmatrix}
			   1 & 0 & 0 \\
			   0 &1 & 0 \\
			   0 & 0 & 1
	\end{gmatrix}\right|\
	\begin{gmatrix}
	 -\frac{47}{23} \\
	 -\frac{3}{23} \\
	 -\frac{74}{23}
	\end{gmatrix}
	\end{align*}
	And read of the solution in the last reduced system: $x_1= -\frac{47}{23}, 
	x_2= -\frac{3}{23}, x_3= -\frac{74}{23}$
	}
\inn{b}{
	\begin{align*}
	&\qquad\quad \left.\begin{gmatrix}
			   1 & -3 & -5 \\
			   2 &-2 & 1 \\
			   -3 & 5 & -6
	\end{gmatrix}\right|\
	\begin{gmatrix}
	 26 \\
	 12 \\
	 2
	\rowops
	\add[\cdot (-2)]01
	\add[\cdot 3]02
	\end{gmatrix}
	 &&\sim\quad
	\left.\begin{gmatrix}
			   1 & -3 & -5 \\
			   0 &4 & 11 \\
			   0 & -4 & -21
	\end{gmatrix}\right|\
	\begin{gmatrix}
	 26 \\
	-40 \\
	 80
	\rowops
	\add[\cdot 1]12
	\end{gmatrix} \\[1Em]
	&\quad\sim\quad
	\left.\begin{gmatrix}
			   1 & -3 & -5 \\
			   0 &4 & 11 \\
			   0 & 0 & -10
	\end{gmatrix}\right|\
	\begin{gmatrix}
	 26 \\
	-40 \\
	 40
	\rowops
	\mult2{(-10)}
	\end{gmatrix}
	 && \sim\quad
	\left.\begin{gmatrix}
			  1 & -3 & -5 \\
			   0 &4 & 11 \\
			   0 & 0 & 1
	\end{gmatrix}\right|\
	\begin{gmatrix}
	 26 \\
	-40 \\
	-4
	\rowops
	\add[\cdot (-11)]21
	\add[\cdot 5]20
	\end{gmatrix} \\[1Em]
	&\quad\sim\quad
	\left.\begin{gmatrix}
			   1 & -3 & 0 \\
			   0 &4 & 0 \\
			   0 & 0 & 1
	\end{gmatrix}\right|\
	\begin{gmatrix}
	6 \\
	4 \\
	-4
	\rowops
	\mult1{4}
	\add[\cdot 3]10
	\end{gmatrix}
	&&\sim\quad
	\left.\begin{gmatrix}
			   1 & 0 & 0 \\
			   0 &1 & 0 \\
			   0 & 0 & 1
	\end{gmatrix}\right|\
	\begin{gmatrix}
	9\\
	1\\
	-4
	\end{gmatrix}
	\end{align*}
	And read of the solution in the last reduced system: $x_1= 9, 
	x_2= 1, x_3= -4$
	}
\inn{c}{
	\begin{align*}
	&\qquad\quad \left.\begin{gmatrix}
			   3 & 1 & -2 \\
			   24 &10 & -13 \\
			   -6 & -4 & 1
	\end{gmatrix}\right|\
	\begin{gmatrix}
	 3 \\
	 25 \\
	 -7
	\rowops
	\mult0{3}
	\add[\cdot (-24)]01
	\add[\cdot 6]02
	\end{gmatrix}
	 &&\sim\quad \left.\begin{gmatrix}
			   1 & \frac{1}{3} & -\frac{2}{3} \\
			   0 &2 & 3 \\
			   0 & -2 & -3
	\end{gmatrix}\right|\
	\begin{gmatrix}
	 1 \\
	 1 \\
	 -1
	\rowops
	\add[\cdot 1]12
	\end{gmatrix}\\[1em]
	&\quad\sim\quad \left.\begin{gmatrix}
			   1 & \frac{1}{3} & -\frac{2}{3} \\
			   0 &2 & 3 \\
			   0 & 0 & 0
	\end{gmatrix}\right|\
	\begin{gmatrix}
	 1 \\
	 1 \\
	 0
	\rowops
	\end{gmatrix}
	 \end{align*}
	This is a singular form and we introduce a parameter $z=t$ and get the general solution: 
	$x_1=\frac{5}{6}+\frac{7}{6}t, x_2=\frac{1}{2}-\frac{3}{2}t, x_3=t, t\in\mathbb{R}$
	}
\inn{d}{
	\begin{align*}
	&\qquad\quad \left.\begin{gmatrix}
			   1 & 2 & 1 & 1 \\
			   2 & 1 & 1 & 2 \\
			   1 & 2 & 1 & 1
	\end{gmatrix}\right|\
	\begin{gmatrix}
	 0 \\
	 0 \\
	 0
	\rowops
	\add[\cdot(-2)]01
	\add[\cdot(-1)]02
	\end{gmatrix}
	& &\sim\quad
	\left.\begin{gmatrix}
			   1 & 2 & 1 & 1 \\
			   0 & -3 & -1 & 0 \\
			   0 & 0 & 0 & 0
	\end{gmatrix}\right|\
	\begin{gmatrix}
	 0 \\
	 0 \\
	 0
	\end{gmatrix} \\[1Em]
	\end{align*}
	This is a singular form and we introduce a parameter $x_3=\alpha,x_4=\beta$ and get the general solution: 
	$x_1=-\frac{\alpha}{3}-\beta, x_2=-\frac{\alpha}{3}, x_3=\alpha,x_4=\beta.\quad \alpha,\beta\in\mathbb{R}$\\
	}
}
\es{2}{
\begin{align*}
&\qquad\quad \left.\begin{gmatrix}
		   1 & 1 & 1 & 1 \\
		   1 & -1 & -1 & -1 \\
		   1 & 1 & -1 & -1 \\
           3 & 1 & 1 & -1
\end{gmatrix}\right|\
\begin{gmatrix}
 \alpha \\
 \alpha-4 \\
 \alpha+1 \\
 0
\rowops
\add[\cdot(-1)]01
\add[\cdot(-1)]02
\add[\cdot(-3)]03
\end{gmatrix}
& &\sim\quad
\left.\begin{gmatrix}
		   1 & 1 & 1 & 1 \\
		   0 & -2 & -2 & -2 \\
		   0 & 0 & -2 & -2 \\
           0 & -2 & -2 & -4
\end{gmatrix}\right|\
\begin{gmatrix}
 \alpha \\
 -4 \\
 1 \\
 -3\alpha
\rowops
\mult1{:-2}
\mult2{:-2}
\mult3{:-2}
\add[\cdot(-1)]13
\end{gmatrix} \\[1Em]
&\quad\sim\quad
\left.\begin{gmatrix}
		   1 & 1 & 1 & 1 \\
		   0 & 1 & 1 & 1 \\
		   0 & 0 & 1 & 1 \\
           0 & 0 & 0 & 1
\end{gmatrix}\right|\
\begin{gmatrix}
 \alpha \\
 2 \\
 -\frac{1}{2} \\
 \frac{3}{2}\alpha-2
\rowops
\add[\cdot(-1)]32
\add[\cdot(-1)]31
\add[\cdot(-1)]30
\end{gmatrix}
& & \sim\quad
\left.\begin{gmatrix}
		   1 & 1 & 1 & 0 \\
		   0 & 1 & 1 & 0 \\
		   0 & 0 & 1 & 0 \\
           0 & 0 & 0 & 1
\end{gmatrix}\right|\
\begin{gmatrix}
 2-\frac{1}{2}\alpha \\
 4-\frac{3}{2}\alpha \\
 \frac{3}{2}-\frac{3}{2}\alpha \\
 \frac{3}{2}\alpha-2
\rowops
\add[\cdot(-1)]21
\add[\cdot(-1)]20
\end{gmatrix}\\[1Em]
&\quad\sim\quad \left.\begin{gmatrix}
		   1 & 1 & 0 & 0 \\
		   0 & 1 & 0 & 0 \\
		   0 & 0 & 1 & 0 \\
          	   0 & 0 & 0 & 1
\end{gmatrix}\right|\
\begin{gmatrix}
 \alpha+\frac{1}{2} \\
 \frac{5}{2} \\
 \frac{3}{2}-\frac{3}{2}\alpha \\
 \frac{3}{2}\alpha-2
\rowops
\add[\cdot(-1)]10
\end{gmatrix}
& &\sim\quad
\left.\begin{gmatrix}
		   1 & 0 & 0 & 0 \\
		   0 & 1 & 0 & 0 \\
		   0 & 0 & 1 & 0 \\
           0 & 0 & 0 & 1
\end{gmatrix}\right|\
\begin{gmatrix}
 \alpha-2 \\
 \frac{5}{2} \\
 \frac{3}{2}-\frac{3}{2}\alpha \\
 \frac{3}{2}\alpha-2
\end{gmatrix} \\
\end{align*}
Thus given $\alpha\in\mathbb{R}$, the system has the corresponding solution:
\begin{equation*}
x_1= \alpha-2,\quad x_2=\frac{5}{2},\quad x_3=\frac{3}{2}-\frac{3}{2}\alpha,\quad x_4=\frac{3}{2}\alpha-2
\end{equation*}
}
\es{3}{
According to Kirchhoffs laws applied to each node:
\begin{equation*}
\begin{cases}
I_1=I_2+I_4\\
I_3+I_4=I_5\\
V_1-3I_1R-I_2R+V_2=0\\
V_2+2I_4R+V_3-I_3R-I_2R=0\\
V_3-I_3R-I_5R=0
\end{cases}
\end{equation*}
Plug in the numbers and write it into a linear system:
\begin{equation*}
\begin{bmatrix}
		   1 & -1 & 0 & -1 & 0 \\
		   0 & 0 & 1 & 1 & -1 \\
		   1 & \frac{1}{3} & 0 & 0 & 0 \\
           0 & 1 & 1 & -2 & 0 \\
           0 & 0 & 1 & 0 & 1
\end{bmatrix}
\begin{bmatrix}
I_1 \\
I_2 \\
I_3 \\
I_4 \\
I_5
\end{bmatrix}
=
\begin{bmatrix}
 0 \\
 0 \\
 \frac{20}{33} \\
 \frac{50}{33} \\
 \frac{20}{33}
\end{bmatrix}
\end{equation*}
Giving the result:
\begin{equation*}
\begin{bmatrix}
I_1 \\
I_2 \\
I_3 \\
I_4 \\
I_5
\end{bmatrix}
=
\begin{bmatrix}
 \frac{170}{429} \\
 \frac{90}{143} \\
 \frac{60}{143}\\
 -\frac{100}{429}\\
\frac{80}{429}
\end{bmatrix}
({\rm A})
\end{equation*}
}
\es{4}{
(1) Plugging in all the angle we get
\begin{equation*}
   \begin{cases}
m_1-m_2-m_3+m_4=0\\
z_1m_1-z_2m_2-z_3m_3+z_4m_4=0\\
m_1+m_2+m_3+m_4=0\\
z_1m_1+z_2m_2+z_3m_3+z_4m_4=0
   \end{cases}
\end{equation*}
And consider the matrix
\[
\left.\begin{gmatrix}
1&-1&-1&1\\
z_1&-z_2&-z_3&z_4\\
1&1&1&1\\
z_1&z_2&z_3&z_4
\end{gmatrix}\right|
\begin{gmatrix}
0\\
0\\
0\\
0
\end{gmatrix}
\quad\sim\quad
\left.\begin{gmatrix}
1&0&0&0\\
0&1&0&0\\
0&0&1&0\\
0&0&0&1
\end{gmatrix}\right|
\begin{gmatrix}
0\\
0\\
0\\
0
\end{gmatrix}
\]
Therefore only trivial solution is possible, and the \sedb is impossible.\\[0.5 em]
(ii) Plugging all the angles, note that eight equations reduce to only four different equations:
\[
\left.\begin{gmatrix}
0&\frac{\sqrt3}{2}&-\frac{\sqrt3}{2}&-\frac{\sqrt3}{2}&\frac{\sqrt3}{2}&0\\
0&\frac{\sqrt3}{2}z_2&-\frac{\sqrt3}{2}z_3&-\frac{\sqrt3}{2}z_4&\frac{\sqrt3}{2}z_5&0\\
1&-\frac{1}{2}&-\frac{1}{2}&-\frac{1}{2}&-\frac{1}{2}&1\\
z_1&-\frac{1}{2}z_2&-\frac{1}{2}z_3&-\frac{1}{2}z_4&-\frac{1}{2}z_5&z_6
\end{gmatrix}\right|
\begin{gmatrix}
0\\
0\\
0\\
0
\end{gmatrix}
\]
By the fundamental lemma, we know that this has a non-trivial solution since the number of unknowns is larger than the number of equations. By arranging $z_i$'s we can obtain a positive solution and it's possible to have a \sedb balance solution.\\
In fact, if we write out the solution explicitly:
\begin{align*}
&m_1=\ff{z_5-z_2}{z_1-z_2}m_5+\ff{z_2-z_6}{z_1-z_2}m_6,\quad m_2=\ff{z_5-z_1}{z_1-z_2}m_5+\ff{z_1-z_6}{z_1-z_2}m_6\\
&m_3=\ff{z_5-z_2}{z_1-z_2}\ff{z_1-z_4}{z_3-z_4}m_5+\ff{z_1-z_6}{z_1-z_2}\ff{z_2-z_4}{z_3-z_4}m_6,\quad m_4=\ff{z_2-z_5}{z_1-z_2}\ff{z_1-z_3}{z_3-z_4}m_5+\ff{z_2-z_5}{z_3-z_4}\ff{z_1-z_6}{z_1-z_2}m_6
\end{align*}
with $m_5,m_6>0$, then one can verify that as long as $z_6<z_4<z_3<z_2<z_1<z_5$, the $m_i$'s can be achieved positively.
\\[0.5 em]
(iii)
Consider the skeleton
\[
\left.\begin{gmatrix}
1&1&1&1\\
3&1&-1&-3
\end{gmatrix}\right|
\begin{gmatrix}
0\\
0
\end{gmatrix}
\]
has the solution of the form:
$$
x_1=p+2q\quad x_2=-2p-3q\quad x_3=p\quad x_4=q\qquad p,q\in\mathbb{R}
\eqno{(*)}$$
which all the trigonometric functions have to satisfy, in particular:
\[
\begin{bmatrix}
\cos\alpha_1\\
\cos\alpha_2\\
\cos\alpha_3\\
\cos\alpha_4
\end{bmatrix}
=
\begin{bmatrix}
p+2q\\
-2p-3q\\
p\\
q
\end{bmatrix}
\quad
\begin{bmatrix}
\cos2\alpha_1\\
\cos2\alpha_2\\
\cos2\alpha_3\\
\cos2\alpha_4
\end{bmatrix}
=
\begin{bmatrix}
2{(p+2q)}^2-1\\
2{(-2p-3q)}^2-1\\
2p^2-1\\
2q^2-1
\end{bmatrix}
\]
and we are lead to
\begin{align*}
&2{(p+2q)}^2-1=2p^2-1+2(2q^2-1)\\
&2{(-2p-3q)}^2-1=(-2)(2p^2-1)-3(2q^2-1)
\end{align*}
The solution is
\[
\begin{bmatrix}
\cos\alpha_1\\
\cos\alpha_2\\
\cos\alpha_3\\
\cos\alpha_4
\end{bmatrix}
=
\begin{bmatrix}
-\ff{\sqrt2}{2}\\
\ff{\sqrt2}{2}\\
\ff{\sqrt2}{2}\\
-\ff{\sqrt2}{2}
\end{bmatrix}
\quad{\rm or}\quad
\begin{bmatrix}
\ff{\sqrt2}{2}\\
-\ff{\sqrt2}{2}\\
-\ff{\sqrt2}{2}\\
\ff{\sqrt2}{2}
\end{bmatrix}
\quad{\rm or}\quad
\begin{bmatrix}
\ff{\sqrt10}{10}\\
-\ff{3\sqrt10}{10}\\
\ff{3\sqrt10}{10}\\
-\ff{\sqrt10}{10}
\end{bmatrix}
\quad{\rm or}\quad
\begin{bmatrix}
-\ff{\sqrt10}{10}\\
\ff{3\sqrt10}{10}\\
-\ff{3\sqrt10}{10}\\
\ff{\sqrt10}{10}
\end{bmatrix}
\]
And we obtain from $\cos^2{\alpha}+\sin^2{\alpha}=1$ and $(*)$:
\[
\begin{bmatrix}
\sin\alpha_1\\
\sin\alpha_2\\
\sin\alpha_3\\
\sin\alpha_4
\end{bmatrix}
=
\begin{bmatrix}
\ff{\sqrt2}{2}\\
-\ff{\sqrt2}{2}\\
-\ff{\sqrt2}{2}\\
\ff{\sqrt2}{2}
\end{bmatrix}
\quad{\rm or}\quad
\begin{bmatrix}
-\ff{\sqrt2}{2}\\
\ff{\sqrt2}{2}\\
\ff{\sqrt2}{2}\\
-\ff{\sqrt2}{2}
\end{bmatrix}
\]
For the first or the second solution of $\cos\alpha$.\\
Now by the fact that $\sin{2\alpha}=2\sin{\alpha}\cos{\alpha}$ we know that the solution of the form
\[
\begin{bmatrix}
\sin2\alpha_1\\
\sin2\alpha_2\\
\sin2\alpha_3\\
\sin2\alpha_4
\end{bmatrix}
\]
has the same sign for $i=1,2,3,4$, and is impossible to satisfy $(*)$,
thereby there's no solution to the angles and a \sedb balance is impossible.\\
}
{\bf Exercise 5}\\
{\it Solutions}:
We set
\[
\begin{vmatrix}
1 & 3x \\ 
x+2 & 2x\\ 
\end{vmatrix}
\neq0
\]
Which yields $x\neq 0$ and $x\neq -\frac{4}{3}$.\\
[1em]
{\bf Exercise 6.}\\
{\it Proof}\\
(i) False\\
Take $a_1 = (1,0), a_2 = (0,1), a_3 = (1,1)$, which are mutually linearly independent, whereas $\{a_1,a_2,a_3\}$ is not independent since $a_1+a_2-a_3=0$.\\[0.5 em]
(ii) False\\
Consider system:
\begin{align*}
a_1=
\begin{bmatrix}
1\\
0\\
0
\end{bmatrix}
\quad
a_2=
\begin{bmatrix}
0\\
1\\
0
\end{bmatrix}
\quad
a_3=
\begin{bmatrix}
0\\
0\\
1
\end{bmatrix}
\quad
a_4=
\begin{bmatrix}
1\\
0\\
1
\end{bmatrix}
\quad
\end{align*}
Then it's evident that $a_2$ can't be represented by the linear combinations of other vectors.\\
[1em]
{\bf Exercise 7.}\\
(i)
If $x,y\in U$, then
$x+y=(x_1+y_1,x_2+y_2,x_3+y_3,x_4+y_4)$ with $f_1(x)=f_2(x)=f_3(x)=f_1(y)+f_2(y)=f_3(y)=0$
$f_1(x+y)=f_1(x)+f_1(y)=0,~f_2(x+y)=f_2(x)+f_2(y),~f_3(x+y)=f_3(x)+f_3(y)$\\
That is, $x+y\in U$\\
Because the addition is associative and commutative for $x,y \in R^4$, so it will also have these properties for $x,y \in U$.\\
The unit element is $e=(0,0,0,0)\in U$, and the inverse element to x is $-x=(-x_1,-x_2,-x_3,-x_4)\in U$ since $-f_1(x)=-f_2(x)=-f_3(x)=0$. Similarly , we can check that scalar multiplication is a map $R\times U\rightarrow U$ and satisfies all conditions to ensure that $(L,+,\cdot)$ is a vector space.\\
Since $U\in R^4$, $U$ is a subspace of $R^4$.\\[0.5 em]
(ii)
\begin{equation*}
\begin{cases}
f_1(x)=x_1+2x_2+x_3-x_4=0\\
f_2(x)=3x_1+5x_2-x_3-6x_4=0\\
f_3(x)=-2x_1-x_2+10x_3+11x_4=0
\end{cases}
\end{equation*}
\begin{align*}
&\qquad\quad \left.\begin{gmatrix}
		   1 & 2 & 1 & -1 \\
		   3 & 5 & -1 & -6 \\
           -2 & -1 & 10 & 11
\end{gmatrix}\right|\
\begin{gmatrix}
 0 \\
 0 \\
 0
\end{gmatrix}
& &\sim\quad
\left.\begin{gmatrix}
		   1 & 2 & 1 & -1 \\
		   0 & 1 & 4 & 3 \\
           0 & 0 & 0 & 0
\end{gmatrix}\right|\
\begin{gmatrix}
 0 \\
 0 \\
 0
\end{gmatrix}
\end{align*}
$(x_1,x_2,x_3,x_4)$ is the solutions of the equation set:
\begin{equation*}
\begin{cases}
x_1+2x_2+x_3-x_4=0\\
x_2+4x_3+3x_4=0
\end{cases}
\end{equation*}
Let $x_3=x_4=1$ or $x_3=-1$, $x_4=1$.\\
Then we get basis for $U$ is $(14,-7,1,1)$, and $(0,-1,1,-1)$.
\\[1em]
{\bf Exercise 8.}\\
(1)
$\langle a,a\rangle=2a_1^2+a_1a_2+a_2a_1+2a_2^2=2(a_1+\frac{a_2}{2})^2+\frac{3}{2}a_2^2\ge0$\\
(Self-positivity)
If $a=0$, then $a_1=a_2=0\Longrightarrow\langle a,a\rangle=0$\\
If $\langle a,a\rangle=0$, then $(a_1+\frac{a_2}{2})^2=a_2^2=0\Longrightarrow a_1=a_2=0\Longrightarrow a=0$\\
(Linearity in the second component) $\langle a,b+c\rangle=0=2a_1(b_1+c_1)+a_1(b_2+c_2)+a_2(b_1+c_1)+2a_2(b_2+c_2)=(2a_1b_1+a_1b_2+a_2b_1+2a_2b_2)+(2a_1c_1+a_1c_2+a_2c_1+2a_2c_2)=\langle a,b\rangle+\langle a,c\rangle$\\
$\langle a,\lambda b\rangle=2\lambda a_1b_1+\lambda a_1b_2+\lambda a_2b_1 +2\lambda a_2b_2=\lambda\langle a,b\rangle$\\
(Conjugate Symmetry)$\langle a,b\rangle=2a_1b_1+a_1b_2+a_2b_1 +2a_2b_2=\overline{2a_1b_1+a_1b_2+a_2b_1 +2a_2b_2}=\overline{\langle b,a\rangle}$\\
So it defines a inner product.\\[0.5 em]
(2)
$\langle a,a\rangle=a_1^2+a_1a_2+a_2a_1+a_2^2=(a_1+a_2)^2\ge0$\\
If $\langle a,a\rangle=0$, then $(a_1+a_2)^2=0\Longrightarrow a_1=-a_2\not\Longrightarrow a=0$ $($for example $a_1=1,a_2=-1)$\\
So it does NOT define a inner products.\\[0.5 em]
(3)
$ \langle a,a\rangle=a_1^3+a_2^3\not\Longrightarrow a_1^3+a_2^3\ge0$ (for~example~$a_1=-1$,$a_2=-1$)\\
So it does NOT define a inner products.\\[0.5 em]
(4)
$\langle a,a\rangle=a_1^2-a_2^2\not\Longrightarrow a_1^2-a_2^2\ge0$ (for~example~$a_1=0$,$a_2=1$)\\
So it does NOT define a inner products.

\end{document}