\documentclass[12pt]{article}
\usepackage{geometry}
\usepackage{amsmath} 
\usepackage{amsthm}
\usepackage{amsfonts}
\usepackage{cases}
\usepackage{graphicx}
\usepackage{caption}
\usepackage{subcaption}
\usepackage{gauss}
\usepackage{tikz}
\usetikzlibrary{positioning}
\usetikzlibrary{matrix}
\usepackage{tkz-graph}
\usetikzlibrary{arrows}
% set arrows as stealth fighter jets
\tikzset{>=stealth}



\def\dotp#1#2{\langle#1,#2\rangle}
\def\ss#1#2{\sum_{#1=1}^{#2}}
\def\lam{\lambda}
\def\vec#1#2{\{#1_1,#1_2\ldots#1_{#2}\}}

\def\es#1#2{{\bf Exercise #1}\\{\it Solution:}\begin{itemize}
 \setlength{\itemsep}{2pt}
 \setlength{\parskip}{0pt}
 \setlength{\parsep}{0pt}
 #2
 \end{itemize}\vskip 1 em}
 
 \def\ess#1#2{{\bf Exercise #1}\\{\it Solution:}
 #2
\vskip 1 em}
 
\def\ep#1#2{{\bf Exercise #1}\\{\it Proof:}\begin{itemize}
 \setlength{\itemsep}{2pt}
 \setlength{\parskip}{0pt}
 \setlength{\parsep}{0pt}
 #2
 \end{itemize}\vskip 1 em}
 
  \def\eps#1#2{{\bf Exercise #1}\\{\it Solution:}
 #2
\vskip 1 em}

\def\sgn{{\rm sgn}}
\def\mat#1{{\rm Mat}(#1\times #1,\mathbb{R})}
\def\GL{{\rm GL}(n,\mathbb{R}) }
\def\SL{SL$(n,\mathbb{R})$ }
\def\O{O$(n,\mathbb{R})$}
\def\SO{SO$(n,\mathbb{R})$}
\def\Sp{Sp$(n,\mathbb{R})$ }

\def\ff#1#2{\frac{#1}{#2}}
\def\cgu#1{\overline{#1}}
\def\ran#1{{\rm ran}\,#1}
\def\ker#1{{\rm ker}\,#1}
\def\tr#1{{\rm tr}\left(#1\right)}
\def\dotr#1#2{\dotp{#1}{#2}_{\rm tr}}
\def\lr#1#2#3{\left#1#3\right#2}
\newcommand{\R}{\mathbb{R}}
\newcommand{\bb}[1]{\mathbb{#1}}
\newcommand{\eq}[1]{\begin{align*}#1\end{align*}}
\newcommand{\mm}[1]{\begin{pmatrix}#1\end{pmatrix}}
\def\sg{\sigma}
\def\ct{\cos{\theta}}
\def\st{\sin{\theta}}
\def\sq#1{\sqrt{#1}}
\newcommand{\B}{\mathcal{B}}

\linespread{1.4}
\geometry{a4paper,centering,scale=0.8}
\rmfamily 
\normalsize
\setlength{\parindent}{0em}

\begin{document} 
\begin{flushleft}
  Junlong Gao 5133709126\\ 
  Prof.  Hohberger\\ 
  VV285 Assignment 5\\
  \today 
\end{flushleft}
\es{1}{
 \item[(i)]
	Notice that $f(x)=1/x$ for $x\in[1,+\infty)$,  $\ran f=(0,1]$. 
	Clearly the domian $A=[1,+\infty)$ is closed yet the range is not.
 \item[(ii)]
	Notice that $f(x)=\sin x$ for $x\in(0,2\pi)$ 
	then $\ran f=[-1,1]$ the open set is mapped into a closed set.
 \item[(iii)]	
	No. Again, consider $f(x)=\sin x$. Then for $K=[-1,1]$, 
	the pre-image $f^{-1}(K)=\mathbb{R}$ clearly is unbounded, 
	therefore is not compact.
}
\es{2}{
 \item[(i)]
 	$A_1$ is neither open nor closed\\
	$A_2$ is  neither open nor closed\\
	$A_3$ is open.
 \item[(ii)]
 	$\overline{A_1}=A_1\cup\{(x,y):x=0,y\ge0\}$\\
	$\overline{A_2}=\{(x,y):1\leq x\leq 2, 0\leq y\leq 1/x\}$\\
	$\overline{A_3}=\{(x,y):1\leq x\leq 2, 0\leq |y|\leq 1/x\}$
 \item[(iii)]	
 	${\rm int} A_1=\{(x,y): x>0, 0< y< 1/x\}$\\
	${\rm ext} A_1=\{(x,y): x<0
	\quad{\rm or }\quad x>0, y<0 
	\quad{\rm or }\quad x>0,y>1/x\}$\\
	$\partial A_1=\{(x,y): x>0, y=1/x
	\quad {\rm or}\quad y=0, x>0
	\quad {\rm or}\quad x=0, y\ge0\}$\\
	${\rm int} A_2=\{(x,y):1< x< 2, 0< y<1/x\}$\\
	${\rm ext} A_2=\{(x,y):x< 1
	\quad {\rm or}\quad 1<x<2, y>1/x
	\quad {\rm or}\quad 1<x<2, y<0
	\quad{\rm or}\quad x>2\}$
	$\partial A_2=\{(x,y): y=0,1\leq x\leq 2
	\quad{\rm or}\quad x=2, 0\leq y\leq 1/2
	\quad{\rm or}\quad 1\leq x\leq 2, y=1/x
	\quad {\rm or}\quad x=1,0\leq y\leq 1 \}$\\
	${\rm int} A_3=A_3=\{(x,y):1< x< 2, 0< |y|< 1/x\}$\\
	${\rm ext} A_3=\{(x,y):x< 1
	\quad {\rm or}\quad 1<x<2, |y|>1/x
	\quad{\rm or}\quad x>2\}$\\
	$\partial A_3=\{(x,y): y=0,1\leq x\leq 2
	\quad{\rm or}\quad x=2, 0\leq |y|\leq 1/2
	\quad{\rm or}\quad 1\leq x\leq 2, |y|=1/x
	\quad {\rm or}\quad x=1,0\leq |y|\leq 1 \}$
\item[(iv)]	
	For $A_1:\{(x,y): x>0, y=1/x
	\quad {\rm or}\quad y=0, x>0\}$\\
	For $A_2:\{(x,y):x=1, 0<y<1\quad{\rm or}\quad x=2, 0<y<1/2\}$\\
	For $A_3: \emptyset$
}
\ep{3}{
\item[(i)]
$(\Longrightarrow)$ If $x\in\overline{M}$, then we assume $x\in \partial M$,
otherwise the proof is done. Then since $B_{1/n}(x)\cap M$ is none empty by the definition of boundary for $n=1,2\ldots$, we may construct a sequence $z_n\in M$ such that $z_n\to x$ and finally $\inf_{\forall y\in M}\|x-y\|\leq\|x-z_n\|$ for all $n=1,2\ldots$.
Take the limit $n\to\infty$ then $d(x,M)\to0$.\\
$(\Longleftarrow)$  If $d(x,M)=0$ then by definition $\forall \epsilon>0, \exists y\in M$
 such that $\|x-y\|<\epsilon$, thus $x$ can't be an exterior point of $M$, and thus
 $x\in \overline{M}$.
 \item[(ii)]
 By triangle inequality we have:
 \begin{equation}
 \|x-y\|+\|x-z\|\ge\|z-y\|
 \end{equation}
 for all $y\in M$
 then for any $x,z$ fixed, we have
 \begin{equation}
\inf_{y\in M}\|x-y\|+\|x-z\|\ge\inf_{y\in M}\|z-y\|
\Longrightarrow\quad d(z,M)-d(x,M)\leq\|x-z\|
 \end{equation}
 Since (1) is symmtric in $x,z$ we may exchange them to get
  \begin{equation}
d(x,M)-d(z,M)\leq\|z-x\|
 \end{equation}
 Then finally we combine (2) and (3) to get
 $$
 |d(x,M)-d(z,M)|\leq\|x-z\|
 $$
 Then for $\forall \epsilon>0$ we have $\delta=\epsilon/2$ such that if $\|x-z\|<\delta$ then
 $$
  |d(x,M)-d(z,M)|\leq\|x-z\|<\delta=\epsilon/2<\epsilon
 $$
 thus the continouity is established.
 \item[(iii)]
 Consider the closed ball $B$ around $x$ such that $B\cap M$ is none empty, 
 then $B_r(x)\cap M$ is closed and bounded, therefore compact. And the mapping $y\mapsto \|x-y\|$ is continous as a special case of (ii) $\|x-y\|=d(x,\{y\})$, and assumes its minimum $d(x,M)$ at, say, $y_0$, locally. Now it remains to be shown that this minimum is global:
 for $\exists y_0'\in M$ such that $\|x-y_0'\|<\|x-y_0\|$ then $y_0'$ must be in $B_r(x)$, and $y_0'$ is the new minimum point in $B_r(x)\cap M$, contracdicts to our $y_0$.
Therefore $y_0$ is the global minimum and is $d(x,M)$ by definition.
\item[(iv)]
Since the mapping $x\mapsto d(x,M)$ is continous on a compact set, we let $x_0\in K$
 denote the point at which it assumes minimum. We now have
 $$
 d(K,M)=\inf_{x\in K,y\in M}\|x-y\|=\inf_{y\in M}\|x_0-y\|=d(x_0,M)
 $$
and by (iii) we know there exists $y_0\in M$ such that $d(x_0,M)=\|x_0-y_0\|$. Combining the two gives us $ d(K,M)=\|x_0-y_0\|$ for some $x_0\in K, y_0\in M$.\\
Now, $K$ is compact is essential to the proof. Consider $K=\{(x,y):y=1/x,x>0\}$, $M=x$-axis. Then $d(K,M)=0$ yet it's impossible to find such a pair $x_0\in K,y_0\in M$ such that $\|x_0-y_0\|=0$.
}
\ep{4}{
\item[(i)] The second holds. Sincewe have 
$$
\|f\|_1=\int_0^1|f(x)|dx\leq1\cdot|f(x)|=\|f\|_{\infty}
$$
where $C=1$.
\item[(ii)] Consider the sequence of function
$$
f_n(x)=x^n,\quad x\in [0,1],\quad n=1,2,\cdots.
$$
we have that 
$$
\|f_n\|_{\infty}=1\to 1
$$
yet 
$$
\|f_n\|_{1}=\frac{1}{n+1}\to 0
$$
}
\ep{5}{
\item[]
It's bounded by definition that $\|(a_n)\|_1=\sum_{n=0}^{\infty}|a_n|\leq1$. The closeness follows from the fact that it's complement is open: for 
$\overline{B_1(0)}^{c}=\{(a_n):1<\|(a_n)\|<\infty\}$, because given
 $(a_n)\in\overline{B_1(0)}^{c}$, we set $r<\|(a_n)\|-1$ then
 $$
 \|(b_n)-(a_n)\|<r\Longrightarrow \|(b_n)\|\ge\|(a_n)\|-\|(b_n)-(a_n)\|>1+r-r=1
 $$
 from triangle inequality, thus each point of $\overline{B_1(0)}^{c}=\{(a_n):1<\|(a_n)\|<\infty\}$ is an interior point, $\overline{B_1(0)}^{c}$ is open and $\overline{B_1(0)}$
 is closed.\\
 Now consider a sequence of unit sequence $e_n=(0,0\ldots0,1,0\ldots)$, where the $1$ is at the $n$th position, then we have 
 $$
 \|e_n\|=\sum_{k=0}^{\infty}(e_k)_n=1
 $$
 thus $e_n\in\overline{B_1(0)}$. Yet we have
 $$
 \|e_n-e_m\|\ge 1\quad m\neq n
 $$
thus $e_n$ can't be a Cauchy sequence, nor its subsequence, so it can't have a convergent subsequence, and then $e_n$ must not converge since every convergent sequence must be a Cauchy sequence.
}
\ep{6}{
\item[(i)] We define norm
$$
\|H\|=\max_{i,j}\{h_{ij}\}, \quad H\in {\rm Mat}(n\times n)
$$
then for any sequence of $A_n\in {\rm Mat}(n\times n)\to A$, each entry must converges to $a_{ij}^{(n)}\to a_{ij}$, and we have
$$
\det(A_n)= \sum_{\pi\in S_n}\sgn\pi a_{1\pi(1)}^{(n)}a_{2\pi(2)}^{(n)}\cdots a_{n\pi(n)}^{(n)}
$$
by Leibniz formula. Notice that it's a finite sum of finite product, thus the limit process can
be take pointwisely:
\eq{
\lim_{n\to\infty}\det(A_n)
&=\sum_{\pi\in S_n} \sgn\pi\lim_{n\to\infty}a_{1\pi(1)}^{(n)}a_{2\pi(2)}^{(n)}\cdots a_{n\pi(n)}^{(n)}\\
&=\sum_{\pi\in S_n} \sgn\pi a_{1\pi(1)}a_{2\pi(2)}\cdots a_{n\pi(n)}\\
&=\det(A)
}
This estabilishes the continuity of determinant function.
\item[(ii)]
\eq{
\det(I+\epsilon A)&=\sum_{\pi\in S_n}\sgn\pi(\delta_{1\pi(1)}+\epsilon a_{1\pi(1)})
(\delta_{2\pi(2)}+\epsilon a_{2\pi(2)})\cdots(\delta_{n\pi(n)}+\epsilon a_{n\pi(n)})\\
&=(1+\epsilon a_{1\pi(1)})
(1+\epsilon a_{2\pi(2)})\cdots(1+\epsilon a_{n\pi(n)})
\\
&+\sum_{\pi\in S_n,\pi\neq i.d}\sgn\pi(\delta_{1\pi(1)}+\epsilon a_{1\pi(1)})
(\delta_{2\pi(2)}+\epsilon a_{2\pi(2)})\cdots(\delta_{n\pi(n)}+\epsilon a_{n\pi(n)})
}
Now, the first term will produce:
$$
1+\epsilon(a_{11}+a_{22}+\cdots +a_{nn})+o(\epsilon^2)=1+\epsilon\cdot\tr A+o(\epsilon^2)
$$
The second term, however, will have at least $\epsilon^2$ since a permutation must 
have at least two index out of order:
$$
(\delta_{i\pi(i)}+\epsilon a_{i\pi(i)})(\delta_{j\pi(j)}+\epsilon a_{j\pi(j)})
=o(\epsilon^2)
$$
combining the two we conclude that
$$
\det(I+\epsilon A)=1+\epsilon\cdot\tr A+o(\epsilon^2)+o(\epsilon^2)=1+\epsilon\cdot\tr A+o(\epsilon^2)
$$
then we are done.
\item[(iii)] 
Since $A$ is invertable, $\forall H\in\mat{n}$ we have
\eq{
&\det(A+\epsilon H)=\det(A)\det(I+\epsilon A^{-1}H)
=\det(A)+\epsilon\cdot\det(A)\tr {A^{-1}H}+o(\epsilon^2)\\
&\Longrightarrow\quad \det(A+\epsilon H)-\det(A)=\epsilon\cdot\det(A)\tr {A^{-1}H}+o(\epsilon^2)\\
&\quad\quad=\det(A)\tr {A^{-1}\epsilon H}+o(\|\epsilon H\|^2)
}
Now the first term is linear in $\epsilon H$ since we have
$$
\tr{A^{-1}(\epsilon H_1+\epsilon H_2)}=\tr{A^{-1}\epsilon H_1+A^{-1}\epsilon H_2}
=\tr{A^{-1}\epsilon H_1}+\tr{A^{-1}\epsilon H_2}
$$
by the definition of trace. \\
The second term is the higher orders of $\epsilon H$ for any given $H$. Thus we conclude that under the norm of (i) we have
$$
\det(A+H)-\det(A)=\det(A)\tr {A^{-1} H}+o(\|H\|^2)
$$
Now it's safe to invoke the definition of differential:
$$
\left.(D \det)\right|_AH=\det(A)\tr {A^{-1} H}
$$
\item[(iv)] 
By Cramer's rule we have
\begin{equation}
\det(A)\tr {A^{-1} H}=\tr {\det(A)A^{-1} H}=\tr {A^{\sharp} H}
\end{equation}
Now we need to show that this definition is invariant under limit process, we first prove that each non-invertible matrix is a sequence of some invertible matrix:\\
Let $A\in\GL$ to be non-invertible, consider :
$$
\det(A+tI)
$$ 
is a polynomial in $t$ of degree at most $n$, and have at most $n$ real zero points. If we set the minimal $k\in\mathbb{N}$ such that $0<1/k<|\alpha|$ where $\alpha$ is the zero point with the smallest absolute value, we know that $\det(A+t_nI)$ will not vanish for $t_n=1/(k+n)$, $n\in\mathbb{N}$ and we choose our invertable sequence to be $A_n=A+t_nI$
, therefore each non-invertible
 matirx is a limit of some sequence in $ \GL$.\\
 Now back to (4), we know that the mappings
 $$
 \det A, \tr{A^{\sharp}}
 $$
 are continous since $\det$ function is continous and trace function is continous. Thus 
 for non-invertible matrix $A$, we pick a sequence of inverible function $A_n$ and 
 define 
 $$
\left.(D \det)\right|_AH:=\lim_{n\to\infty}\tr {A_n^{\sharp} H}
 $$
 since the continous function's value is invariant under limit process, this gives us a
 consistent analytic continuation on $\mat{n}$.
}
\end{document}