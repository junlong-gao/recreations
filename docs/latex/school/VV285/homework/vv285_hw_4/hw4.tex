\documentclass[12pt]{article}
\usepackage{geometry}
\usepackage{amsmath} 
\usepackage{amsthm}
\usepackage{amsfonts}
\usepackage{cases}
\usepackage{graphicx}
\usepackage{caption}
\usepackage{subcaption}
\usepackage{gauss}
\usepackage{tikz}
\usetikzlibrary{positioning}
\usetikzlibrary{matrix}
\usepackage{tkz-graph}
\usetikzlibrary{arrows}
% set arrows as stealth fighter jets
\tikzset{>=stealth}
\usepackage{bbold}


\def\dotp#1#2{\langle#1,#2\rangle}
\def\ss#1#2{\sum_{#1=1}^{#2}}
\def\lam{\lambda}
\def\vec#1#2{\{#1_1,#1_2\ldots#1_{#2}\}}
\def\es#1#2{{\bf Exercise #1}\\~{\it Solution:}\\~#2\\[1em]}
\def\ep#1#2{{\bf Exercise #1}\\~{\it Proof:}\\~#2\\[1em]}
\def\inn#1#2{(#1): #2\\[0.5em]}
\def\ff#1#2{\frac{#1}{#2}}
\def\cgu#1{\overline{#1}}
\def\ran#1{{\rm ran}\,#1}
\def\ker#1{{\rm ker}\,#1}
\def\tr#1{{\rm tr}\left(#1\right)}
\def\dotr#1#2{\dotp{#1}{#2}_{\rm tr}}
\def\lr#1#2#3{\left#1#3\right#2}
\newcommand{\R}{\mathbb{R}}
\newcommand{\bb}[1]{\mathbb{#1}}
\newcommand{\eq}[1]{\begin{align*}#1\end{align*}}
\newcommand{\mm}[1]{\begin{pmatrix}#1\end{pmatrix}}
\def\sg{\sigma}
\def\ct{\cos{\theta}}
\def\st{\sin{\theta}}
\def\sq#1{\sqrt{#1}}
\newcommand{\B}{\mathcal{B}}

\linespread{1.4}
\geometry{a4paper,centering,scale=0.8}
\rmfamily 
\normalsize
\setlength{\parindent}{0em}

\begin{document} 
\begin{flushleft}
  Junlong Gao 5133709126\\ 
  Prof.  Hohberger\\ 
  VV285 Assignment 4\\
  \today 
\end{flushleft}

\es{1}{
	\inn{i}{
	Notice the (2) in question implies that
	\eq{
	&u_1=z_{11}i_1+z_{12}i_2\\
	&u_2=z_{21}i_1+z_{22}i_2
	}
	Setting $i_1=0$ gives that
	\[
	\ff{u_1}{i_2}=z_{12}
	\]
	Setting $i_2=0$ gives that
	\[
	\ff{u_2}{i_1}=z_{21}
	\]
	}
	\inn{ii}{
	By relation that 
	\eq{
	u_1=A_{11}u_2+A_{12}i_2\\
	i_1=A_{21}u_2+A_{22}i_2
	}
	We may solve for the transformation:
	\[
	\mm{z_{11}&z_{12}\\z_{21}&z_{22}}=\mm{\ff{A_{11}}{A_{21}}&A_{12}-\ff{A_{11}A_{22}}{A_{21}}\\\ff{1}{A_{21}}&-\ff{A_{22}}{A_{21}}}
	\]
	Thus $\det A=1$ implies that 
	\[
	z_{12}=A_{12}-\ff{A_{11}A_{22}}{A_{21}}=\ff{A_{12}A_{21}-A_{11}A_{22}}{A_{21}}=\ff{-1}{A_{21}}=-z_{21}	
	\] 
	On the other hand, provided that $z_{12}=-z_{21}$ we have
	\eq{
	A_{12}-\ff{A_{11}A_{22}}{A_{21}}=-\ff{1}{A_{21}}\iff A_{12}A_{21}-A_{11}A_{22}=1 \iff \det A=1
	}
	\inn{iii}{
	By Kirchhoff's laws we have currents and voltage equations:
	\eq{
	i_1=i_3+\frac{u_1}{R_1}\\
	i_2=i_3-\frac{u_2}{R_2}\\
	i_3R_3+u_2=u_1\\
	}
	By eliminating the $i_3$ we have:
	\eq{
	i_1=\frac{u_1-u_2}{R_3}+\frac{u_1}{R_1}\\
	i_2=\frac{u_1-u_2}{R_3}-\frac{u_2}{R_2}\\
	}
	Solving $u_1$, $i_1$ in terms of $u_2$ and $i_2$ gives 
	\eq{
	&u_1=\frac{R_2+R_3}{R_2}u_2+R_3i_2\\
	&i_1=\frac{R_1+R_2+R_3}{R_1R_2}u_2+\frac{R_1+R_3}{R_1}i_2
	}
	That is , the matrix $A$.
	}
	\inn{iv}{
	Set $R_1=R_2=R$, we read off the entries of A:
	\[
	\ff{R_3R^2}{2R+R_3}=\ff{Z\sinh g}{\ff{1}{Z}\sinh g}\Longrightarrow Z=R\sq{\ff{R_3}{2R+R_3}}\quad(Z>0)
	\]
	Then by the first entry of $A$:
	\[
	\frac{R+R_3}{R}=\cosh g=\frac{e^g+e^{-g}}{2}
	\]
	Gives $g=\ln\left(1+\frac{R_3}{R}+\sqrt{\left(1+\frac{R_3}{R}\right)^2-1}\right)$ since $g\ge0$.
	}
	\inn{v}{
	Notice that (6) is equivalent to
	\eq{
	&A\mm{u_2\\i_2}-\lam\mm{1&0\\0&1}\mm{u_2\\i_2}=0\\
	&\iff \left(A-\mm{\lam&0\\0&\lam}\right)\mm{u_2\\i_2}=0
	}
	for non-trivial $\mm{u_2\\i_2}$.\\[0.7 em]
	Thus, by the fundamental theorem of linear equation, we see that this happens if and only if the matrix
	\[
	A-\mm{\lam&0\\0&\lam}
	\]
	has a non-trivial kernel, that is, NOT invertible, which is equivalent to
	\[
	\det\left(A-\mm{\lam&0\\0&\lam}\right)=0.
	\]
	}
	\inn{vi}{
	\eq{
	\det\left(A-\mm{\lam&0\\0&\lam}\right)=(\cosh g-\lambda)^2-\sinh^2 g=0
	}
	Use the identity 
	\[
	\cosh^2g-\sinh^2g=1
	\]
	We have 
	\[
	\lam^2-2\lam\cosh g+1=0
	\]
	giving us a pair of conjugate solutions:
	\[
	\lam_1=\cosh g+\sinh g;\quad\lam_2=\cosh g-\sinh g
	\]
	}
	\inn{vii}{
	By (6) we have
	\eq{
	&\cosh g \frac{u_2}{i_2}+Z\sinh g =\lambda \frac{u_2}{i_2}\\
	&\frac{u_2}{i_2}\frac{1}{Z}\sinh g+\cosh g=\lambda
	}
	which gives the same solution (as it should be), and if we plug in $\lambda_1$ or $\lambda_2$ above:
	\[
	\frac{u_1}{i_1}=\frac{u_2}{i_2}=\frac{Z\sinh g}{\lambda-\cosh g}=\pm Z
	\]
	}
	Let's take the positive solution and plug in the numbers:
	\[
	\frac{u_1}{i_1}=Z=R\sq{\ff{R_3}{2R+R_3}}=\sqrt{\frac{20}{21}}
	\]
	\inn{viii}{
	Using the identity that:
	\[
	\cosh^2g+\sinh^2g=\cosh2g;\qquad2\sinh g\cosh g=\sinh2g
	\]
	By induction we arrive at:
	\[
	A^n=\mm{\cosh ng&Z\sinh ng\\ \ff{1}{Z}\sinh ng&\cosh ng}
	\]
	and the explicit form of the quotient of voltage and current:
	\[
	\ff{u_1}{i_1}=\ff{u_2+Zi_2\tanh ng}{\ff{1}{Z}u_2\tanh ng +i_2}
	\]
	and since
	\[
	\tanh ng=\ff{1-e^{-2ng}}{1+e^{-2ng}}\to 1\quad \hbox{\rm as}\, n\to \infty
	\]
	The quotient goes to:
	\[
	\lim_{n\to\infty}\ff{u_1(n)}{i_1(n)}=Z \ff{\ff{1}{Z}u_2+i_2}{\ff{1}{Z}u_2 +i_2}=Z
	\]
	}
	}
}
\es{2}{
Expanding by row we have:
\[
\det A=\frac{1}{43200},\qquad\det B=24
\]
}
\es{3}{
	\eq{
	&\det\begin{pmatrix}
	2&0&0&0&0&1\\
	1&2&0&0&0&1\\
	3&1&2&0&0&1\\
	0&3&1&2&0&1\\
	-5&0&3&1&2&1\\
	1&1&1&1&1&1
	\end{pmatrix}
	=
	32\det\begin{pmatrix}
	1    &0    &0   &0   &0&1\\
	1/2 &1    &0   &0   &0&1\\
	3/2 &1/2 &1   &0   &0&1\\
	0    &3/2    &1/2&1   &0&1\\
	-5/2&0    &3/2   &1/2&1&1\\
	1/2 &1/2 &1/2   &1/2&1/2&1
	\end{pmatrix}\\
	&
	=16\det\begin{pmatrix}
	1    &0    &0   &0   &0&1\\
	1/2 &1    &0   &0   &0&1\\
	3/2 &1/2 &1   &0   &0&1\\
	0    &3/2    &1/2&1   &0&1\\
	-5/2&0    &3/2   &1/2&1&1\\
	1 &1 &1   &1 &1&2
	\end{pmatrix}\\
	&=16\det\begin{pmatrix}
	1    &0    &0   &0   &0&0\\
	1/2 &1    &0   &0   &0&0\\
	3/2 &1/2 &1   &0   &0&0\\
	0    &3/2    &1/2&1   &0&0\\
	-5/2&0    &3/2   &1/2&1&0\\
	1 &1 &1   &1 &1&-59/16
	\end{pmatrix}\\
	&=16\times\ff{-59}{16}=-59
	}
The last line follows that by expansion only the terms on diagonal survive.
}
\ep{4}{
(i)$\Longrightarrow$(ii):
Notice, that
\eq{
0&=f(a_1,\ldots,a_{j-1},a_j+a_k,a_{j+1},\ldots,a_k+a_j,a_{k+1},\ldots,a_p)\\
&\quad=f(a_1,\ldots,a_{j-1},a_j,a_{j+1},\ldots,a_k,a_{k+1},\ldots,a_p)\\
&\qquad+f(a_1,\ldots,a_{j-1},a_j,a_{j+1},\ldots,a_j,a_{k+1},\ldots,a_p)\\
&\qquad+f(a_1,\ldots,a_{j-1},a_k,a_{j+1},\ldots,a_k,a_{k+1},\ldots,a_p)\\
&\qquad+f(a_1,\ldots,a_{j-1},a_k,a_{j+1},\ldots,a_j,a_{k+1},\ldots,a_p)\\
&\qquad=f(a_1,\ldots,a_{j-1},a_j,a_{j+1},\ldots,a_k,a_{k+1},\ldots,a_p)\\
&\quad\qquad+f(a_1,\ldots,a_{j-1},a_k,a_{j+1},\ldots,a_j,a_{k+1},\ldots,a_p)
}
Thus 
\eq{
f(a_1,\ldots,a_{j-1},a_j,a_{j+1},\ldots,a_k,a_{k+1},\ldots,a_p)=-f(a_1,\ldots,a_{j-1},a_k,a_{j+1},\ldots,a_j,a_{k+1},\ldots,a_p)
}
(ii)$\Longrightarrow$(iii):
Without the loss of generality, say $a_1=\sum_{i\neq1}^{n}\lam_ia_i$
Then 
\[
f(a_1,\ldots,a_p)=\lam_2f(a_2,\ldots,a_p)+\lam_3f(a_3,\ldots,a_p)+\cdots+\lam_nf(a_n,\cdots,a_p)
\]
And each term in the right has the form by (ii)
\[
f(a_i,\cdots,a_i,\cdots,a_p)=-f(a_i,\cdots,a_i,\cdots,a_p)\Longrightarrow f(a_i,\cdots,a_i,\cdots,a_p)=0
\]
Thereby $f(a_1,\ldots,a_p)=0$\\
(iii)$\Longrightarrow$(i): If $a_i=a_j$ for some $i\neq j$, then the collection $\{a_1,\cdots,a_n\}$ is linear dependent, thus $f(a_1,\ldots,a_p)=0$\\
This completes the proof.
}
\ep{5}{
\def\mat{Mat$(n\times n,\mathbb{R})$}
\def\GL{GL$(n,\mathbb{R})$ }
\def\SL{SL$(n,\mathbb{R})$ }
\def\O{O$(n,\mathbb{R})$}
\def\SO{SO$(n,\mathbb{R})$}
\def\Sp{Sp$(n,\mathbb{R})$ }
\def\1{\mathbb{1}}
Consider $\forall A,B,C\in$ \GL, we have:\\
a) ${\left(AB\right)}^{-1}=B^{-1}A^{-1}$ is invertible, thus it's closed under multiplication and inverse ( since the inverse of a matrix is invertible), and\\
b) $(AB)C=A(BC)$ guaranteed by the association law of matrix multiplication.
Thereby \GL forms a group.\\
In order to verify a subgroup, one only have to show that it's closed under multiplication and inverse since the association law is guaranteed by the whole set:\\
i): Consider $\forall A,B\in$ \SL, we have:\\
 a) $\det{AB}=\det{A}\det{B}=1$, thus $AB\in$ \SL.\\
 b) Since $A\in $\SL$\subset$\GL, we note that $A^{-1}$ exists and $\det{A^{-1}}=\det{A}=1$, thus $A^{-1}\in$ \SL.\\
[0.5em]
A small lemma:
\[
(A^{-1})^{T}=(A^T)^{-1}
\]
follows by the fact that
\[
(A^{-1})^{T}A^T=(AA^{-1})^T=I^T=I
\]
\\[0.5 em]
ii): Consider $\forall A,B\in$ \O, we have:\\
 a) ${(AB)}^{T}=B^TA^T=B^{-1}A^{-1}={(AB)}^{-1}$ (by the above lemma), thus $AB\in$ \O.\\
 b) Since $A\in $\O$\subset$\GL, we note that $A^{-1}$ exists, and ${(A^{-1})}^{T}={(A^{T})}^{-1}={(A^{-1})}^{-1}$, thus $A^{-1}\in$ \O.\\
[0.5em] iii): Consider $\forall A,B\in$ \SO, we have:\\
 a) $\det{AB}=\det{A}\det{B}=1$, and ${(AB)}^{T}=B^TA^T=B^{-1}A^{-1}={(AB)}^{-1}$, thus $AB\in$ \SO.\\
 b) Since $A\in $\SO$\subset$\GL, we note that $A^{-1}$ exists, and $\det{A^{-1}}=\det{A}=1$, and ${(A^{-1})}^{T}={(A^{T})}^{-1}={(A^{-1})}^{-1}$ thus $A^{-1}\in$ \SO.\\
[0.5em] iv): Consider $\forall A,B\in$ \Sp, we have:\\
 a) ${(AB)}^T=B^{T}A^{T}=$
 \eq{
 &\begin{pmatrix}0&\1\\ -\1&0\end{pmatrix}B^{-1}\begin{pmatrix}0&-\1\\ \1&0\end{pmatrix}\begin{pmatrix}0&\1\\ -\1&0\end{pmatrix}A^{-1}\begin{pmatrix}0&-\1\\ \1&0\end{pmatrix}
 \\
 =&\begin{pmatrix}0&\1\\ -\1&0\end{pmatrix}{(AB)}^{-1}\begin{pmatrix}0&-\1\\\1&0\end{pmatrix}
 }
 thus $AB\in$ \Sp.\\ 
 b) Since $A\in $\Sp$\subset$\GL, we note that $A^{-1}$ exists and 
 \[
 {\left(A^{-1}\right)}^{T}={\left(A^{T}\right)}^{-1}={\begin{pmatrix}0&-\1\\\1&0\end{pmatrix}}^{-1}\left(A^{-1}\right)^{-1}{\begin{pmatrix}0&\1\\ -\1&0\end{pmatrix}}^{-1}=\begin{pmatrix}0&\1\\ -\1&0\end{pmatrix}{\left(A^{-1}\right)}^{-1}\begin{pmatrix}0&-\1\\\1&0\end{pmatrix}
 \]
thus $A^{-1}\in$ \Sp.\\[0.5 em]
 \inn{i}{
 	Given $\forall a,b\in\mathbb{R}^2$, the spanned parallelogram is 
	$P=\begin{pmatrix}a&b\end{pmatrix}$
	with the area $|\det P\,|$.\\
	Then the transformed parallelogram is $AP=\begin{pmatrix}Aa&Ab\end{pmatrix}$
	with the area $|\det AP\,|=|\det A\,||\det P\,|=|\det P\,|$.
 }
 \inn{ii}{
 	A similar argument can be made that given $\forall a,b,c\in\mathbb{R}^3$, the spanned parallepiped is
	$P=\begin{pmatrix}a&b&c\end{pmatrix}$
	with the volume $|\det P\,|$.\\
	Then the transformed parallelogram is $AP=\begin{pmatrix}Aa&Ab&Ac\end{pmatrix}$
	with the volume $|\det AP\,|=|\det A\,||\det P\,|=|\det P\,|$.
 }
 \inn{iii}{
 	Notice that $AA^{T}=AA^{-1}=\mathbb{I}$ implies that the columns of $A$ is an orthonormal system (see exercise 6). Since $\forall x\in\mathbb{R}^n$, $Ax$ is merely a linear combination of this orthonormal system with $x_i$ as its coefficients, the norm $\|Ax\|^2=\dotp{Ax}{Ax}=\ss{i}{n}x_i^2=\|x\|^2$, therefore we have
	\[
	\alpha\,(Ax,Ay)=\frac{\dotp{Ax}{Ay}}{\|Ax\|\|Ay\|}=\frac{\dotp{A^TAx}{y}}{\|x\|\|y\|}=\frac{\dotp{x}{y}}{{\|x\|\|y\|}}=\alpha\,(x,y)
	\]
 }
 }
 \ep{6}{
 	\inn{i}{
	(a)$\iff$(b): Let $A_j$ denote the $j$th column of the matrix $A\in$ \O, 
	then the fact that $A^{-1}=A^{T}$ leads to:
	\eq{
	&A^{T}A=A^{-1}A=I\\
	&\iff A_jA_i=\delta_{ij}\\
	}
	by the definition of matrix multiplication.\\
	The last line is equivalent to that the column vectors of A are an orthonormal basis of $\mathbb{R}^n$
	(a)$\iff$(c):
	 Let $A_i$ denote the $i$th row of the matrix $A\in$ \O, 
	then the fact that $A^{-1}=A^{T}$ leads to:
	\eq{
	&AA^{T}=AA^{-1}=I\\
	&\iff A_jA_i=\delta_{ij}\\
	}
	by the definition of matrix multiplication.\\
	The last line is equivalent to that the row vectors of A are an orthonormal basis of $\mathbb{R}^n$\\
	This completes the proof.
	}
	\inn{ii}{
	By (i) it's suffice to verify the columns are an orthonormal basis of $\mathbb{R}^n$. 
	Let $A_j$ denote the $j$th column of the matrix, it takes on the form that 
	\[
	A_j=\mm{
	\ff{1}{\sq{n}}\\
	0\\
	\vdots\\
	-\ff{j-1}{\sq{j(j-1)}}\\
	\ff{1}{\sq{(j+1)j}}\\
	\vdots\\
	\ff{1}{\sq{n(n-1)}}
	}
	\]
	The second none zero entry appears at the $j$th row, unless when $j=1$ it takes on the form in question.
	}
	Then we have, without the loss of generality, $i<j$:
	\eq{
	\dotp{A_i}{A_j}&=\ff{1}{n}+0-\ff{j-1}{j(j-1)}+\ff{1}{j(j+1)}	+\cdots+\ff{1}{n(n-1)}\\
			      &\quad=\ff{1}{n}-\ff{1}{j}+\ff{1}{j}-\ff{1}{j+1}	+\cdots+\ff{1}{n-1}-\ff{1}{n}=0
	}
	And
	\eq{
	\dotp{A_j}{A_j}&=\ff{1}{n}+0+\ff{(j-1)^2}{j(j-1)}+\ff{1}{j(j+1)}	+\cdots+\ff{1}{n(n-1)}\\
			      &\quad=\ff{1}{n}+1-\ff{1}{j}+\ff{1}{j}-\ff{1}{j+1}	+\cdots+\ff{1}{n-1}-\ff{1}{n}=1
	}
	Therefore
	\[
	\dotp{A_i}{A_j}=\delta_{ij}
	\]
	showing that the the columns of $A$ are an orthonormal basis of $\mathbb{R}^n$. \\
	This completes the proof.
 }
\es{7}{
	a): \eq{
	&\qquad\quad \left.\begin{gmatrix}
			   0 & 1 & -4 \\
			   1 &2 & -1 \\
			   1 & 1 & 2
	\end{gmatrix}\right|\
	\begin{gmatrix}
	 1 \\
	 1 \\
	 1
	\rowops
	\swap 01
	\end{gmatrix}
	 &&\sim\quad
	\left.\begin{gmatrix}
			   1& 1 & 2 \\
			   1 &2 & -1 \\
			   0 & 1 & -4
	\end{gmatrix}\right|\
	\begin{gmatrix}
	 1 \\
	 1 \\
	 1
	\rowops
	\add[\cdot (-1)]01
	\end{gmatrix} \\[0.5em]
	&\qquad\quad \left.\begin{gmatrix}
			   1& 1 & 2 \\
			   0 &1 & -3 \\
			   0 & 1 & -4
	\end{gmatrix}\right|\
	\begin{gmatrix}
	 1 \\
	 0 \\
	 1
	\rowops
	\add[\cdot (-1)]12
	\end{gmatrix}
	 &&\sim\quad
	\left.\begin{gmatrix}
			   1& 1 & 2 \\
			   0 &1 & -3 \\
			   0 & 0 & -1
	\end{gmatrix}\right|\
	\begin{gmatrix}
	 1 \\
	 0 \\
	 1
	\rowops
	\mult2{-1}
	\end{gmatrix} \\[0.5 em]
	&\qquad\quad \left.\begin{gmatrix}
			   1& 1 & 2 \\
			   0 &1 & -3 \\
			   0 & 0 & 1
	\end{gmatrix}\right|\
	\begin{gmatrix}
	 1 \\
	 0 \\
	 -1
	\rowops
	\add[\cdot 3]21
	\add[\cdot (-2)]20
	\end{gmatrix}
	 &&\sim\quad
	\left.\begin{gmatrix}
			   1& 1 & 0 \\
			   0 &1 & 0 \\
			   0 & 0 & 1
	\end{gmatrix}\right|\
	\begin{gmatrix}
	 3 \\
	 -3 \\
	 -1
	\rowops
	\add[\cdot (-1)]10
	\end{gmatrix} \\[0.5 em]
	&\qquad\quad \left.\begin{gmatrix}
			   1& 0 & 0 \\
			   0 &1 & 0\\
			   0 & 0 & 1
	\end{gmatrix}\right|\
	\begin{gmatrix}
	 6 \\
	 -3 \\
	 -1
	\rowops
	\end{gmatrix}
	}
	We may read off the solution $x_1=6$, $x_2=-3$, $x_3=-1$.\\
	b): By Cramer's rule we have:
	\eq{
	&x_1=\ff{1}{\det A}\det{\mm{1&1&-4\\1&2&-2\\1&1&2}}=6\\
	&x_2=\ff{1}{\det A}\det{\mm{0&1&-4\\1&1&-2\\1&1&2}}=-3\\
	&x_3=\ff{1}{\det A}\det{\mm{0&1&1\\1&2&1\\1&1&1}}=-1\\
	}
	c) Finally, by the transformation in a), the elementary matrix is
	\[
	A^{-1}=S=\mm{5&-6&7\\-3&4&-4\\-1&1&-1}
	\]
	Acting upon $b$ gives us
	\[
	x=Sb=\mm{6\\-3\\-1}
	\]
}
\end{document} 