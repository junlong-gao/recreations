\documentclass[12pt]{article}
\usepackage{geometry}
\usepackage{amsmath} 
\usepackage{amsthm}
\usepackage{amsfonts}
\usepackage{cases}
\usepackage{graphicx}
\usepackage{caption}
\usepackage{subcaption}
\usepackage{gauss}
\usepackage{tikz}
\usetikzlibrary{positioning}
\usetikzlibrary{matrix}
\usepackage{tkz-graph}
\usetikzlibrary{arrows}
% set arrows as stealth fighter jets
\tikzset{>=stealth}
\usepackage{bbold}


\def\dotp#1#2{\langle#1,#2\rangle}
\def\ss#1#2{\sum_{#1=1}^{#2}}
\def\lam{\lambda}
\def\vec#1#2{\{#1_1,#1_2\ldots#1_{#2}\}}
\def\es#1#2{{\bf Exercise #1}\\~{\it Solution:}\\~#2\\[1em]}
\def\ep#1#2{{\bf Exercise #1}\\~{\it Proof:}\\~#2\\[1em]}
\def\inn#1#2{(#1): #2\\[0.5em]}
\def\ff#1#2{\frac{#1}{#2}}
\def\cgu#1{\overline{#1}}
\def\ran#1{{\rm ran}\,#1}
\def\ker#1{{\rm ker}\,#1}
\def\tr#1{{\rm tr}\left(#1\right)}
\def\dotr#1#2{\dotp{#1}{#2}_{\rm tr}}
\def\lr#1#2#3{\left#1#3\right#2}
\newcommand{\R}{\mathbb{R}}
\newcommand{\bb}[1]{\mathbb{#1}}
\newcommand{\eq}[1]{\begin{align*}#1\end{align*}}
\def\sg{\sigma}
\def\ct{\cos{\theta}}
\def\st{\sin{\theta}}
\def\sq#1{\sqrt#1}
\newcommand{\B}{\mathcal{B}}

\linespread{1.4}
\geometry{a4paper,centering,scale=0.8}
\rmfamily 
\normalsize
\setlength{\parindent}{0em}

\begin{document} 
\begin{flushleft}
  Junlong Gao 5133709126\\ 
  Prof.  Hohberger\\ 
  VV285 Assignment 4\\
  \today 
\end{flushleft}

\es{2}{
\[
\det A=\frac{1}{43200},\qquad\det B=24
\]
}
\es{3}{
\eq{
&\det\begin{pmatrix}
2&0&0&0&0&1\\
1&2&0&0&0&1\\
3&1&2&0&0&1\\
0&3&1&2&0&1\\
-5&0&3&1&2&1\\
1&1&1&1&1&1
\end{pmatrix}
=
32\det\begin{pmatrix}
1    &0    &0   &0   &0&1\\
1/2 &1    &0   &0   &0&1\\
3/2 &1/2 &1   &0   &0&1\\
0    &3/2    &1/2&1   &0&1\\
-5/2&0    &3/2   &1/2&1&1\\
1/2 &1/2 &1/2   &1/2&1/2&1
\end{pmatrix}\\
&
=16\det\begin{pmatrix}
1    &0    &0   &0   &0&1\\
1/2 &1    &0   &0   &0&1\\
3/2 &1/2 &1   &0   &0&1\\
0    &3/2    &1/2&1   &0&1\\
-5/2&0    &3/2   &1/2&1&1\\
1 &1 &1   &1 &1&2
\end{pmatrix}\\
&=16\det\begin{pmatrix}
1    &0    &0   &0   &0&0\\
1/2 &1    &0   &0   &0&0\\
3/2 &1/2 &1   &0   &0&0\\
0    &3/2    &1/2&1   &0&0\\
-5/2&0    &3/2   &1/2&1&0\\
1 &1 &1   &1 &1&-59/16
\end{pmatrix}\\
&=-59
}
\ep{4}{
(i)$\Longrightarrow$(ii):
Notice, that
\eq{
0&=f(a_1,\ldots,a_{j-1},a_j+a_k,a_{j+1},\ldots,a_k+a_j,a_{k+1},\ldots,a_p)\\
&\quad=f(a_1,\ldots,a_{j-1},a_j,a_{j+1},\ldots,a_k,a_{k+1},\ldots,a_p)\\
&\qquad+f(a_1,\ldots,a_{j-1},a_j,a_{j+1},\ldots,a_j,a_{k+1},\ldots,a_p)\\
&\qquad+f(a_1,\ldots,a_{j-1},a_k,a_{j+1},\ldots,a_k,a_{k+1},\ldots,a_p)\\
&\qquad+f(a_1,\ldots,a_{j-1},a_k,a_{j+1},\ldots,a_j,a_{k+1},\ldots,a_p)\\
&\qquad=f(a_1,\ldots,a_{j-1},a_j,a_{j+1},\ldots,a_k,a_{k+1},\ldots,a_p)\\
&\quad\qquad+f(a_1,\ldots,a_{j-1},a_k,a_{j+1},\ldots,a_j,a_{k+1},\ldots,a_p)
}
Thus 
\eq{
f(a_1,\ldots,a_{j-1},a_j,a_{j+1},\ldots,a_k,a_{k+1},\ldots,a_p)=-f(a_1,\ldots,a_{j-1},a_k,a_{j+1},\ldots,a_j,a_{k+1},\ldots,a_p)
}
}
(ii)$\Longrightarrow$(iii):
Without the loss of generality, say $a_1=\sum_{i\neq1}^{n}\lam_ia_i$
Then 
\[
f(a_1,\ldots,a_p)=\lam_2f(a_2,\ldots,a_p)+\lam_3f(a_3,\ldots,a_p)+\cdots+\lam_nf(a_n,\cdots,a_p)
\]
And each term in the right has the form by (ii)
\[
f(a_i,\cdots,a_i,\cdots,a_p)=-f(a_i,\cdots,a_i,\cdots,a_p)\Longrightarrow f(a_i,\cdots,a_i,\cdots,a_p)=0
\]
Thereby $f(a_1,\ldots,a_p)=0$\\
(iii)$\Longrightarrow$(i): If $a_i=a_j$ for some $i\neq j$, then the collection $\{a_1,\cdots,a_n\}$ is linear dependent, thus $f(a_1,\ldots,a_p)=0$\\
This completes the proof.
}
\ep{5}{
\def\mat{Mat$(n\times n,\mathbb{R})$}
\def\GL{GL$(n,\mathbb{R})$ }
\def\SL{SL$(n,\mathbb{R})$ }
\def\O{O$(n,\mathbb{R})$}
\def\SO{SO$(n,\mathbb{R})$}
\def\Sp{Sp$(n,\mathbb{R})$ }
\def\1{\mathbb{1}}
Consider $\forall A,B,C\in$ \GL, we have:\\
a) ${\left(AB\right)}^{-1}=B^{-1}A^{-1}$ is invertible, thus it's closed under multiplication and inverse, and\\
b) $(AB)C=A(BC)$ guaranteed by the association law of matrix multiplication.
Thereby \GL forms a group.\\
In order to verify a subgroup, one only have to show that it's closed under multiplication and inverse since the association law is guaranteed by the whole set:\\
i): Consider $\forall A,B\in$ \SL, we have:\\
 a) $\det{AB}=\det{A}\det{B}=1$, thus $AB\in$ \SL.\\
 b) $\det{A^{-1}}=\det{A}=1$, thus $A^{-1}\in$ \SL.\\
ii): Consider $\forall A,B\in$ \O, we have:\\
 a) ${(AB)}^{T}=B^TA^T=B^{-1}A^{-1}={(AB)}^{-1}$, thus $AB\in$ \O.\\
 b) ${(A^{-1})}^{T}={(A^{T})}^{-1}={(A^{-1})}^{-1}$, thus $A^{-1}\in$ \O.\\
 iii): Consider $\forall A,B\in$ \SO, we have:\\
 a) $\det{AB}=\det{A}\det{B}=1$, and ${(AB)}^{T}=B^TA^T=B^{-1}A^{-1}={(AB)}^{-1}$, thus $AB\in$ \SO.\\
 b) $\det{A^{-1}}=\det{A}=1$, and ${(A^{-1})}^{T}={(A^{T})}^{-1}={(A^{-1})}^{-1}$ thus $A^{-1}\in$ \SO.\\
 iv): Consider $\forall A,B\in$ \Sp, we have:\\
 a) ${(AB)}^T=B^{T}A^{T}=\begin{pmatrix}0&\1\\ -\1&0\end{pmatrix}B^{-1}\begin{pmatrix}0&-\1\\ \1&0\end{pmatrix}\begin{pmatrix}0&\1\\ -\1&0\end{pmatrix}A^{-1}\begin{pmatrix}0&-\1\\ \1&0\end{pmatrix}$\\
 $=\begin{pmatrix}0&\1\\ -\1&0\end{pmatrix}{(AB)}^{-1}\begin{pmatrix}0&-\1\\\1&0\end{pmatrix}$, thus $AB\in$ \Sp.\\ 
 b) ${\left(A^{-1}\right)}^{T}={\left(A^{T}\right)}^{-1}={\begin{pmatrix}0&-\1\\\1&0\end{pmatrix}}^{-1}\left(A^{-1}\right)^{-1}{\begin{pmatrix}0&\1\\ -\1&0\end{pmatrix}}^{-1}=\begin{pmatrix}0&\1\\ -\1&0\end{pmatrix}{\left(A^{-1}\right)}^{-1}\begin{pmatrix}0&-\1\\\1&0\end{pmatrix}$, thus $A^{-1}\in$ \Sp.\\[0.5 em]
 \inn{i}{
 	Given $\forall a,b\in\mathbb{R}^2$, the spanned parallelogram is 
	$P=\begin{pmatrix}a&b\end{pmatrix}$
	with the area $|\det P\,|$.\\
	Then the transformed parallelogram is $AP=\begin{pmatrix}Aa&Ab\end{pmatrix}$
	with the area $|\det AP\,|=|\det A\,||\det P\,|=|\det P\,|$.
 }
 \inn{ii}{
 	A similar argument can be made that given $\forall a,b,c\in\mathbb{R}^3$, the spanned parallepiped is
	$P=\begin{pmatrix}a&b&c\end{pmatrix}$
	with the volume $|\det P\,|$.\\
	Then the transformed parallelogram is $AP=\begin{pmatrix}Aa&Ab&Ac\end{pmatrix}$
	with the volume $|\det AP\,|=|\det A\,||\det P\,|=|\det P\,|$.
 }
 \inn{iii}{
 	Notice that $AA^{T}=AA^{-1}=\mathbb{I}$ implies that the columns of $A$ is an orthonormal system (see exercise 6). Since $\forall x\in\mathbb{R}^n$, $Ax$ is merely a linear combination of this orthonormal system with $x_i$ as its coefficients, the norm $\|Ax\|^2=\dotp{Ax}{Ax}=\ss{i}{n}x_i^2=\|x\|^2$, therefore we have
	\[
	\alpha\,(Ax,Ay)=\frac{\dotp{Ax}{Ay}}{\|Ax\|\|Ay\|}=\frac{\dotp{A^TAx}{y}}{\|x\|\|y\|}=\frac{\dotp{x}{y}}{{\|x\|\|y\|}}=\alpha\,(x,y)
	\]
 }
 \ep{6}{
 	\inn{i}{
	(a)$\iff$(b): Let $A_j$ denote the $j$th column of the matrix $A\in$ \O, 
	then the fact that $A^{-1}=A^{T}$ leads to:
	\eq{
	&A^{T}A=A^{-1}A=I\\
	&\iff A_jA_i=\delta_{ij}\\
	}
	}
 }
 }

\end{document} 