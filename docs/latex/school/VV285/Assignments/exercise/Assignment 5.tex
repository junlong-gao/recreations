\documentclass{article}
\usepackage{amsmath}
\usepackage{mathrsfs}
\usepackage{amssymb}
\usepackage{amsfonts}
\usepackage{gauss}
\usepackage[top=1in,bottom=1in,left=1.25in,right=1.25in]{geometry}
\usepackage{dsfont}
\usepackage{amsthm}
\usepackage{graphicx}
\usepackage{multirow}
%一般要用到的package就以上这些,其实我也不知道它们分别有什么用..
\newtheorem{ex}{Exercise}
\title{Assignment 5}
\author{Sun Chen\\ ID:5123709223}
\begin{document}
\maketitle

\section*{Exercise 1.}
\subsection*{i)}
For example, we let
\begin{align*}
f(x)=
\left\lbrace  \begin{gmatrix}
		   \frac{1}{2},~&x=0\\
		   x ,~&x\not=0~and~x\not=1\\
		   \frac{1}{2},~&x=1
\end{gmatrix}\right.\\
\end{align*}
Let closed set A=$[0,1]\subset \mathbb{R}$, then $f(A)=ran~f=(0,1)$, which is not a closed set.

\subsection*{ii)}
For example, we let
\begin{align*}
f(x)=sinx
\end{align*}
Let open set B=$(-2\pi,2\pi)\subset \mathbb{R}$, then $f(B)=ran~f=[-1,1]$, which is not a open set.

\subsection*{iii)}
$f^{-1}(K)$ does NOT have to be compact.
Following is a counter-example.\\
Let m = n = 1, then $f\in C(\mathbb{R}^n,\mathbb{R}^m)=C(\mathbb{R},\mathbb{R})$.
\begin{align*}
f(x)=
\left\lbrace  \begin{gmatrix}
		   \frac{1}{x},~&x\not=0\\
		   0 ,~&x=0
\end{gmatrix}\right.\\
\end{align*}
Then 
\begin{align*}
f^{-1}(x)=
\left\lbrace  \begin{gmatrix}
		   \frac{1}{x},~&x\not=0\\
		   0 ,~&x=0
\end{gmatrix}\right.\\
\end{align*}
Let K = [0,1], then K is compact because it is closed and bounded in a finite-dimensional vector space.\\
But $f^{-1}(K)=\lbrace0\rbrace\cup[1,\infty)$ is not compact because it is not bounded in a finite-dimensional vector space.

\section*{Exercise 2.}
\subsection*{i)}
$A_1$ is neither open or closed. $A_2$ is neither open or closed. $A_3$ is open.
\subsection*{ii)}
The closure of $A_1$ is $\lbrace(x,y):x\geq0,0\leq y\leq 1/x\rbrace$.\\
The closure of $A_2$ is $\lbrace(x,y):1\leq x\leq2,0\leq y\leq 1/x\rbrace$.\\
The closure of $A_3$ is $\lbrace(x,y):1\leq x\leq2,\mid y\mid\leq 1/x\rbrace$.
\subsection*{iii)}
The set of interior points is
\begin{align*}
intA_1=\lbrace(x,y):x>0,0<y<1/x\rbrace
\end{align*}
The set of external points is
\begin{align*}
extA_1=\lbrace(x,y):x<0\rbrace\cup\lbrace(x,y):x\geq0,y<0\rbrace\cup\lbrace(x,y):x>0,y>1/x\rbrace
\end{align*}
The set of boundary points is
\begin{align*}
\partial A_1=\lbrace(x,y):x=0,y\geq0\rbrace\cup\lbrace(x,y):x>0,y=0\rbrace\cup\lbrace(x,y):x>0,y=1/x\rbrace
\end{align*}
The set of interior points is
\begin{align*}
intA_2=\lbrace(x,y):1<x<2,0<y<1/x\rbrace
\end{align*}
The set of external points is
\begin{align*}
extA_2=\lbrace(x,y):x<1\rbrace\cup\lbrace(x,y):x>2\rbrace\cup\lbrace(x,y):1<x<2,y<0\rbrace\cup\lbrace(x,y):1<x<2,y>1/x\rbrace
\end{align*}
The set of boundary points is
\begin{align*}
\partial A_2=\lbrace(x,y):1\leq x\leq2,y=0\rbrace\cup\lbrace(x,y):x=1,0<y<1\rbrace\cup\\
\lbrace(x,y):x=2,0<y<\frac{1}{2}\rbrace\cup\lbrace(x,y):1\leq x\leq2,y=1/x\rbrace
\end{align*}
The set of interior points is
\begin{align*}
intA_3=\lbrace(x,y):1<x<2,0<\mid y\mid<1/x\rbrace
\end{align*}
The set of external points is
\begin{align*}
extA_3=\lbrace(x,y):x<1\rbrace\cup\lbrace(x,y):x>2\rbrace\cup\lbrace(x,y):1<x<2,\mid y\mid>1/x\rbrace
\end{align*}
The set of boundary points is
\begin{align*}
\partial A_3=\lbrace(x,y):x=1,-1<y<1\rbrace\cup\lbrace(x,y):x=2,-\frac{1}{2}<y<\frac{1}{2}\rbrace\cup\\
\lbrace(x,y):1\leq x\leq2,y=0\rbrace\cup\lbrace(x,y):1\leq x\leq2,\mid y\mid=1/x\rbrace
\end{align*}
\subsection*{iv)}
The boundary points of $A_1$ that are part of the set is
\begin{align*}
\lbrace(x,y):x>0,y=0\rbrace\cup\lbrace(x,y):x>0,y=1/x\rbrace
\end{align*}
The boundary points of $A_2$ that are part of the set is
\begin{align*}
\lbrace(x,y):x=1,0<y<1\rbrace\cup\lbrace(x,y):x=2,0<y<\frac{1}{2}\rbrace
\end{align*}
The boundary points of $A_2$ that are part of the set is $\emptyset$.

\section*{Exercise 3.}
\subsection*{i)}
$(\Rightarrow)$ We will show by contradiction.\\ Suppose that $x \in \overline{M}$, $d(x,M)=\lambda>0$. Then we cannot find an open ball for all $y\in M$ such that
\begin{align*}
B_\varepsilon(y):=\lbrace x\in \mathbb{R}^n:\Vert x-y\Vert<\frac{\lambda}{2}\rbrace
\end{align*}
Because $x$ is neither an interior point nor a boundary point of $\overline{M}$, so $x$ is an exterior point of $\overline{M}$ and $x\not\in \overline{M}$.\\
So, 
\begin{align*}
d(x,M)=0
\end{align*}
$(\Leftarrow)$ $d(x,M)=$ inf $\Vert x-y\Vert=0$. If $x\in M$, then $x\in \overline{M}$.\\
If $x\not\in M$, $x\in \mathbb{R}^n \setminus M $. Then for all $\varepsilon>0$ there exists a $y$ such that
\begin{align*}
\Vert x-y\Vert<\varepsilon
\end{align*}
So for every $\varepsilon>0$, $B_\varepsilon(x)\cap M\not= \varnothing$ and $B_\varepsilon(x)\cap (\mathbb{R}^n \setminus M)\not= \varnothing$.\\
So $x$ is a boundary point of M and $x\in \overline{M}$.

\subsection*{ii)}
From iii), $\exists x_0\in \overline{M}$ such that
\begin{align*}
\mathrm{inf}\Vert x-y\Vert = \Vert x-x_0\Vert
\end{align*}
Since $x_0$ is fixed, for any $x\in\mathbb{R}^n$
\begin{align*}
\Vert x-x_0\Vert=\mathrm{inf}\Vert x-y\Vert
\end{align*}
For some $\delta>0$ and $a\in \mathbb{R}^n$, we let
\begin{align*}
\Vert x-a\Vert<\delta
\end{align*}
Then,
\begin{align*}
\Vert \mathrm{inf}\Vert x-y\Vert-\mathrm{inf}\Vert a-y\Vert\Vert&=\Vert \Vert x-x_1\Vert-\Vert a-x_2\Vert\Vert \\&\leq\Vert (x-a)-(x_1-x_2)\Vert\\ &\leq\Vert(x-a)\Vert+\Vert(x_1-x_2)\Vert\\ &<\delta +\Vert(x_1-x_2)\Vert:=\varepsilon
\end{align*}
Hence, the function is continuous.

\subsection*{iii)}
We set a closed ball such that
\begin{align*}
B_\varepsilon(a):=\lbrace x\in \mathbb{R}^n:\Vert x-a\Vert\leq \varepsilon\rbrace
\end{align*}
We set the ball $B \cap M \not= \varnothing$.\\
We know that $B \cap M$ is closed (because all boundary points are contained), so $\Vert x-y\Vert\leq \varepsilon$, $y\in B \cap M $.\\
So the set $P:=\lbrace p:p=x-y,y\in B \cap M\rbrace$ is closed and bounded, and compact.\\
Then, we set a sequence 
\begin{align*}
p_n=x-y_n, \Vert p_n\Vert=\Vert x-y_n\Vert
\end{align*}
,and there exists some subsequence such that $x$ $\Vert x-y_n^{'}\Vert\rightarrow d(x,M)$ when $n\rightarrow \infty$.\\
Since $P$ is compact, the limit of $x-y_n^{'}$ is contained in $P$.\\
Set the limit to be $x-y$. So, $\exists y\in M$ such that 
\begin{align*}
d(x,M)=\Vert x-y\Vert
\end{align*}

\subsection*{iv)}
From iii), $\forall x\in K$, there exists some corresponding $y_n\in M $ such that
\begin{align*}
d(x,M)=\Vert x-y_n\Vert<\infty
\end{align*}
Since $K$ is compact, $\forall x_n\in K $, there exists a convergent subsequence $x_n^{'}$ converge to some $x\in K$.\\
Since the map is continuous,
\begin{align*}
x_n^{'}\rightarrow x^{'} \Rightarrow \Vert x_n^{'}-y_n^{'}\Vert\rightarrow \Vert x{'}-y{'}\Vert
\end{align*}
So, $\Vert x-y\Vert$ is compact.\\
Then we set a sequence $\Vert x_n-y_n\Vert$ that converges to $\mathrm{inf}\Vert x-y\Vert$ when $n\rightarrow \infty$, and such limit exists, i.e., there exists some $x_0,y_0$ such that
\begin{align*}
\Vert x_0-y_0\Vert = \mathrm{inf}\Vert x-y\Vert
\end{align*} 
When $K$ is not compact, we let
\begin{align*}
K:=\lbrace (x,y)\in \mathbb{R}^2:x>0,y=1/x\rbrace\\ 
M:=\lbrace (x,y)\in \mathbb{R}^2:x>0,y=0\rbrace
\end{align*}
So $d(K,M)=0$ but we cannot find that $x_0,y_0$ because the limit reaches when $x\rightarrow\infty$.




\section*{Exercise 4.}
\subsection*{i)}
The second inequality holds.\\
Proof:\\
We take the norm in $\mathbb{R}$ to be $\mid \cdot\mid$.\\
According to 2.2.27. Corollary of The Mean Value Theorem,
\begin{align*}
\parallel f\parallel_1=\int^1_0\mid f(x)\mid dx=\mid \int^1_0 f(x)dx\mid\leq\mid1-0\mid\mathop{sup}_{x\in[0,1]}\mid f(x)\mid=\mathop{sup}_{x\in[0,1]}\mid f(x)\mid=\parallel f\parallel_\infty
\end{align*}
So, $\parallel f\parallel_1\leq C\parallel f\parallel_\infty$, where C = 1.
\subsection*{ii)}
Let $f^n(x)=x^n$, where $n\in \mathbb{R}$.\\
Then
\begin{align*}
\|f_n\|_1=\int^1_0\mid f_n(x)\mid dx=\int^1_0\mid x^n\mid dx=\mid \int^1_0 x^n dx\mid=\frac{1}{n+1}x^{n+1}\mid^1_0=\frac{1}{n+1}
\end{align*}
As $n\rightarrow \infty$, $\|f_n\|_1\rightarrow0$.\\
But
\begin{align*}
\|f_n\|_{\infty}=\mathop{sup}_{x\in[0,1]}\mid f_n(x)\mid=\mathop{sup}_{x\in[0,1]}\mid x^n\mid=1
\end{align*}
As $n\rightarrow \infty$, $\|f_n\|_{\infty}\not\rightarrow0$

\section*{Exercise 5.}
Because $\overline{B_1(0)}=\lbrace(a_n)\in l^1:\displaystyle{\sum_0^\infty \mid a_n\mid\leq1}\rbrace$, it is bounded as every element is bounded.\\
Next, we want to prove $\overline{B_1(0)}$ is closed by proving $l^1\backslash\overline{B_1(0)}$ is open.\\
\begin{align*}
l^1\backslash\overline{B_1(0)}=\lbrace(a_n)\in l^1:\displaystyle{\sum_0^\infty \mid a_n\mid>1}\rbrace
\end{align*}
For some set $(a_n)\in l^1\backslash\overline{B_1(0)}$ that $\|(a_n)\|_1$ = C $>$ 1.\\
Then 
\begin{align*}
B_{\varepsilon}((a_n)):=\lbrace(x_n)\in l^1:\Vert (x_n)-(a_n)\Vert <\frac{C-1}{2}
\end{align*}
And $\displaystyle{\sum^\infty_{n=0}}\vert x_n\vert>\frac{C+1}{2}>1$, so $l^1\backslash\overline{B_1(0)}$ is open. And $\overline{B_1(0)}$ is closed.\\
Let $e_n=(0,0,\cdots,0,1,0,\cdots)$, where 1 is in the nth position.\\
Then
\begin{align*}
\| e_n\|=\displaystyle{\sum_0^\infty} \mid a_n\mid=1
\end{align*}
So $e_n\in\overline{B_1(0)}$.\\
Since $a_{n+1}-a_n=(0,0,\cdots,0,-1,1,\cdots)\not\rightarrow0$ as $n\rightarrow\infty$. \\
So ($e_n$) does not have a convergent subsequence.\\
Hence $\overline{B_1(0)}$ is not compact.




\end{document}